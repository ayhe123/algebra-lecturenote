\documentclass{ctexart}
\usepackage[bgcolor]{lecturenote}

\title{第1,2章笔记和习题}

\begin{document}
\maketitle
\section{第1章笔记}
\subsection{映射(对应教材1.5节)}
规定两个映射相等当且仅当定义域, 值域和对应关系都相等的原因之一是为了严格地进行映射的复合.
\begin{example}
    考察映射 $f:\mathbb{R}\to\{-1,0,1\},g:\mathbb{R}\to\mathbb{N}$,
    \[f(x)=g(x)=\begin{cases}
        1, & x>0, \\
        0, & x=0, \\
        -1, & x<0. \\
    \end{cases}\]

    $f$ 可以与 $\{-1,0,1\}$ 上的映射 $h$ 复合, 但 $g$ 不能与 $h$ 复合, $\because h$ 在 $\mathbb{N}\backslash\{-1,0,1\}$ 上没有定义.
\end{example}
\begin{theorem}[书上第2小节的引理的推广]
    (1) 映射 $f:X\to Y$ 是单射当且仅当 $\exists$ 满射 $g:Y\to X,gf=e_X$. (2) 映射 $f:X\to Y$ 是满射当且仅当 $\exists$ 单射 $g:Y\to X,fg=e_Y$.
\end{theorem}
\begin{proof}
    ($\Leftarrow$) 由书上第2小节的引理得.
    
    ($\Rightarrow$) (1) 设 $f$ 是单射, 定义映射
    \[g:\begin{array}{rcll}
    Y & \to & X \\
    f(x) & \mapsto & x & (\forall f(x)\in\operatorname{Im}f) \\
    y & \mapsto & x_0 & (\forall y\in Y\backslash\operatorname{Im}f)
    \end{array},\]

    其中 $x_0\in X$.

    $\because f$ 是单射, $\therefore\forall y\in\operatorname{Im}f,\exists!x\in X$ 使得 $y=f(x)\Rightarrow x=g(y).\therefore g$ 是确切定义(well-defined)的.

    容易验证 $gf=e_X$(到这里就可以用引理得到 $g$ 是满射, 下面采用与引理不同的方法证明).

    $\because f$ 是映射, $\therefore\forall x\in X,\exists y\in Y$ 使得 $y=f(x)\Rightarrow x=g(y)$. $\therefore X\subset\operatorname{Im}g$.

    $\because g:Y\to X,\therefore\operatorname{Im}g\subset X.\therefore\operatorname{Im}g=X$, 即 $g$ 是满射.

    (2) 设 $f$ 是满射, 则 $\operatorname{Im}f=Y$.

    $\therefore\forall y\in Y,\exists x\in X$ 使得 $f(x)=y$.

    对于 $\forall y\in Y$, 取定满足 $f(x_y)=y$ 的 $x_y\in X$, 定义
    \[g:\begin{array}{rcl}
    Y & \to & X \\
    y & \to & x_y
    \end{array}.\]

    容易验证 $fg=e_Y$(到这里就可以用引理得到 $g$ 是单射, 下面采用与引理不同的方法证明).

    $\because f$ 是映射, $\therefore\forall y,y'\in Y,y\neq y'\Rightarrow x_y\neq x_{y'}.\therefore g$ 是单射.
\end{proof}
书上定理2的推论没有证明双射的合成是双射. 这里证明一下.
\begin{theorem}[书上定理2推论的补充]
    (1) 有限个单射的复合是单射.
    
    (2) 有限个满射的复合是满射.
\end{theorem}
\begin{proof}
    只需证明: 若 $f:X'\to X'',g:X\to X'$ 是单射/满射, 则 $f\circ g:X\to X''$ 是单射/满射, 有限个的结论可以用归纳法得到.

    (1) $\because g$ 是单射, $\therefore\forall x_1,x_2\in X,x_1\neq x_2\Rightarrow g(x_1)\neq g(x_2)$.
    
    $\because f$ 是单射, $\therefore\forall x_1',x_2'\in X',x_1'\neq x_2'\Rightarrow f(x_1')\neq f(x_2')$.

    $\therefore\forall x_1,x_2\in X,g(x_1),g(x_2)\in X',x_1\neq x_2\Rightarrow g(x_1)\neq g(x_2)\Rightarrow f(g(x_1))\neq f(g(x_2))$.

    $\therefore f\circ g$ 是单射.

    (2) $\because f$ 是满射, $\therefore\forall x''\in X'',\exists x'\in X'$ 使得 $f(x')=x''$.

    $\because g$ 是满射, $\therefore\forall x'\in X',\exists x\in X$ 使得 $g(x)=x'$.

    $\therefore\forall x''\in X'',\exists x\in X$ 使得 $f(g(x))=x''$.

    $\therefore f\circ g$ 是满射.
\end{proof}
\subsection{置换(对应教材1.8节)}
在这里补充证明几个定理.
\begin{definition}
    设映射 $f:X\to Y$, $X_0\subset X$, 称映射
    \[\bar{f}:\begin{array}{rcl}
        X_0 & \to & Y \\
        x & \to & f(x) \\
    \end{array}\]

    为 $f$ 在 $X_0$ 上的\textbf{限制}, 记作 $f|_{X_0}$.
\end{definition}
\begin{lemma}\label{l1.1}
    设有限集 $X$, $f:X\to Y$ 是单射, 则 $|\operatorname{Im}f|=|X|$.
\end{lemma}
\begin{proof}
    对 $|X|$ 用数学归纳法. 当 $|X|=1$ 时, $\operatorname{Im}f=\{f(x)\},|\operatorname{Im}f|=1$.

    假设对 $\forall$ 基数为 $n$ 的集合 $X$ 和 $X\to Y$ 的映射 $f$ 都有 $|\operatorname{Im}f|=|X|$. 对基数为 $n+1$ 的集合 $X$, 考虑 $X$ 的一个基数为 $n$ 的子集 $X_0$ 和 $f$ 在 $X_0$ 上的限制 $\bar{f}=f|_{X_0}$. 有 $f(X_0)=\operatorname{Im}\bar{f}$.

    设 $X=\{x_0\}\cup X_0$, 则 $\operatorname{Im}f=\{f(x_0)\}\cup f(X_0)=\{f(x_0)\}\cup\operatorname{Im}\bar{f}$. $\therefore|\operatorname{Im}f|=|\operatorname{Im}\bar{f}|+1$.

    由归纳假定, $|\operatorname{Im}\bar{f}|=|X_0|$. $\therefore|\operatorname{Im}f|=|X_0|+1=|X|$.
\end{proof}
\begin{theorem}
    设有限集 $X,Y$ 满足 $|X|=|Y|<\infty$, $f:X\to Y$ 是单射/满射, 则 $f$ 是双射.
\end{theorem}
\begin{proof}
    设 $f$ 是单射, 由引理 \ref{l1.1} 得 $|\operatorname{Im}f|=|X|=|Y|.\because\operatorname{Im}f\subset Y,\therefore\operatorname{Im}f=Y$, 即 $f$ 是满射.

    假设 $f$ 是满射而不是单射, 由 $f$ 是满射得 $\operatorname{Im}f=Y$, 由 $f$ 不是单射得 $|\operatorname{Im}f|<|X|=|Y|$, 与 $\operatorname{Im}f=Y$ 矛盾.
\end{proof}

当 $X,Y$ 是无限集时定理 1 不一定成立. 比如:
\[f:\begin{array}{rcl}
\mathbb{N} & \to & \mathbb{N} \\
n & \to & n+1
\end{array}\]

是单射不是满射($0\notin\operatorname{Im}f$),
\[g:\begin{array}{rcl}
\mathbb{N} & \to & \mathbb{N} \\
n & \to & \dfrac{n+(-1)^nn}{4}
\end{array}\]

是满射不是单射(所有的奇数的像都是 $0$).

从最简单的例子开始讨论置换. $\Omega$ 中所有元素不动的置换即恒等映射 $e$. $\because$ 不存在一个双射 $\sigma\in S_n$ 只改变 $\Omega$ 中 $1$ 个元素, $\therefore$ 考虑只改变 $\Omega$ 中 $2$ 个元素的 $\sigma\in S_n$.
\begin{lemma}
    如果 $\sigma\in S_n$ 只改变 $\Omega$ 中 $2$ 个元素, 那么 $\exists i,j\in\Omega,i\neq j$ 使得 $\sigma(i)=j,\sigma(j)=i,\forall k\in\Omega\backslash\{i,j\},\sigma(k)=k$.
\end{lemma}
\begin{proof}
    设 $\sigma(i)=j\neq i,\sigma(j)=k$, 则 $k\neq j$, 否则
    \begin{align*}
        & \sigma(\sigma(i))=\sigma(j)=k=j=\sigma(i) \\
        \Rightarrow & \sigma^{-1}(\sigma(\sigma(i)))=\sigma^{-1}(\sigma(i)) \\
        \Rightarrow & e(\sigma(i))=e(i) \\
        \Rightarrow & \sigma(i)=i,
    \end{align*}

    与 $\sigma(i)=j\neq i$ 矛盾.

    假设 $\sigma(k)=k$, 则
    \begin{align*}
        & \sigma(\sigma(j))=\sigma(k)=k=\sigma(j) \\
        \Rightarrow & \sigma^{-1}(\sigma(\sigma(j)))=\sigma^{-1}(\sigma(j)) \\
        \Rightarrow & e(\sigma(j))=e(j) \\
        \Rightarrow & \sigma(j)=j,
    \end{align*}
    
    与 $\sigma(j)\neq j$ 矛盾. $\therefore\sigma(k)\neq k$.
    
    假设 $k\neq i$, 则 $\sigma$ 改变了 $\Omega$ 中至少 $3$ 个元素 $i,j,k$, 与 $\sigma\in S_n$ 只改变 $\Omega$ 中 $2$ 个元素矛盾.

    $\therefore\sigma(j)=k=i$.

    $\because\sigma\in S_n$ 只改变 $\Omega$ 中 $2$ 个元素, $\therefore\forall k\in\Omega\backslash\{i,j\},\sigma(k)=k.$
\end{proof}
\begin{definition}
    设 $X=\{i_1,i_2,\cdots,i_k\}\subset\Omega,\{j_1,j_2,\cdots,j_k\}=\Omega\backslash X$, 称
    \[\sigma=\begin{pmatrix}
        j_1 & j_2 & \cdots & j_n & i_1 & i_2 & \cdots & i_{k-1} & i_k \\
        j_1 & j_2 & \cdots & j_n & i_2 & i_3 & \cdots & i_k & i_1 \\
    \end{pmatrix}\]

    为一个\textbf{循环置换}, 简称\textbf{循环}, 记作 $(i_1i_2\cdots i_k)$, $k$ 称为 $\sigma$ 的长度.

    规定长度为 $1$ 的循环为 $e$.
\end{definition}
\begin{definition}
    两个循环 $(i_1i_2\cdots i_p),(j_1j_2\cdots j_q)$ 是\textbf{无关的}当且仅当
    \[\{i_1i_2\cdots i_p\}\cap\{j_1j_2\cdots j_q\}=\varnothing.\]
\end{definition}

我们希望能通过简单的置换了解复杂的置换. 下面我们像把整数分解为素数的乘积一样将置换分解为一些置换的乘积, 这就是书中的定理 1.

为了证明书中的定理 1, 我们证明一个引理.
\begin{lemma}\label{l1.3}
    无关的循环的乘积是交换的.
\end{lemma}
\begin{proof}
    设 $\tau=(i_1i_2\cdots i_p),\sigma=(j_1j_2\cdots j_q)$ 是无关的, 对应的集合为
    
    \[I=\{i_1,i_2,\cdots,i_p\},\quad J=\{j_1,j_2,\cdots,j_q\}.\]

    由定义, $I\cap J=\varnothing$. $\therefore I,J,\Omega\backslash(I\cup J)$ 构成 $\Omega$ 的一个划分.

    考察 $x\in\Omega$.
    \[\tau(\sigma(x))=\begin{cases}
        x, & x\notin I\cup J, \\
        \tau(x), & x\in I, \\
        \sigma(x), & x\in J, \\
    \end{cases}\]
    \[\sigma(\tau(x))=\begin{cases}
        x, & x\notin I\cup J, \\
        \tau(x), & x\in I, \\
        \sigma(x), & x\in J. \\
    \end{cases}\]
    
    $\therefore\tau\sigma=\sigma\tau$.
\end{proof}

容易看出相交的循环的乘积不一定是交换的, 比如 $(1\ 2)(2\ 3)\neq(2\ 3)(1\ 2)$.
\begin{theorem}[书上的定理1]
    $S_n$ 中的任一置换 $\pi\neq e$ 都是长度 $\geq2$, 不相交的循环的乘积. 这一分解式在不考虑循环乘积的顺序(因为循环的乘积是交换的)的情况下是唯一的. 
\end{theorem}
\begin{proof}
    对 $m=|\Omega|$ 用数学归纳法. $S_2$ 的情况是显然的.

    假设 $\forall t<m,S_{t}$ 中任一置换 $\sigma\neq e$ 都是长度 $\geq2$, 不相交的循环的乘积. 考虑 $m$ 元集合 $\Omega$ 中的元素 $1$ 和 $\sigma\in S_m$.

    如果 $\sigma(1)=1$, 则 $\sigma$ 可以看成 $\Omega\backslash\{1\}$ 的置换 $\sigma'$. 由归纳假定, $\sigma'$ 是长度 $\geq2$, 不相交的循环的乘积. $\therefore\sigma$ 是长度 $\geq2$, 不相交的循环的乘积.

    设 $\sigma(1)\neq1$. $\because\exists k\leq m,\sigma^k(1)=1,\therefore\exists s$ 满足 $1<s\leq m$ 使得 $\sigma^s(1)=1$ 但 $\forall t<s,\sigma^t(1)\neq1$. 设
    \[i_1=1,\ i_2=\sigma(1),\ i_3=\sigma^2(1),\ \cdots,\ i_s=\sigma^{s-1}(1),\ \sigma_1=(i_1i_2\cdots i_s),\]
    
    则 $\sigma^{-1}_1=(i_si_{s-1}\cdots i_1)$.

    考虑 $\sigma_1^{-1}\sigma$. 设 $i_{s+1}=\sigma(i_s)=1$, 则
    \[\sigma_1^{-1}\sigma(i_r)=\sigma_1^{-1}(i_{r+1})=i_r\quad(r=1,2,\cdots,s).\]

    $\therefore$ 设 $I=\{i_1,i_2,\cdots,i_p\}$, 则 $\sigma_1^{-1}\sigma$ 可以看成是 $\Omega\backslash I$ 的置换.

    $\because|\Omega\backslash I|<|\Omega|=m$, 由归纳假定,
    \[\sigma_1^{-1}\sigma=\sigma_2\sigma_3\cdots\sigma_k,\]

    其中 $\sigma_2,\sigma_3,\cdots,\sigma_k$ 是 $\Omega\backslash I$ 上的长度 $\geq2$, 不相交的循环的乘积.

    $\because\sigma_1$ 是 $I$ 上的循环, $\therefore\sigma_1$ 与 $\sigma_2,\sigma_3,\cdots,\sigma_k$ 不相交. $\therefore\sigma_1,\sigma_2,\cdots,\sigma_k$ 不相交. $\therefore$
    \[\sigma=\sigma_1\sigma_2\sigma_3\cdots\sigma_k,\]

    即 $\sigma$ 是长度 $\geq2$, 不相交的循环的乘积.

    下面证明上述分解是唯一的. 设
    \begin{equation}\label{eq1.1}
        \sigma=\sigma_1\sigma_2\cdots\sigma_k=\tau_1\tau_2\cdots\tau_r\ (k,r\leq t)
    \end{equation}

    是 $\sigma$ 的两个长度 $\geq2$, 不相交的循环的分解.

    对 $t$ 用数学归纳法. 当 $t=1$ 时有
    \[\sigma=\sigma_1=\tau_1.\]

    假设 $\forall k,r\leq t-1$, 若
    \[\sigma=\sigma_1\sigma_2\cdots\sigma_k=\tau_1\tau_2\cdots\tau_r\]

    是 $\sigma$ 的两个长度 $\geq2$, 不相交的循环的分解, 则有 $k=r$ 且 $\sigma_p=\tau_p\ (p=1,2,\cdots,k)$. 下面证明 (\ref{eq1.1}) 式也满足 $k=r$ 且 $\sigma_p=\tau_p\ (p=1,2,\cdots,k)$.

    $\because\forall p,p',q,q'$ 满足 $1\leq p,p'\leq k,1\leq q,q'\leq r,\sigma_p,\sigma_{p'},\tau_q,\tau_{q'}$ 不相交, $\therefore$ 设 $i$ 满足 $\sigma(i)\neq i$, 则 $\exists!a,b$ 满足 $1\leq a\leq k,1\leq b\leq r$, 有
    \[\sigma(i)=\sigma_a(i)=\tau_b(i)\neq i.\]

    $\because\sigma_1,\sigma_2,\cdots,\sigma_k$ 不相交, 由引理 \ref{l1.3} 得
    \begin{align*}
        \sigma_{a}\sigma & =\sigma_{a}\sigma_1\sigma_2\cdots\sigma_{a-1}\sigma_{a}\sigma_{a+1}\cdots\sigma_k \\
        & =\sigma_{a}\sigma_1\sigma_2\cdots\sigma_{a-1}\sigma_{a+1}\cdots\sigma_k\sigma_{a} \\
        & =\sigma_1\sigma_2\cdots\sigma_{a-1}\sigma_{a}\sigma_{a+1}\cdots\sigma_k\sigma_{a}=\sigma\sigma_{a}.
    \end{align*}

    同理得 $\tau_b\sigma=\sigma\tau_b$.

    $\therefore\forall u\geq1$,
    \[\sigma^2(i)=\sigma\sigma_a(i)=\sigma_a\sigma(i)=\sigma_a^2(i),\]
    \[\sigma^2(i)=\sigma\tau_b(i)=\tau_b\sigma(i)=\tau_b^2(i),\]
    \[\sigma^3(i)=\sigma^2\sigma_a(i)=\sigma_a\sigma^2(i)=\sigma_a^3(i),\]
    \[\sigma^3(i)=\sigma^2\tau_b(i)=\tau_b\sigma^2(i)=\tau_b^3(i),\]
    \[\cdots\]
    \[\sigma^u(i)=\sigma^{u-1}\sigma_a(i)=\sigma_a\sigma^{u-1}(i)=\sigma_a^u(i),\]
    \[\sigma^u(i)=\sigma^{u-1}\tau_b(i)=\tau_b\sigma^{u-1}(i)=\tau_b^u(i).\]

    $\therefore\forall u\geq1,\sigma_a^u(i)=\tau_b^u(i)$.

    $\because\exists s>1$ 使得 $\sigma_a^s(i)=i$ 但 $\forall t<s,\sigma_a^t(i)\neq i$, $\therefore$
    \[\sigma_a=(i\ \sigma_a(i)\ \sigma_a^2(i)\cdots\sigma_a^{s-1}(i)).\]

    $\because\forall u\geq1,\sigma_a^u(i)=\tau_b^u(i),\therefore$
    \[\tau_b=(i\ \tau_b(i)\ \tau_b^2(i)\cdots\tau_b^{s-1}(i))=(i\ \sigma_a(i)\ \sigma_a^2(i)\cdots\sigma_a^{s-1}(i))=\sigma_a.\]

    $\therefore\tau_b^{-1}=\sigma_a^{-1}$. 由式 (\ref{eq1.1}) 得
    \[\sigma_a^{-1}\sigma_1\sigma_2\cdots\sigma_k=\tau_b^{-1}\tau_1\tau_2\cdots\tau_r.\]

    $\because\sigma_1,\sigma_2,\cdots,\sigma_{a-1},\sigma_a^{-1}$ 不相交, $\tau_1,\tau_2,\cdots,\tau_{b-1},\tau_b^{-1}$ 不相交, $\therefore$
    \begin{align*}
        \sigma_a^{-1}\sigma_1\sigma_2\cdots\sigma_{a-1}\sigma_a\cdots\sigma_k & =\sigma_1\sigma_2\cdots\sigma_{a-1}\sigma_a^{-1}\sigma_a\cdots\sigma_k \\
        & =\sigma_1\sigma_2\cdots\sigma_{a-1}\sigma_{a+1}\cdots\sigma_k,
    \end{align*}
    \begin{align*}
        \tau_b^{-1}\tau_1\tau_2\cdots\tau_{b-1}\tau_b\cdots\tau_r & =\tau_1\tau_2\cdots\tau_{b-1}\tau_b^{-1}\tau_b\cdots\tau_r \\
        & =\tau_1\tau_2\cdots\tau_{b-1}\tau_{b+1}\cdots\tau_r.
    \end{align*}

    $\therefore$
    \[\sigma_1\sigma_2\cdots\sigma_{a-1}\sigma_{a+1}\cdots\sigma_k=\tau_1\tau_2\cdots\tau_{b-1}\tau_{b+1}\cdots\tau_r.\]

    由归纳假定, $k-1=r-1\Rightarrow k=r$, (加上之前证明的 $\sigma_a=\tau_b$) $\sigma_p=\tau_p\ (p=1,2,\cdots,k)$.
\end{proof}

书上引入了\textbf{轨道}的概念来证明这一题.

下面给出书上引理 2 的另一种证明.
\begin{lemma}[书上的引理 2]
    交换斜对称函数 $f(x_1,x_2,\cdots,x_n)$ 的任意两个变量的位置, $f$ 变号.
\end{lemma}
\begin{proof}
    设 $\sigma=(i,j)\ (i<j),f$ 是斜对称函数. 只需证明
    \[\sigma\circ f(x_1,x_2,\cdots,x_n)=-f(x_1,x_2,\cdots,x_n).\]

    考虑等式
    \[\sigma=(i,i+1)(i+1,i+2)\cdots(j-1,j)\cdot(j-1,j-2)(j-2,j-3)\cdots(i+1,i).\]

    设 $\sigma_k=(k,k+1)\ (k=1,2,\cdots,m-1)$, 则
    \[\sigma=\sigma_i\sigma_{i+1}\cdots\sigma_{j-1}\sigma_{j-2}\cdots\sigma_i.\]

    $\therefore$
    \[\sigma\circ f(x_1,x_2,\cdots,x_n)=(\sigma_i\sigma_{i+1}\cdots\sigma_{j-1}\sigma_{j-2}\cdots\sigma_i)\circ f(x_1,x_2,\cdots,x_n).\]

    由书上的引理 1,
    \[(\sigma_i\sigma_{i+1}\cdots\sigma_{j-1}\sigma_{j-2}\cdots\sigma_i)\circ f(x_1,x_2,\cdots,x_n)=(\sigma_i\sigma_{i+1}\cdots\sigma_{j-1}\sigma_{j-2}\cdots\sigma_{i-1})\circ(\sigma_i\circ f)(x_1,x_2,\cdots,x_n).\]

    由定义,
    \[(\sigma_i\circ f)(x_1,x_2,\cdots,x_n)=-f(x_1,x_2,\cdots,x_n),\]

    $\therefore$
    \begin{align*}
        & (\sigma_i\sigma_{i+1}\cdots\sigma_{j-1}\sigma_{j-2}\cdots\sigma_{i-1})\circ(\sigma_i\circ f)(x_1,x_2,\cdots,x_n)\\
        = & \ -(\sigma_i\sigma_{i+1}\cdots\sigma_{j-1}\sigma_{j-2}\cdots\sigma_{i-1})\circ f(x_1,x_2,\cdots,x_n) \\
        = & \ -(\sigma_i\sigma_{i+1}\cdots\sigma_{j-1}\sigma_{j-2}\cdots\sigma_{i-2})\circ(\sigma_{i-1}\circ f)(x_1,x_2,\cdots,x_n) \\
        = & \ (-1)^2(\sigma_i\sigma_{i+1}\cdots\sigma_{j-1}\sigma_{j-2}\cdots\sigma_{i-2})\circ f(x_1,x_2,\cdots,x_n) \\
        = & \ \cdots \\
        = & \ (-1)^{2(j-i)-1}f(x_1,x_2,\cdots,x_n) \\
        = & \ -f(x_1,x_2,\cdots,x_n).\qedhere
    \end{align*}
\end{proof}
\section{第2章笔记}
补充证明一个定理.
\begin{theorem}
    若 $A$ 非退化, 则 ${}^t(A^{-1})=({}^tA)^{-1}$.
\end{theorem}
\begin{proof}
    在 $AA^{-1}=E$ 两边转置得
    \[{}^tA{}^t(A^{-1})=E\Rightarrow{}^t(A^{-1})=({}^tA)^{-1}.\]
\end{proof}

下面讨论一下线性映射的核与像.

设矩阵 $A=(A^{(1)},A^{(2)},\cdots,A^{(n)})$ 对应的线性映射为 $\phi_A$.

$\because\forall X=[x_1,x_2,\cdots,x_n]\in\mathbb{R}^n,\phi_A(X)=x_1A^{(1)}+x_2A^{(2)}+\cdots+x_nA^{(n)}\in\left<A^{(1)},A^{(2)},\cdots,A^{(n)}\right>$,

$\therefore\operatorname{Im}\phi_A\subset\left<A^{(1)},A^{(2)},\cdots,A^{(n)}\right>$.

$\because\forall Y\in\left<A^{(1)},A^{(2)},\cdots,A^{(n)}\right>,\exists y_1,y_2,\cdots,y_n$ 使得 $Y=y_1A^{(1)}+y_2A^{(2)}+\cdots+y_nA^{(n)}$,

$\therefore\exists Y_0=(y_1,y_2,\cdots,y_n)$ 使得 $\phi_A(Y_0)=Y$.

$\therefore\operatorname{Im}\phi_A\supset\left<A^{(1)},A^{(2)},\cdots,A^{(n)}\right>$.

$\therefore\operatorname{Im}\phi_A=\left<A^{(1)},A^{(2)},\cdots,A^{(n)}\right>,\dim\operatorname{Im}\phi_A=\operatorname{rank}A$.

上述讨论对于实际的计算也有帮助. 比如求 $\left<X_1,\cdots,X_n\right>$ 的维数, 只需要求矩阵 $(X_1,\cdots,X_n)$ 的秩, 而后者可以通过行或列变换得到.

按照定义,
\begin{align*}
    \ker\phi_A & =\{X\in\mathbb{R}^n|\phi_A(X)=0\} \\
    & =\{X\in\mathbb{R}^n|AX=0\}.
\end{align*}

$\therefore\ker\phi_A$ 是方程 $AX=0$ 的解空间.

由第 3 节第 8 小节的定理 7 得
\[\dim\ker\phi_A=n-\operatorname{rank}A.\]

$\therefore$
\[\dim\ker\phi_A+\dim\operatorname{Im}\phi_A=n.\]
\section{第1章习题}
\setcounter{exsection}{5}
\begin{exercise}% 5.1
    设 $\Omega=\{+,-,++,+-,-+,--,+++,\cdots\}$ 是加号和减号的有限序列的集合, 而 $f:\Omega\to\Omega$ 是一个变换, 将元素 $\omega'=\omega_1\omega_2\cdots\omega_n\in\Omega$ 对应到 $\omega=\omega_1\dot{\omega_1}\omega_2\dot{\omega_2}\cdots\omega_n\dot{\omega_n}$, 其中
    \[\dot{\omega_k}=\begin{cases}
        -, & \omega_k=+, \\
        +, & \omega_k=-.
    \end{cases}\]
    
    证明在 $f(f(\omega))$ 的任意长度 $>4$ 的任意区间 $\omega_0$ 内包含 $++$ 或 $--$.    
\end{exercise}
\begin{proof}
    由定义得对 $\omega=\omega_1\omega_2\cdots\omega_n\in\Omega$,
    \[f(\omega)=\omega_1\dot{\omega_1}\omega_2\dot{\omega_2}\cdots\omega_n\dot{\omega_n},\]
    \[f(f(\omega))=\omega_1\dot{\omega_1}\dot{\omega_1}\omega_1\omega_2\dot{\omega_2}\dot{\omega_2}\omega_2\cdots\omega_n\dot{\omega_n}\dot{\omega_n}\omega_n.\]

    $f(f(\omega))$ 可以写成下列形式:
    \[f(f(\omega))=a_1b_1b_1a_2a_2'b_2b_2\cdots a_{n-1}a_{n-1}'b_{n-1}b_{n-1}a_n,\]

    其中 $a_i,a_i',b_j\ (i=1,2,\cdots,n,j=1,2,\cdots,n-1)$ 为 $+$ 或 $-$.

    考察 $\omega_0$ 的前 $4$ 个元素, 记这 $4$ 个元素组成的序列为 $\omega_1$. 可以证明: $\omega_1$ 中一定有 $2$ 个 $a$(包括 $a'$, 下同) 和 $2$ 个 $b$. 事实上, 假设 $\omega_1$ 中有 $3$ 个 $a,\because \omega$ 中没有 $3$ 个连续的 $a,\therefore \omega_1$ 只有可能有如下两种情况:
    \[aaba,\quad abaa,\]

    这与 $\omega$ 中没有 $aba$ 矛盾.

    $\therefore \omega_1$ 只有可能有如下 $3$ 种情况:
    \[abba,\quad aabb,\quad baab.\]

    对于前两种情况, 由于两个 $b$ 是相同的, $\therefore \omega_1$ 包含 $++$ 或 $--$, $\therefore \omega_0$ 包含 $++$ 或 $--$.

    对于第三种情况, $\omega_0$ 为 $baabb\cdots$, 也包含 $++$ 或 $--$.
\end{proof}
\begin{exercise}% 5.2
    由法则 $n\to n^2$ 给出的映射 $f:\mathbb{N}\to\mathbb{N}$ 有右逆吗? 给出 $f$ 的两个左逆.
\end{exercise}
\begin{solution}
    $\because f:\mathbb{N}\to\mathbb{N},\therefore$ 若 $f$ 有右逆 $g$, 则 $g:\mathbb{N}\to\mathbb{N}$.

    考察元素 $2\in\mathbb{N}$. 有 $f\circ g(2)=e_{\mathbb{N}}(2)=2$.

    另一方面, $\because 2$ 不是任一自然数的平方, $\therefore2\notin f(\mathbb{N}),\therefore2\notin f\circ g(\mathbb{N})$, 与 $f\circ g(2)=2$ 矛盾. $\therefore f$ 没有右逆.

    $f$ 的两个左逆 $h_1,h_2$ 为:
    \[h_1(n)=\begin{cases}
        \sqrt{n}, & n\in\{x|x=k^2,k\in\mathbb{N}\}, \\
        1, & \text{otherwise}, \\
    \end{cases}\]
    \[h_2(n)=\begin{cases}
        \sqrt{n}, & n\in\{x|x=k^2,k\in\mathbb{N}\}, \\
        2, & \text{otherwise}. \\
    \end{cases}\]
\end{solution}
\begin{exercise}% 5.3
    设 $f:X\to Y$ 是一个映射, 且 $S,T\subset X$, 证明:
    \[(1)f(S\cup T)=f(S)\cup f(T),\quad(2)f(S\cap T)\subset f(S)\cap f(T).\]

    举例说明 (2) 中的包含关系不能换成相等关系.
\end{exercise}
\begin{proof}
    (1)
    \begin{align*}
        y\in f(S\cup T) & \Leftrightarrow\exists x(x\in S\cup T\land f(x)=y) \\
        & \Leftrightarrow\exists x((x\in S\vee x\in T)\land f(x)=y) \\
        & \Leftrightarrow\exists x((x\in S\land f(x)=y)\vee(x\in T\land f(x)=y)) \\
        & \Leftrightarrow\exists x(x\in S\land f(x)=y)\vee\exists x(x\in T\land f(x)=y) \\
        & \Leftrightarrow y\in f(S)\vee y\in f(T) \\
        & \Leftrightarrow y\in f(S)\cup f(T).
    \end{align*}

    (2)
    \begin{align*}
        y\in f(S\cap T) & \Leftrightarrow\exists x(x\in S\cap T\land f(x)=y) \\
        & \Leftrightarrow\exists x((x\in S\land x\in T)\land f(x)=y) \\
        & \Leftrightarrow\exists x((x\in S\land f(x)=y)\land(x\in T\land f(x)=y)) \\
        & \Rightarrow\exists x(x\in S\land f(x)=y)\land\exists x(x\in T\land f(x)=y) \\
        & \Leftrightarrow y\in f(S)\land y\in f(T) \\
        & \Leftrightarrow y\in f(S)\cap f(T).
    \end{align*}

    考虑映射
    \[f:\begin{array}{rcl}
        \mathbb{N} & \to & \{1\} \\
        x & \to & 1 \\
    \end{array}\]

    和 $S=\{1,2\},T=\{3,4\}$. 有 $f(S\cap T)=\varnothing,f(S)\cap f(T)=\{1\}$.
\end{proof}
\begin{exercise}% 5.4
    $S$ 的全体子集的集合记作
    \[\mathcal{P}(S)=\{T|T\subset S\}.\]

    设 $|S|=n$, 求 $|\mathcal{P}(S)|$.
\end{exercise}
\begin{solution}
    将 $S$ 中的元素排成一列:
    \[S=\{a_1,a_2,\cdots,a_n\}.\]

    设映射 $f:\mathcal{P}(S)\to B$, 其中 $B$ 是全体 $n$ 位二进制序列组成的集合.

    $\forall A\in\mathcal{P}(S)$, 若 $a_k\in A\ (k=1,2,\cdots,n)$, 则 $f$ 的第 $k$ 位为 $1$, 否则为 $0$.

    容易验证 $f$ 是单射, $\because\mathcal{P}(S)$ 包含 $S$ 的全部子集, $\therefore f$ 是满射. $\therefore|\mathcal{P}(S)|=|B|=2^n$.
\end{solution}
\begin{exercise}% 5.5
    设 $f:X\to Y$ 是一个映射, 且对某个元素 $a\in X,b=f(a)$. 原像
    \[f^{-1}(b)=f^{-1}(f(a))=\{x|f(x)=f(a)\}\]

    称为元素 $b\in\operatorname{Im}f$ 上的\textbf{纤维}. 证明集合 $X$ 是互不相交的纤维的并.
\end{exercise}
\begin{proof}
    设 $b_1,b_2\in\operatorname{Im}f,a\in X$.

    若 $a\in f^{-1}(b_1)\land a\in f^{-1}(b_2)$, 则 $f(a)=b_1\land f(a)=b_2\Rightarrow b_1=b_2$.

    $\therefore$ 对 $b_1,b_2\in\operatorname{Im}f$, 若 $b_1\neq b_2$, 则 $f^{-1}(b_1),f^{-1}(b_2)$ 互不相交.

    $\forall a\in X,a\in f^{-1}(f(a)),\therefore X$ 是纤维的并.
\end{proof}
\stepcounter{exsection}
\begin{exercise}% 6.1
    设 $\mathbb{R}^2/\sim$ 是图 8 (课本上的)给出的商集, $l$ 是与 $Ox$ 轴相交的任一直线, 试给出 $\mathbb{R}^2/\sim$ 的元素和 $l$ 的点之间的一一对应.
\end{exercise}
\begin{solution}
    $\mathbb{R}^2/\sim$ 的元素为直线 $l',l'$ 与 $l$ 的交点与 $l'$ 一一对应.
\end{solution}
\begin{exercise}% 6.2
    对 $\mathbb{R}^2$ 中的元素定义关系 $\sim$: $(a,b)\sim(c,d)$ 当且仅当 $a-c\in\mathbb{Z},b-d\in\mathbb{Z}$. 证明关系 $\sim$ 是 $\mathbb{R}^2$ 上的一个等价关系.
\end{exercise}
\begin{proof}
    $\because a-a=b-b=0$, $\therefore(a,b)\sim(a,b)$.

    $\because a-c\in\mathbb{Z}\Rightarrow c-a\in\mathbb{Z},b-d\in\mathbb{Z}\Rightarrow d-b\in\mathbb{Z}$, $\therefore(a,b)\sim(c,d)\Rightarrow(c,d)\sim(a,b)$.

    $\because a-a'\in\mathbb{Z},a'-a''\in\mathbb{Z}\Rightarrow a-a''\in\mathbb{Z}$, $\therefore(a,b)\sim(a',b'),(a',b')\sim(a'',b'')\Rightarrow(a,b)\sim(a'',b'')$.

    $\therefore\ \sim$ 是 $\mathbb{R}^2$ 上的一个等价关系.
\end{proof}
\begin{note}
    $\because\forall\bar{x}\in\mathbb{R}^2/\sim,\exists x'\in[0,1]^2$ 使得 $x'\in\bar{x}$, $\therefore\mathbb{R}^2/\sim$ 可以与 $[0,1]^2$ 等同起来. 书上的图 11 给出了将 $[0,1]^2$ 变形为环面的方法: 将 $[0,1]^2$ 卷成一个柱面, 使得 $(0,0)$ 和 $(0,1)$ 重合, $(1,0)$ 和 $(1,1)$ 重合, 再将这个柱面变形为环面, 使得柱面的两个圆形边缘重合.
\end{note}
\begin{exercise}% 6.3
    证明 2 元, 3 元和 4 元集分别有 2,5 和 15 个不同的商集.
\end{exercise}
\begin{proof}
    设集合 $S$ 的元素为正整数 $\{1,2,\cdots\}$.

    2 元集合 $S$ 的商集只有 $\{\{1,2\}\},\{\{1\},\{2\}\}$ 两个.

    3 元集合 $S$ 的全体子集的集合 $\mathcal{P}(S)$ 中可以组成商集的组合有:
    \begin{enumerate}
        \item 1 个 3 元集合 $\{1,2,3\}$ 组成一个商集;
        \item 3 个 2 元集合与与其对应的 1 元集合, 如 $\{1,2\},\{3\}$, 组成 3 个商集;
        \item 1 元集合 $\{1\},\{2\},\{3\}$ 组成一个商集.
    \end{enumerate}

    一共有 $1+3+1=5$ 个商集.

    4 元集合 $S$ 的全体子集的集合 $\mathcal{P}(S)$ 中可以组成商集的组合有:
    \begin{enumerate}
        \item 1 个 4 元集合 $\{1,2,3,4\}$ 组成一个商集;
        \item $\dbinom{4}{1}$ 个 3 元集合与与其对应的 1 元集合, 如 $\{1,2,3\},\{4\}$, 组成 $\dbinom{4}{1}$ 个商集;
        \item $\dbinom{4}{2}$ 个 2 元集合中的一半与与其对应的 2 元集合, 如 $\{1,2\},\{3,4\}$, 组成 $\dbinom{4}{2}/2$ 个商集;
        \item $\dbinom{4}{2}$ 个 2 元集合与与其对应的 1 元集合, 如 $\{1,2\},\{3\},\{4\}$, 组成 $\dbinom{4}{2}$ 个商集;
        \item 1 元集合 $\{1\},\{2\},\{3\},\{4\}$ 组成一个商集.
    \end{enumerate}

    一共有 $1+\dbinom{4}{1}+\dbinom{4}{2}/2+\dbinom{4}{2}+1=15$ 个商集.
\end{proof}
\begin{exercise}% 6.4
    设 $\sim$ 是集合 $X$ 上的一个等价关系,且 $f:X\to Y$ 是一个映射, 使得
    \[x\sim x'\Rightarrow f(x)=f(x').\]

    证明 $f$ 与 $\sim$ 的这一相容性条件允许我们定义一个从 $X/\sim$ 到 $Y$ 的诱导映射 $\bar{f}:\bar{x}\mapsto f(x)$, 它给出了分解式 $f=\bar{f}\circ p$, 但 $\bar{f}$ 不一定是单射. 讨论 $\bar{f}$ 成为单射的条件.
\end{exercise}
\begin{proof}
    $\because\bar{x}'=\bar{x}\Rightarrow f(x')=f(x),\therefore\bar{f}$ 是有确切定义的.

    $\because\bar{f}$ 是单射 $\Leftrightarrow\forall\bar{x},\bar{x}'\in\pi_\sim(X)$, 若 $\bar{x}\neq\bar{x}'$, 则 $\bar{f}(\bar{x})\neq\bar{f}(\bar{x}')\Leftrightarrow$ 若 $\bar{f}(\bar{x})=\bar{f}(\bar{x}')$, 则 $\bar{x}=\bar{x}'$, $\therefore$
    \[x\sim x'\Leftrightarrow f(x)=f(x')\]

    是 $\bar{f}$ 成为单射的条件.
\end{proof}
\addtocounter{exsection}{1}
\stepcounter{exsection}
\stepcounter{exercise}
\begin{exercise}% 8.2
    求置换 $(1\ 2\ 3\ 4\ 5)(6\ 7)(8)$ 和
    \[\pi=\begin{pmatrix}
        1 & 2 & 3 & 4 & 5 & 6 & 7 & 8 \\
        3 & 6 & 8 & 2 & 1 & 4 & 5 & 7 \\
    \end{pmatrix}\]

    的阶.
\end{exercise}
\begin{solution}
    对于 $m$ 元循环 $\sigma$, $\forall k\in\mathbb{N_+},\forall i\in\Omega,\sigma^{mk}(i)=i$.

    $\therefore$ 将置换 $\sigma$ 分解为不相交的循环的乘积 $\sigma_1,\sigma_2,\cdots,\sigma_s$, 由习题 4.2 的第 2.13 题得 $\sigma$ 的阶是 $\sigma_1,\sigma_2,\cdots,\sigma_s$ 的阶的最小公倍数.

    $\therefore$ 置换 $(1\ 2\ 3\ 4\ 5)(6\ 7)(8)$ 的阶为 $10$.

    将 $\pi$ 分解为不相交的循环的乘积:
    \[\pi=(1\ 3\ 8\ 7\ 5)(2\ 6\ 4).\]

    $\therefore$ 置换 $\pi$ 的阶为 $15$.
\end{solution}
\begin{exercise}% 8.3
    形如
    \[\pi=\pi_1\pi_2\cdots\pi_m,\quad l_k>1,\quad 1\leq k\leq m\]
    
    的置换 $\pi$ 含有 $m$ 个不相交的循环, 令
    \[m'=n-\sum\limits_{i=1}^{m}l_k,\]

    则 $\pi$ 使 $m'$ 个点保持不变. $d(\pi)=n-(m+m')$ 称为 $\pi$ 的\textbf{减量}. 验证 $\varepsilon_\pi=(-1)^{d(\pi)}$.
\end{exercise}
\begin{solution}
    用书上定理 1 推论的方法可以将 $\pi_k$ 分解成 $l_k-1$ 个对换的乘积. 由书上的定理 2 得 $\varepsilon_{\pi_k}=(-1)^{l_k-1}$,
    \begin{align*}
        \varepsilon_{\pi} & =\varepsilon_{\pi_1\pi_2\cdots\pi_m} \\
        & =\varepsilon_{\pi_1}\varepsilon_{\pi_2}\cdots\varepsilon_{\pi_m} \\
        & =(-1)^{l_1-1}(-1)^{l_2-1}\cdots(-1)^{l_m-1} \\
        & =(-1)^{n-m'-m}=(-1)^{d(\pi)}.
    \end{align*}
\end{solution}
\stepcounter{exsection}
\begin{exercise}% 9.1
    证明: 形如 $4k-1$ 的素数有无穷多个.
\end{exercise}
\begin{proof}
    $\forall k_1,k_2\in\mathbb{Z}$, 有
    \[(4k_1-1)(4k_2-1)=4(4k_1k_2-k_1-k_2)+1,\]
    \[(4k_1+1)(4k_2-1)=4(4k_1k_2-k_1+k_2)-1,\]
    \[(4k_1+1)(4k_2+1)=4(4k_1k_2+k_1+k_2)+1.\]

    从上面最后一个式子可以得出: $\{4k+1|k\in\mathbb{Z}\}$ 对乘法封闭(即 $\{4k+1|k\in\mathbb{Z}\}$ 中元素的乘积仍然在 $\{4k+1|k\in\mathbb{Z}\}$ 中).

    假设 $\{4k-1|k\in\mathbb{Z}\}$ 中只有有限个素数 $p_1,p_2,\cdots,p_n$.
    
    (a) 如果 $n$ 是偶数, 那么 $p_1p_2\cdots p_n\in\{4k+1|k\in\mathbb{Z}\},q=p_1p_2\cdots p_n+2\in\{4k-1|k\in\mathbb{Z}\}$.
    
    $\because p_1p_2\cdots p_n$ 不是偶数, $\therefore q$ 不是偶数.
    
    $\because2$ 不能被 $p_1,p_2,\cdots,p_n$ 整除, $\therefore q$ 不能被 $p_1,p_2,\cdots,p_n$ 整除.

    $\because$ 每个 $\neq2$ 的素数都能被写成 $4k-1$ 或 $4k+1$ 的形式, $\therefore$ 如果 $q$ 不是素数, 那么 $q$ 是 $\{4k+1|k\in\mathbb{Z}\}$ 中的素数的乘积, 这与 $\{4k+1|k\in\mathbb{Z}\}$ 对乘法封闭矛盾. $\therefore q$ 是素数, 这与 $\{4k-1|k\in\mathbb{Z}\}$ 中只有有限个素数 $p_1,p_2,\cdots,p_n$ 矛盾.

    (b) 如果 $n$ 是奇数, 那么 $p_1p_2\cdots p_n\in\{4k-1|k\in\mathbb{Z}\},p_1p_2\cdots p_n+4\in\{4k-1|k\in\mathbb{Z}\}$. 与 (a) 类似, 可以证明 $p_1p_2\cdots p_n+4$ 是素数, 这与 $\{4k-1|k\in\mathbb{Z}\}$ 中只有有限个素数 $p_1,p_2,\cdots,p_n$ 矛盾.
\end{proof}
\begin{note}
    用相同的方法可以证明: 形如 $3k-1,6k-1$ 的素数有无穷多个.
\end{note}
\addtocounter{exercise}{2}
\begin{exercise}% 9.4
    证明: 若 $p$ 是素数, 则 $\forall n\in\mathbb{Z},p|(n^p-n)$.
\end{exercise}
\begin{proof}
    用数学归纳法. 当 $n=1$ 时由 $p|0$ 得. 假设 $(n-1)^p-(n-1)=rp$, 则
    \begin{align*}
        n^p-n & =((n-1)+1)^p-1-(n-1) \\
        & =\sum\limits_{k=1}^{p-1}\dbinom{p}{k}(n-1)^k+(n-1)^p+1-1-(n-1) \\
        & =\sum\limits_{k=1}^{p-1}\dbinom{p}{k}(n-1)^k+rp.
    \end{align*}

    $\because p$ 整除 $\dbinom{p}{k}\ (k=1,2,\cdots,p-1)$, $\therefore p$ 整除 $n^p-n$.
\end{proof}
\begin{exercisec}[1.10]\label{exc1.10}
    如果方程组
    \begin{equation}\label{eq3.1}
        \begin{cases}
            a_{11}x_1+a_{12}x_2+\cdots+a_{1n}x_n=b_1, \\
            a_{21}x_1+a_{12}x_2+\cdots+a_{2n}x_n=b_2, \\
            \cdots \\
            a_{m1}x_1+a_{m2}x_2+\cdots+a_{mn}x_n=b_m \\
        \end{cases}
    \end{equation}
    是确定的, 证明与它相伴的齐次线性方程组
    \begin{equation}\label{eq3.2}
        \begin{cases}
            a_{11}x_1+a_{12}x_2+\cdots+a_{1n}x_n=0, \\
            a_{21}x_1+a_{12}x_2+\cdots+a_{2n}x_n=0, \\
            \cdots \\
            a_{m1}x_1+a_{m2}x_2+\cdots+a_{mn}x_n=0 \\
        \end{cases}
    \end{equation}
    只有零解. 举例说明反之不对.
\end{exercisec}
\begin{proof}
    假设方程组 (\ref{eq3.1}) 有唯一解 $s_1,s_2,\cdots,s_n$, 且方程组 (\ref{eq3.2}) 有非零解 $t_1,t_2,\cdots,t_n$, 那么由书上的定理 1.9(3), $s_1+t_1,s_2+t_2,\cdots,s_n+t_n$ 是方程组 (\ref{eq3.1}) 的解.
    
    $\because t_1,t_2,\cdots,t_n$ 不全为 $0$, $\therefore s_1,s_2,\cdots,s_n$ 与 $s_1+t_1,s_2+t_2,\cdots,s_n+t_n$ 不全相等, $\therefore$ 方程组 (\ref{eq3.1}) 有两个不同的解, 这与方程组 (\ref{eq3.1}) 的确定性矛盾.

    当方程组不相容时, 与它相伴的齐次线性方程组也可能只有零解. 比如与方程组
    \[\begin{cases}
        x_1=0, \\
        x_2=0, \\
        x_1+x_2=1 \\
    \end{cases}\]
    相伴的齐次线性方程组只有零解.
\end{proof}
\begin{exercisec}[1.11]
    设 $m=n$. 证明方程组 (\ref{eq3.1}) 是确定的当且仅当方程组 (\ref{eq3.2}) 只有零解.
\end{exercisec}
\begin{proof}
    ($\Rightarrow$) 是补充题 \ref{exc1.10} 的一个特例.

    ($\Leftarrow$) 设对方程组 (\ref{eq3.1}) 进行初等变换可以得到方程组
    \[\begin{cases}
        g_{11}x_1+g_{12}x_2+\cdots+g_{1n}x_n=h_1, \\
        g_{2k}x_k+g_{2,k+1}x_{k+1}+\cdots+g_{2n}x_n=h_2, \\
        g_{3l}x_l+g_{3,l+1}x_{l+1}+\cdots+g_{3n}x_n=h_3, \\
        \cdots \\
        g_{rs}x_s+g_{r,s+1}x_{s+1}+\cdots+g_{rn}x_n=h_r, \\
        0=h_{r+1}, \\
        \cdots \\
        0=h_n, \\
    \end{cases}\]

    则对方程组 (\ref{eq3.1}) 进行初等变换可以得到方程组
    \begin{equation}\label{eq3.4}
        \begin{cases}
            g_{11}x_1+g_{12}x_2+\cdots+g_{1n}x_n=0, \\
            g_{2k}x_k+g_{2,k+1}x_{k+1}+\cdots+g_{2n}x_n=0, \\
            g_{3l}x_l+g_{3,l+1}x_{l+1}+\cdots+g_{3n}x_n=0, \\
            \cdots \\
            g_{rs}x_s+g_{r,s+1}x_{s+1}+\cdots+g_{rn}x_n=0. \\
        \end{cases}
    \end{equation}
    
    $\because$ 方程组 (\ref{eq3.2}) 只有零解, 由书上的定理 1.7 得将方程组 (\ref{eq3.4}) 满足 $r=n$. $\because n=m$, 由书上的命题 1.3(1) 得方程组 (\ref{eq3.1}) 有解.
    
    假设方程组 (\ref{eq3.1}) 有不相同的两个解 $s_1,s_2,\cdots,s_n$ 和 $t_1,t_2,\cdots,t_n$, 由书上的定理 1.9(2) 得方程组 (\ref{eq3.1}) 有非零解 $s_1-t_1,s_2-t_2,\cdots,s_n-t_n$, 与方程组 (\ref{eq3.2}) 只有零解矛盾.
\end{proof}
\begin{exercisec}[1.15(2)]
    在平面上引进直角坐标系, 求平面上不共线的四点 $(x_i,y_i)\ (i=1,2,3,4)$ 共圆的充要条件.
\end{exercisec}
\begin{solution}
    四点 $(x_i,y_i)\ (i=1,2,3,4)$ 共圆当且仅当关于 $a,b,c$ 的方程组
    \[(x_i-a)^2+(y_i-b)^2=c+a^2+b^2\quad(i=1,2,3,4)\]
    有唯一解. 方程组化简得
    \begin{equation}\label{eq3.4}
        c+2ax_i+2by_i=x_i^2+y_i^2\quad(i=1,2,3,4),
    \end{equation}

    增广矩阵为
    \[\begin{pmatrix}
        1 & 2x_1 & 2y_1 & x_1^2+y_1^2 \\
        1 & 2x_2 & 2y_2 & x_2^2+y_2^2 \\
        1 & 2x_3 & 2y_3 & x_3^2+y_3^2 \\
        1 & 2x_4 & 2y_4 & x_4^2+y_4^2 \\
    \end{pmatrix},\]

    $\because$ 四点不共线, $\therefore x_1,x_2,x_3,x_4$ 不全相等. 不妨设 $x_1\neq x_i\ (i=2,3,4)$. 第 $2,3,4$ 行减去第一行得
    \[\begin{pmatrix}
        1 & 2x_1 & 2y_1 & x_1^2+y_1^2 \\
        0 & 2(x_2-x_1) & 2(y_2-y_1) & (x_2^2+y_2^2)-(x_1^2+y_1^2) \\
        0 & 2(x_3-x_1) & 2(y_3-y_1) & (x_3^2+y_3^2)-(x_1^2+y_1^2) \\
        0 & 2(x_4-x_1) & 2(y_4-y_1) & (x_4^2+y_4^2)-(x_1^2+y_1^2) \\
    \end{pmatrix}\]

    第 $i\ (i=2,3,4)$ 行除以 $x_i-x_1$ 得
    \[\begin{pmatrix}
        1 & 2x_1 & 2y_1 & x_1^2+y_1^2 \\
        0 & 2 & 2\dfrac{y_2-y_1}{x_2-x_1} & x_2+x_1+\dfrac{y_2^2-y_1^2}{x_2-x_1} \\[8pt]
        0 & 2 & 2\dfrac{y_3-y_1}{x_3-x_1} & x_3+x_1+\dfrac{y_3^2-y_1^2}{x_3-x_1} \\[8pt]
        0 & 2 & 2\dfrac{y_4-y_1}{x_4-x_1} & x_4+x_1+\dfrac{y_4^2-y_1^2}{x_4-x_1} \\[8pt]
    \end{pmatrix}.\]

    第 $i\ (i=3,4)$ 行减去第 $2$ 行得
    \[\begin{pmatrix}
        1 & 2x_1 & 2y_1 & x_1^2+y_1^2 \\
        0 & 2 & 2\dfrac{y_2-y_1}{x_2-x_1} & x_2+x_1+\dfrac{y_2^2-y_1^2}{x_2-x_1} \\[8pt]
        0 & 0 & 2\left(\dfrac{y_3-y_1}{x_3-x_1}-\dfrac{y_2-y_1}{x_2-x_1}\right) & x_3-x_2+\left(\dfrac{y_3^2-y_1^2}{x_3-x_1}-\dfrac{y_2^2-y_1^2}{x_2-x_1}\right) \\[8pt]
        0 & 0 & 2\left(\dfrac{y_4-y_1}{x_4-x_1}-\dfrac{y_2-y_1}{x_2-x_1}\right) & x_4-x_2+\left(\dfrac{y_4^2-y_1^2}{x_4-x_1}-\dfrac{y_2^2-y_1^2}{x_2-x_1}\right) \\[8pt]
    \end{pmatrix}.\]

    $\because$ 四点不共线, $\therefore$ 过点 $x_1,x_2$ 的直线的斜率不等于过点 $x_1,x_3$ 的直线的斜率, 即 $\dfrac{y_3-y_1}{x_3-x_1}\neq\dfrac{y_2-y_1}{x_2-x_1}$, 同理得 $\dfrac{y_4-y_1}{x_4-x_1}\neq\dfrac{y_2-y_1}{x_2-x_1}$. 第 $i\ (i=3,4)$ 行除以 $\dfrac{y_i-y_1}{x_i-x_1}-\dfrac{y_2-y_1}{x_2-x_1}$ 得
    \[\begin{pmatrix}
        1 & 2x_1 & 2y_1 & x_1^2+y_1^2 \\
        0 & 2 & 2\dfrac{y_2-y_1}{x_2-x_1} & x_2+x_1+\dfrac{y_2^2-y_1^2}{x_2-x_1} \\[16pt]
        0 & 0 & 2 & \dfrac{x_3-x_2+\left(\dfrac{y_3^2-y_1^2}{x_3-x_1}-\dfrac{y_2^2-y_1^2}{x_2-x_1}\right)}{\dfrac{y_3-y_1}{x_3-x_1}-\dfrac{y_2-y_1}{x_2-x_1}} \\[16pt]
        0 & 0 & 2 & \dfrac{x_4-x_2+\left(\dfrac{y_4^2-y_1^2}{x_4-x_1}-\dfrac{y_2^2-y_1^2}{x_2-x_1}\right)}{\dfrac{y_4-y_1}{x_4-x_1}-\dfrac{y_2-y_1}{x_2-x_1}} \\[16pt]
    \end{pmatrix}.\]

    第 $4$ 行减去第 $3$ 行得
    \[\begin{pmatrix}
        1 & 2x_1 & 2y_1 & x_1^2+y_1^2 \\
        0 & 2 & 2\dfrac{y_2-y_1}{x_2-x_1} & x_2+x_1+\dfrac{y_2^2-y_1^2}{x_2-x_1} \\[8pt]
        0 & 0 & 2 & \dfrac{x_3-x_2+\left(\dfrac{y_3^2-y_1^2}{x_3-x_1}-\dfrac{y_2^2-y_1^2}{x_2-x_1}\right)}{\dfrac{y_3-y_1}{x_3-x_1}-\dfrac{y_2-y_1}{x_2-x_1}} \\[16pt]
        0 & 0 & 0 & d \\
    \end{pmatrix},\]

    其中
    \begin{align*}
        d & =\dfrac{x_4-x_2+\left(\dfrac{y_4^2-y_1^2}{x_4-x_1}-\dfrac{y_2^2-y_1^2}{x_2-x_1}\right)}{\dfrac{y_4-y_1}{x_4-x_1}-\dfrac{y_2-y_1}{x_2-x_1}}-\dfrac{x_3-x_2+\left(\dfrac{y_3^2-y_1^2}{x_3-x_1}-\dfrac{y_2^2-y_1^2}{x_2-x_1}\right)}{\dfrac{y_3-y_1}{x_3-x_1}-\dfrac{y_2-y_1}{x_2-x_1}} \\
        & =\dfrac{(x_4-x_2)(x_4-x_1)(x_2-x_1)+(x_2-x_1)(y_4^2-y_1^2)-(x_4-x_1)(y_2^2-y_1^2)}{(x_2-x_1)(y_4-y_1)-(x_4-x_1)(y_2-y_1)} \\
        & \quad-\dfrac{(x_3-x_2)(x_2-x_1)(x_3-x_1)+(x_2-x_1)(y_3^2-y_1^2)-(x_3-x_1)(y_2^2-y_1^2)}{(y_3-y_1)(x_2-x_1)-(x_3-x_1)(y_2-y_1)}.
    \end{align*}

    方程组 (\ref{eq3.4}) 有唯一解当且仅当 $d=0$.
\end{solution}
\begin{exercisec}[1.18(3)]
    用书上的命题 1.16, 从方程组的角度解释等式
    \[\begin{vmatrix}
        a+a' & b \\
        c+c' & d \\
    \end{vmatrix}=\begin{vmatrix}
        a & b \\
        c & d \\
    \end{vmatrix}+\begin{vmatrix}
        a' & b \\
        c' & d \\
    \end{vmatrix},\quad\begin{vmatrix}
        a & b+b' \\
        c & d+d' \\
    \end{vmatrix}=\begin{vmatrix}
        a & b \\
        c & d \\
    \end{vmatrix}+\begin{vmatrix}
        a & b' \\
        c & d' \\
    \end{vmatrix}.\]
\end{exercisec}
\begin{solution}
    考察方程组
    \[\begin{cases}
        ex_1+bx_2=a, \\
        fx_1+dx_2=c, \\
    \end{cases}\quad\begin{cases}
        ex_1+bx_2=a', \\
        fx_1+dx_2=c', \\
    \end{cases}\quad\begin{cases}
        ex_1+bx_2=a+a', \\
        fx_1+dx_2=c+c'. \\
    \end{cases}\]

    如果 $s_1,s_2$ 是第一个方程的解, $t_1,t_2$ 是第二个方程的解, 那么 $s_1+t_1,s_2+t_2$ 是第三个方程的解. 如果这三个方程都有唯一解, 则用书上的命题 1.16 可以得到题目中的第一个等式. 第二个等式可以类似地得到.
\end{solution}
\begin{exercisec}[1.5.1]
    求实系数多项式 $f(x)$ 使得 $f(1)=8,f(-1)=2,f(2)=14$.
\end{exercisec}
\begin{solution}
    设 $f(x)=ax^2+bx+c$, 则有
    \[\begin{cases}
        a+b+c=8, \\
        a-b+c=2, \\
        4a+2b+c=14.
    \end{cases}\]

    解得 $a=1,b=3,c=4,f(x)=x^2+3x+4$.
\end{solution}
\begin{exercisec}[2.3.2, 2.3.5]
    把下面的置换 $\sigma\in S_n$ 写成不相交的循环的乘积并确定这些置换的奇偶性:
    \begin{align*}
        (2)\quad & \begin{pmatrix}
            1 & 2 & \cdots & k & k+1 & k+2 & \cdots & 2k \\
            2 & 4 & \cdots & 2k & 1 & 3 & \cdots & 2k-1 \\
        \end{pmatrix} \\
        & \text{或} \begin{pmatrix}
            1 & 2 & \cdots & k & k+1 & k+2 & \cdots & 2k+1 \\
            2 & 4 & \cdots & 2k & 1 & 3 & \cdots & 2k+1 \\
        \end{pmatrix} \\
        (3)\quad & \begin{pmatrix}
            1 & 2 & \cdots & k & k+1 & k+2 & \cdots & 2k \\
            1 & 3 & \cdots & 2k-1 & 2 & 4 & \cdots & 2k \\
        \end{pmatrix} \\
        & \text{或} \begin{pmatrix}
            1 & 2 & \cdots & k & k+1 & k+2 & \cdots & 2k-1 \\
            1 & 3 & \cdots & 2k-1 & 2 & 4 & \cdots & 2k \\
        \end{pmatrix} \\
        (4)\quad & \begin{pmatrix}
            1 & 2 & 3 & 4 & \cdots & n \\
            n & 1 & n-1 & 2 & \cdots & \cdot \\
        \end{pmatrix}
    \end{align*}
\end{exercisec}
\begin{solution}
    用补充题 \ref{ex2.3.4} 介绍的反序数来求置换的奇偶性, 可以不对置换进行分解就求出置换的符号.

    (2) 把置换写成映射的形式:
    \[\sigma(m)=\begin{cases}
        2m, & m\leq k, \\
        2(m-k)-1, & m>k. \\
    \end{cases}\]

    当 $1\leq i<j\leq k$ 时有 $\sigma(i)=2i<2j=\sigma(j)$. 当 $k+1\leq i<j\leq n$ 时有 $\sigma(i)=2(i-k)-1<2(j-k)-1=\sigma(j)$.

    当 $1\leq i\leq k<j\leq n$ 时有
    \[\sigma(i)>\sigma(j)\Leftrightarrow2i>2(j-k)-1\Leftrightarrow i+k+1/2>j.\]

    $\therefore\sigma$ 的反序数为(注意 $\forall i,\{j|k<j<i+k+1/2\}\subset\{j|k<j\leq n\}$)
    \[\sum\limits_{i=1}^k|\{j|k<j<i+k+1/2\}|=\sum\limits_{i=1}^k|\{j|k<j\leq i+k\}|=\sum\limits_{i=1}^ki=\dfrac{k(k+1)}{2}.\]

    $\therefore$ 置换的符号为 $(-1)^{k(k+1)/2}$.

    (3)
    \[\sigma(m)=\begin{cases}
        2m-1, & m\leq k, \\
        2(m-k), & m>k. \\
    \end{cases}\]

    与 (2) 类似, 只需考虑 $1\leq i\leq k<j\leq n$ 时的情形. 有
    \[\sigma(i)>\sigma(j)\Leftrightarrow2i-1>2(j-k)\Leftrightarrow i+k-1/2>j.\]

    $\therefore\sigma$ 的反序数为
    \[\sum\limits_{i=1}^k|\{j|k<j<i+k-1/2\}|=\sum\limits_{i=1}^k|\{j|k<j\leq i+k-1\}|=\sum\limits_{i=1}^k(i-1)=\dfrac{k(k-1)}{2}.\]

    $\therefore$ 置换的符号为 $(-1)^{k(k-1)/2}$.

    (4)
    \[\sigma(m)=\begin{cases}
        m/2, & 2|m, \\
        n-(m-1)/2, & 2\nmid m. \\
    \end{cases}\]
    
    当 $2|i,2|j$ 时有 $i<j\Rightarrow\sigma(i)<\sigma(j)$. 当 $2\nmid i,2\nmid j$ 时有 $i<j\Rightarrow\sigma(i)>\sigma(j)$.

    $\because\forall i,j,i+j<2n+1\Rightarrow j/2<n-(i-1)/2$, $\therefore$ 当 $2\nmid i,2|j$ 时 $\sigma(j)=j/2<n-(i-1)/2=\sigma(i)$. $\therefore\sigma$ 的反序数为
    \[|\{(i,j):2\nmid i,2\nmid j,i<j\}|+|\{(i,j):2\nmid i,2|j,i<j\}|=\dfrac{1}{2}\left\lfloor\dfrac{n+1}{2}\right\rfloor\left\lfloor\dfrac{n+1}{2}-1\right\rfloor+\dfrac{1}{2}\left\lfloor\dfrac{n}{2}\right\rfloor\left\lfloor\dfrac{n}{2}+1\right\rfloor.\]

    置换的分解现在还没想出来. 下面的 Mathematica 代码可以生成输出这 3 题的循环的交互式程序. 循环以图的形式给出, 可以方便地看出有多少个不相交的循环.
\begin{verbatim}
rules[cycle_] := Thread[DirectedEdge[cycle, RotateLeft[cycle]]];
permGraph[list_] := 
 Graph[Flatten[rules /@ First[PermutationCycles[list]]], 
  VertexLabels -> Placed["Name", Center], VertexSize -> 0.6]
Manipulate[
 permGraph[Join[Range[2, n, 2], Range[1, n, 2]]], {n, 2, 100, 1}](*第 2 题*)
Manipulate[
 permGraph[Join[Range[1, n, 2], Range[2, n, 2]]], {n, 2, 100, 1}](*第 3 题*)
Manipulate[
 permGraph[
  Riffle[Reverse[Range[Floor[n/2] + 1, n]], Range[Floor[n/2]]]], {n, 
  2, 100, 1}](*第 4 题*)
\end{verbatim}
\end{solution}

证明下面一题需要先证明一个引理.
\begin{lemma}\label{l3.1}
    设 $\sigma_1=(i_1\ i_2\ \cdots\ i_k),\sigma_2=(j_1\ j_2\ \cdots\ j_l)$ 是不相交的循环, 则 $\tau^{-1}\sigma_1\tau,\tau^{-1}\sigma_2\tau$ 也不相交.
\end{lemma}
\begin{proof}
    (1) 先考虑 $\tau$ 是对换的情形. 设 $\tau=(s_1\ s_2)$.
    
    (a) 如果 $\tau$ 与 $\sigma_1,\sigma_2$ 中的至少一个循环 (不妨设为 $\sigma_1$) 不相交, 那么 $\tau\sigma_1\tau^{-1}=\sigma_1\tau\tau^{-1}=\sigma_1$, 被 $\tau^{-1}\sigma_2\tau$ 改变的元素都属于 $\{s_1,s_2,j_1,j_2,\cdots,j_l\}$.
    
    $\because\{i_1,i_2,\cdots,i_k\}\cap\{s_1,s_2\}=\varnothing,\{i_1,i_2,\cdots,i_k\}\cap\{j_1,j_2,\cdots,j_l\}=\varnothing$, $\therefore$
    \begin{align*}
        & \{i_1,i_2,\cdots,i_k\}\cap\{s_1,s_2,j_1,j_2,\cdots,j_l\} \\
        & =(\{i_1,i_2,\cdots,i_k\}\cap\{s_1,s_2\})\cup(\{i_1,i_2,\cdots,i_k\}\cap\{j_1,j_2,\cdots,j_l\}) \\
        & =\varnothing\cup\varnothing=\varnothing.
    \end{align*}

    $\therefore\tau^{-1}\sigma_2\tau$ 与 $\tau\sigma_1\tau^{-1}=\sigma_1$ 不相交.

    (b) 如果 $\tau$ 与 $\sigma_1,\sigma_2$ 都相交, 那么 $\tau$ 中有一个元素 $\in\{i_1,i_2,\cdots,i_k\}$, 另一个元素 $\in\{j_1,j_2,\cdots,j_l\}$. 不妨设 $s_1=i_1,s_2=j_1$. 有
    \begin{align*}
        \tau\sigma_1\tau^{-1} & =(i_1\ j_1)(i_1\ i_2)(i_2\ i_3)\cdots(i_{k-1}\ i_k)(i_1\ j_1) \\
        & =(i_1\ j_1)(i_1\ i_2)(i_1\ j_1)(i_2\ i_3)\cdots(i_{k-1}\ i_k) \\
        & =(j_1\ i_2)(i_2\ i_3)\cdots(i_{k-1}\ i_k) \\
        & =(j_1\ i_2\ i_3\ \cdots\ i_k),
    \end{align*}

    类似地, $\tau\sigma_2\tau^{-1}=(i_1\ j_2\ j_3\ \cdots\ j_k)$.
    
    $\because\{j_1,i_2,i_3,\cdots,i_k\}\cap\{i_1,j_2,j_3,\cdots,j_l\}=\varnothing$, $\therefore\tau^{-1}\sigma_2\tau$ 与 $\tau\sigma_1\tau^{-1}=\sigma_1$ 不相交.

    (2) 对于任意的循环 $\tau$, 由书上的定理 1 得 $\tau$ 是对换 $\tau_1,\tau_2,\cdots,\tau_t$ 的乘积:
    \[\tau=\tau_1\tau_2\cdots\tau_t.\]

    由 (1) 得 $\tau_t\sigma_1\tau_t^{-1}$ 与 $\tau_t\sigma_2\tau_t^{-1}$ 不相交, $\tau_{t-1}\tau_t\sigma_1\tau_t^{-1}\tau_{t-1}^{-1}$ 与 $\tau_{t-1}\tau_t\sigma_2\tau_t^{-1}\tau_{t-1}^{-1}$ 不相交, $\cdots$,
    \[\tau\sigma_1\tau^{-1}=\tau_1\tau_2\cdots\tau_t\sigma_1\tau_t^{-1}\tau_{t-1}^{-1}\cdots\tau_1^{-1}\]
    与
    \[\tau\sigma_1\tau^{-1}=\tau_1\tau_2\cdots\tau_t\sigma_1\tau_t^{-1}\tau_{t-1}^{-1}\cdots\tau_1^{-1}\]
    不相交.
\end{proof}
\begin{exercisec}[2.3.3]
    设 $\sigma$ 是长度为 $k$ 的循环, 证明: 对于任意的 $\tau\in S_n$, 置换 $\tau\sigma\tau^{-1}$ 仍是长度为 $k$ 的循环.
\end{exercisec}
\begin{proof}
    假设 $\tau\sigma\tau^{-1}$ 是 $s\ (s>1)$ 个不相交的循环的乘积:
    \[\tau\sigma\tau^{-1}=\pi_1\pi_2\cdots\pi_s,\]

    则
    \begin{align*}
        \sigma & =\tau^{-1}\pi_1\pi_2\cdots\pi_s\tau \\
        & =(\tau^{-1}\pi_1\tau)(\tau^{-1}\pi_2\tau)\cdots(\tau^{-1}\pi_s\tau),
    \end{align*}

    由引理 \ref{l3.1} 得 $\tau^{-1}\pi_1\tau,\tau^{-1}\pi_2\tau,\cdots,\tau^{-1}\pi_s\tau$ 不相交, 这与 $\sigma$ 不能分解为 $s$ 个不相交的循环的乘积矛盾. $\therefore\tau\sigma\tau^{-1}$ 是循环.

    $\because\sigma$ 的长度为 $k$, $\therefore\sigma^k=e$, 且 $\forall m<k$, $\sigma^m\neq e$. $\therefore$
    \[(\tau\sigma\tau^{-1})^k=\tau\sigma^k\tau^{-1}=e,\]
    \[(\tau\sigma\tau^{-1})^m=\tau\sigma^m\tau^{-1}\neq\tau\tau^{-1}=e.\]

    $\therefore\tau\sigma\tau^{-1}$ 的长度为 $k$.
\end{proof}
\begin{exercisec}[2.3.4]\label{ex2.3.4}
    设 $\sigma\in S_n$. 称整数对 $\left<i,j\right>$ 是 $\sigma$ 的一个\textbf{反序}, 如果 $1\leq i<j\leq n$ 且 $\sigma(i)>\sigma(j)$, 整数对 $\left<i,j\right>$ 是 $\sigma$ 的一个\textbf{顺序}, 如果 $1\leq i<j\leq n$ 且 $\sigma(i)<\sigma(j)$. 显然没有反序的置换是单位置换 $e$. 设 $i<j$, $\left<i,j\right>$ 是 $\sigma$ 的反序, 令对换 $\tau=(\sigma(i)\ \sigma(j)),\tau'=(i\ j)$. 证明:
    \begin{enumerate}
        \def\labelenumi{(\arabic{enumi})}
        \item $\left<i,j\right>$ 是 $\tau\sigma$ 的顺序, 也是 $\sigma\tau'$ 的顺序.
        \item 如果 $\left<a,i\right>$ 和 $\left<a,j\right>$ 都是 $\sigma$ 的反序, 那么它们也都是 $\tau\sigma$ 的反序.
        \item 如果 $\left<a,i\right>$ 和 $\left<a,j\right>$ 中只有一个是 $\sigma$ 的反序, 那么它们中也只有一个是 $\tau\sigma$ 的反序.
        \item 如果 $\left<a,i\right>$ 和 $\left<a,j\right>$ 都是 $\sigma$ 的顺序, 那么它们也都是 $\tau\sigma$ 的顺序.
        \item 如果 $\left<i,b\right>$ 和 $\left<j,b\right>$ 都是 $\sigma$ 的顺序, 那么它们也都是 $\tau\sigma$ 的顺序.
        \item 如果 $\left<i,b\right>$ 和 $\left<j,b\right>$ 中只有一个是 $\sigma$ 的反序, 那么它们中也只有一个是 $\tau\sigma$ 的反序.
        \item 如果 $\left<i,b\right>$ 和 $\left<j,b\right>$ 都是 $\sigma$ 的反序, 那么它们也都是 $\tau\sigma$ 的反序.
        \item 如果 $\left<i,c\right>$ 和 $\left<c,j\right>$ 中只有一个是 $\sigma$ 的反序, 那么它们中也只有一个是 $\tau\sigma$ 的反序.
        \item $\left<i,c\right>$ 和 $\left<c,j\right>$ 不可能都是 $\sigma$ 的顺序.
        \item 如果 $\left<i,c\right>$ 和 $\left<c,j\right>$ 都是 $\sigma$ 的反序, 那么它们也都是 $\tau\sigma$ 的顺序.
    \end{enumerate}
\end{exercisec}
\begin{proof}
    (1) $\because\tau\sigma(i)=\tau((\sigma(i)))=\sigma(j),\tau\sigma(j)=\tau((\sigma(j)))=\sigma(i)$, $\therefore\tau\sigma(i)<\tau\sigma(j)$, 即 $\left<i,j\right>$ 是 $\tau\sigma$ 的顺序. 类似地, $\left<i,j\right>$ 是 $\sigma\tau'$ 的顺序.

    (2) $\because\left<a,i\right>$ 和 $\left<a,j\right>$ 都是 $\sigma$ 的反序, $\therefore a<i,a<j,\sigma(a)>\sigma(i),\sigma(a)>\sigma(j)$.

    $\because\tau\sigma(i)=\sigma(j),\tau\sigma(j)=\sigma(i),\tau\sigma(a)=\sigma(a)$, $\therefore\tau\sigma(a)>\tau\sigma(j)>\tau\sigma(i)$. $\therefore\left<a,i\right>$ 和 $\left<a,j\right>$ 都是 $\tau\sigma$ 的反序.

    (3) 不妨设 $\left<a,i\right>$ 是 $\sigma$ 的顺序, $\left<a,j\right>$ 是 $\sigma$ 的反序, 则 $a<i,a<j,\sigma(a)<\sigma(i),\sigma(a)>\sigma(j)$. $\therefore\tau\sigma(a)>\sigma(j)=\tau\sigma(i),\tau\sigma(a)<\sigma(i)=\tau\sigma(j)$. $\therefore\left<a,i\right>$ 是 $\tau\sigma$ 的反序, $\left<a,j\right>$ 是 $\tau\sigma$ 的顺序.

    (4) (5) (7) 与 (2) 类似. (6) 与 (3) 类似.

    (8) 已知 $i<c<j$. 如果 $\left<i,c\right>$ 是 $\sigma$ 的顺序, $\left<c,j\right>$ 是 $\sigma$ 的反序, 那么 $\sigma(c)>\sigma(i),\sigma(c)>\sigma(j)$. $\therefore\tau\sigma(c)>\tau\sigma(j),\tau\sigma(c)>\tau\sigma(i)$, $\therefore\left<i,c\right>$ 是 $\tau\sigma$ 的顺序, $\left<c,j\right>$ 是 $\tau\sigma$ 的反序. 同理, 如果 $\left<i,c\right>$ 是 $\sigma$ 的反序, $\left<c,j\right>$ 是 $\sigma$ 的顺序, 那么 $\left<i,c\right>$ 是 $\tau\sigma$ 的反序, $\left<c,j\right>$ 是 $\tau\sigma$ 的顺序.

    (9) 已知 $i<c<j$. 如果 $\left<i,c\right>$ 和 $\left<c,j\right>$ 都是 $\sigma$ 的顺序, 那么 $\sigma(i)<\sigma(c)<\sigma(j)$, 即 $\left<i,j\right>$ 是 $\sigma$ 的顺序, 与 $\left<i,j\right>$ 是 $\sigma$ 的反序矛盾.

    (10) 已知 $i<c<j$. 如果 $\left<i,c\right>$ 和 $\left<c,j\right>$ 都是 $\sigma$ 的反序, 那么 $\sigma(j)<\sigma(c)<\sigma(i)$. $\therefore\tau\sigma(i)<\tau\sigma(c)<\tau\sigma(j)$, 即 $\left<i,c\right>$ 和 $\left<c,j\right>$ 都是 $\tau\sigma$ 的顺序.
\end{proof}
\begin{note}
    由这题的结论得 $\tau\sigma$ 的反序的总数 (称为 $\tau\sigma$ 的\textbf{反序数}) 比 $\sigma$ 的反序数少一个奇数. 用数学归纳法容易证明存在对换 $\tau_1,\tau_2,\cdots,\tau_m$ 使得 $\tau_1\tau_2\cdots\tau_m\sigma=e$, 其中 $m$ 与 $\sigma$ 的反序数 $k$ 有相同的奇偶性. $\therefore\sigma$ 的符号 $\varepsilon_\sigma=(-1)^m=(-1)^k$.
\end{note}
\begin{exercisec}[2.4.5]
    定义 $\mathbb{R}^2$ 上的二元关系 $\preceq$ 如下:
    \[(a,b)\preceq(c,d)\Leftrightarrow(a<c)\vee(a=c\wedge b\leq d).\]

    证明: 这个二元关系是 $\mathbb{R}$ 上的全序.
\end{exercisec}
\begin{proof}
    $\because a=a\wedge b\leq b,\therefore(a,b)\preceq(a,b)$, 即 $\preceq$ 满足自反性.
    
    有
    \begin{align*}
        (a,b)\preceq(c,d) & \Leftrightarrow(a<c)\vee(a=c\wedge b\leq d) \\
        & \Leftrightarrow(a<c\vee a=c)\wedge(a<c\vee b\leq d) \\
        & \Leftrightarrow(a\leq c)\wedge(a<c\vee b\leq d) \\
        & \Leftrightarrow(a\leq c\wedge a<c)\vee(a\leq c\wedge b\leq d) \\
        & \Leftrightarrow(a<c)\vee(a\leq c\wedge b\leq d).
    \end{align*}

    $\therefore$
    \begin{align*}
        & (a,b)\preceq(c,d)\wedge(c,d)\preceq(a,b) \\
        & \Rightarrow((a\leq c)\wedge(a<c\vee b\leq d))\wedge((c\leq a)\wedge(c<a\vee d\leq b)) \\
        & \Rightarrow((a\leq c)\wedge(c\leq a))\wedge((a<c\vee b\leq d)\wedge(c<a\vee d\leq b)) \\
        & \Rightarrow(a=c)\wedge((a<c\vee b\leq d)\wedge(c<a\vee d\leq b)) \\
        & \Rightarrow((a=c)\wedge(a<c\vee b\leq d))\wedge((a=c)\wedge(c<a\vee d\leq b)) \\
        & \Rightarrow((a=c\wedge a<c)\vee(a=c\wedge b\leq d))\wedge((a=c\wedge c<a)\vee(a=c\wedge d\leq b)) \\
        & \Rightarrow(a=c\wedge b\leq d)\wedge(a=c\wedge d\leq b) \\
        & \Rightarrow(a=c)\wedge(b\leq d)\wedge(d\leq b) \\
        & \Rightarrow(a=c)\wedge(b=d),
    \end{align*}

    $\therefore\ \preceq$ 满足反对称性. $\because$
    \begin{align*}
        & (a,b)\preceq(a',b')\wedge(a',b')\preceq(a'',b'') \\
        & \Rightarrow((a\leq a')\wedge(a<a'\vee b\leq b'))\wedge((a'\leq a'')\wedge(a'<a''\vee b'\leq b'')) \\
        & \Rightarrow((a\leq a')\wedge(a'\leq a''))\wedge((a<a'\vee b\leq b')\wedge(a'<a''\vee b'\leq b'')) \\
        & \Rightarrow(a\leq a'')\wedge((a<a'\wedge a'<a'')\vee (b\leq b'\wedge a'<a'')\vee(a<a'\wedge b'\leq b'')\vee(b\leq b'\wedge b'\leq b'')) \\
        & \Rightarrow(a\leq a'')\wedge(((a<a'')\vee (b\leq b'\wedge a'<a''))\vee((a<a'\wedge b'\leq b'')\vee(b\leq b''))) \\
        & \Rightarrow(a\leq a'')\wedge(a<a''\vee b\leq b'') \\
        & \Rightarrow(a,b)\preceq(a'',b''),
    \end{align*}

    $\therefore\ \preceq$ 满足传递性. $\because$
    \begin{align*}
        & ((a,b)\leq(c,d))\vee((c,d)\leq(a,b)) \\
        & \Rightarrow((a<c)\vee(a\leq c\wedge b\leq d))\vee((c<a)\vee(c\leq a\wedge d\leq b)) \\
        & \Rightarrow((c<a)\vee(a\leq c\wedge b\leq d))\vee((a<c)\vee(c\leq a\wedge d\leq b)) \\
        & \Rightarrow((c<a\vee a\leq c)\wedge(c<a\vee b\leq d))\vee((a<c\vee c\leq a)\wedge(a<c\vee d\leq b)) \\
        & \Rightarrow(1\wedge(c<a\vee b\leq d))\vee(1\wedge(a<c\vee d\leq b)) \\
        & \Rightarrow(c<a)\vee(b\leq d)\vee(a<c)\vee(d\leq b) \\
        & \Rightarrow(c<a)\vee(a<c)\vee((b\leq d)\vee(d\leq b)) \\
        & \Rightarrow(c<a)\vee(a<c)\vee1=1,
    \end{align*}

    $\therefore\ \preceq$ 是 $\mathbb{R}$ 上的全序.
\end{proof}
\begin{note}
    考虑如下定义的 $\mathbb{R}^n\ (n\in\mathbb{N})$ 上的二元关系 $\preceq_n$:
    \begin{equation}\label{eq3.5}
        (a_1,\cdots,a_n)\preceq_n(b_1,\cdots,b_n)\Leftrightarrow\begin{cases}
            (a_1<b_1)\vee(a_1=b_1\wedge((a_1,\cdots,a_{n-1})\preceq_{n-1}(b_1,\cdots,b_{n-1}))), & n>1, \\
            a_1\leq b_1, & n=1.
        \end{cases}
    \end{equation}

    在
    \[(a_1,a_2)\preceq(b_1,b_2)\Leftrightarrow(a_1<b_1)\vee(a_1=b_1\wedge a_2\leq b_2).\]

    中以 $(a_2,\cdots,a_n)$ 代 $a_2$, 以 $(b_2,\cdots,b_n)$ 代 $b_2$, 以 $\preceq_{n-1}$ 代 $\leq$, 注意到上式的证明中只用到了 ``$a_2\leq b_2$'' 中 ``$\leq$'' 的自反性, 反对称性和传递性, 可以归纳地证明: $\preceq_n$ 是全序.
    
    在第 5 章笔记中介绍的多项式字典序实际上是在 $\mathbb{N}$ 上按照式 \ref{eq3.5} 定义的 $\preceq_n$. $\therefore$ 用相同的方法可以证明: 多项式字典序是全序.
\end{note}
\begin{exercisec}[2.5.1(1)]
    假设 $|X|=n\geq1$. 证明 $X$ 的 $k\ (k\geq1)$ 元子集的个数为 $\dbinom{n}{k}$.
\end{exercisec}
\begin{proof}
    由 $\dbinom{n}{1}=n$ 得当 $k=1$ 时命题成立. 由
    \[\dbinom{1}{k}=\dbinom{1}{1-k}=\begin{cases}
        1, & k=1, \\
        0, & k>1
    \end{cases}\]
    得当 $n=1$ 时命题成立.

    设对 $1\leq s\leq n,1\leq t\leq k,s+t<n+k$, 任一 $s$ 元集合的 $t$ 元子集的个数为 $\dbinom{s}{t}$.
    
    考察 $n$ 元集合 $X$, 由归纳假定, $X$ 的前 $n-1$ 个元素构成的集合 $Y$ 有 $\dbinom{n-1}{k}$ 个 $k$ 元子集, $\dbinom{n-1}{k-1}$ 个 $k-1$ 元子集. $X$ 的 $k$ 元子集包括 $Y$ 的 $k$ 元子集以及 $Y$ 的 $k-1$ 元子集与 $X$ 的第 $n$ 个元素的并, $\therefore X$ 的 $k$ 元子集的个数为
    \[\dbinom{n-1}{k}+\dbinom{n-1}{k-1}=\dbinom{n}{k}.\qedhere\]
\end{proof}
\begin{exercisec}[2.5.3]
    令 $P[x]$ 是实系数多项式全体, 考虑映射
    \[\mathcal{D}:\begin{array}{rcl}
        P[x] & \to & P[x] \\
        f & \to & f'=\dfrac{\mathrm{d}f}{\mathrm{d}x} \\
    \end{array},\quad\mathcal{T}:\begin{array}{rcl}
        P[x] & \to & P[x] \\
        f & \to & xf \\
    \end{array}.\]

    证明: 对 $m,n\geq1$, 有
    \begin{equation}\label{eq3.6}
        \mathcal{D}^m\mathcal{T}^n=\sum\limits_{i=0}^{\min\{m,n\}}i!\dbinom{m}{i}\dbinom{n}{i}\mathcal{T}^{n-i}\mathcal{D}^{m-i}
    \end{equation}
    (约定 $\mathcal{D}^0=\mathcal{T}^0=\mathcal{E}$).
\end{exercisec}
\begin{proof}
    容易验证:
    \begin{equation}\label{eq3.7}
        \mathcal{DT}=\mathcal{TD}+\mathcal{E}.
    \end{equation}

    当 $n=1$ 时, 式 (\ref{eq3.6}) $\Rightarrow$
    \begin{equation}\label{eq3.8}
        \mathcal{D}^m\mathcal{T}=\mathcal{T}\mathcal{D}^m+m\mathcal{D}^{m-1}.
    \end{equation}

    先用数学归纳法证明式 (\ref{eq3.8}) 成立. 当 $m=1$ 时上式 (\ref{eq3.8}) 为式 (\ref{eq3.7}). 假设式
    \[\mathcal{D}^{m-1}\mathcal{T}=\mathcal{T}\mathcal{D}^{m-1}+(m-1)\mathcal{D}^{m-2}\]

    成立, 则由式 (\ref{eq3.7}) 得
    \begin{align*}
        \mathcal{D}^m\mathcal{T} & =\mathcal{D}\mathcal{D}^{m-1}\mathcal{T} \\
        & =(\mathcal{DT})\mathcal{D}^{m-1}+(m-1)\mathcal{D}^{m-1} \\
        & =(\mathcal{TD}+\mathcal{E})\mathcal{D}^{m-1}+(m-1)\mathcal{D}^{m-1} \\
        & =\mathcal{T}\mathcal{D}^m+m\mathcal{D}^{m-1}.
    \end{align*}

    $\therefore$ 式 (\ref{eq3.8}) 对 $\forall m$ 成立.

    对 $n$ 用数学归纳法证明: 式 (\ref{eq3.6}) 对 $\forall m,\forall n$ 成立. 当 $n=1$ 时由式 (\ref{eq3.8}) 对 $\forall m$ 成立得. 假设
    \begin{align*}
        \mathcal{D}^m\mathcal{T}^{n-1} & =\sum\limits_{i=0}^{\min\{m,n-1\}}i!\dbinom{m}{i}\dbinom{n-1}{i}\mathcal{T}^{n-1-i}\mathcal{D}^{m-i} \\
        & =\sum\limits_{i=0}^{\min\{m,n-1\}}i!\dfrac{m!}{i!(m-i)!}\dfrac{(n-1)!}{i!(n-1-i)!}\mathcal{T}^{n-1-i}\mathcal{D}^{m-i} \\
        & =\sum\limits_{i=0}^{\min\{m,n-1\}}\dfrac{m!(n-1)!}{(m-i)!i!(n-1-i)!}\mathcal{T}^{n-1-i}\mathcal{D}^{m-i}
    \end{align*}
    对 $\forall m$ 成立, 则由式 (\ref{eq3.8}) 得
    \begin{align*}
        \mathcal{D}^m\mathcal{T}^n & =\left(\sum\limits_{i=0}^{\min\{m,n-1\}}\dfrac{m!(n-1)!}{(m-i)!i!(n-1-i)!}\mathcal{T}^{n-1-i}\mathcal{D}^{m-i}\right)\mathcal{T} \\
        & =\sum\limits_{i=0}^{\min\{m,n-1\}}\dfrac{m!(n-1)!}{(m-i)!i!(n-1-i)!}\mathcal{T}^{n-1-i}(\mathcal{D}^{m-i}\mathcal{T}) \\
        & =\sum\limits_{i=0}^{\min\{m,n-1\}}\dfrac{m!(n-1)!}{(m-i)!i!(n-1-i)!}\mathcal{T}^{n-1-i}(\mathcal{T}\mathcal{D}^{m-i}+(m-i)\mathcal{D}^{m-i-1}).
    \end{align*}
    
    有
    \begin{align*}
        & \sum\limits_{i=0}^{\min\{m,n-1\}}\dfrac{m!(n-1)!}{(m-i)!i!(n-1-i)!}\mathcal{T}^{n-1-i}(\mathcal{T}\mathcal{D}^{m-i}+(m-i)\mathcal{D}^{m-i-1}) \\
        & =\sum\limits_{i=0}^{\min\{m,n-1\}}\dfrac{m!(n-1)!}{(m-i)!i!(n-1-i)!}\mathcal{T}^{n-i}\mathcal{D}^{m-i}+\sum\limits_{i=0}^{\min\{m,n-1\}}\dfrac{m!(n-1)!(m-i)}{(m-i)!i!(n-1-i)!}\mathcal{T}^{n-1-i}\mathcal{D}^{m-i-1}.
    \end{align*}

    当 $m>n-1$ 时, 有 $m\geq n>i$, $m-i>0$,
    \begin{align*}
        & \sum\limits_{i=0}^{\min\{m,n-1\}}\dfrac{m!(n-1)!}{(m-i)!i!(n-1-i)!}\mathcal{T}^{n-i}\mathcal{D}^{m-i}+\sum\limits_{i=0}^{\min\{m,n-1\}}\dfrac{m!(n-1)!(m-i)}{(m-i)!i!(n-1-i)!}\mathcal{T}^{n-1-i}\mathcal{D}^{m-i-1} \\
        & =\sum\limits_{i=0}^{n-1}\dfrac{m!(n-1)!}{(m-i)!i!(n-1-i)!}\mathcal{T}^{n-i}\mathcal{D}^{m-i}+\sum\limits_{i=0}^{n-1}\dfrac{m!(n-1)!}{(m-i-1)!i!(n-1-i)!}\mathcal{T}^{n-1-i}\mathcal{D}^{m-i-1} \\
        & =\sum\limits_{i=0}^{n-1}\dfrac{m!(n-1)!}{(m-i)!i!(n-1-i)!}\mathcal{T}^{n-i}\mathcal{D}^{m-i}+\sum\limits_{i=1}^n\dfrac{m!(n-1)!}{(m-i)!(i-1)!(n-i)!}\mathcal{T}^{n-i}\mathcal{D}^{m-i} \\
        & =\mathcal{T}^{n}\mathcal{D}^{m}+\mathcal{D}^{m-n}+\sum\limits_{i=1}^{n-1}\left(\dfrac{m!(n-1)!}{(m-i)!i!(n-1-i)!}+\dfrac{m!(n-1)!}{(m-i)!(i-1)!(n-i)!}\right)\mathcal{T}^{n-i}\mathcal{D}^{m-i} \\
        & =\mathcal{T}^{n}\mathcal{D}^{m}+\mathcal{D}^{m-n}+\sum\limits_{i=1}^{n-1}i!\dbinom{m}{i}\left(\dfrac{(n-1)!}{i!(n-1-i)!}+\dfrac{(n-1)!}{(i-1)!(n-1-(i-1))!}\right)\mathcal{T}^{n-i}\mathcal{D}^{m-i} \\
        & =\mathcal{T}^{n}\mathcal{D}^{m}+\mathcal{D}^{m-n}+\sum\limits_{i=1}^{n-1}i!\dbinom{m}{i}\left(\dbinom{n-1}{i}+\dbinom{n-1}{i-1}\right)\mathcal{T}^{n-i}\mathcal{D}^{m-i} \\
        & =\mathcal{T}^{n}\mathcal{D}^{m}+\mathcal{D}^{m-n}+\sum\limits_{i=1}^{n-1}i!\dbinom{m}{i}\dbinom{n}{i}\mathcal{T}^{n-i}\mathcal{D}^{m-i} \\
        & =\sum\limits_{i=0}^ni!\dbinom{m}{i}\dbinom{n}{i}\mathcal{T}^{n-i}\mathcal{D}^{m-i}=\sum\limits_{i=0}^{\min\{m,n\}}i!\dbinom{m}{i}\dbinom{n}{i}\mathcal{T}^{n-i}\mathcal{D}^{m-i}.
    \end{align*}

    当 $m\leq n-1$ 时, 有 $m<n$,
    \begin{align*}
        & \sum\limits_{i=0}^{\min\{m,n-1\}}\dfrac{m!(n-1)!}{(m-i)!i!(n-1-i)!}\mathcal{T}^{n-i}\mathcal{D}^{m-i}+\sum\limits_{i=0}^{\min\{m,n-1\}}\dfrac{m!(n-1)!(m-i)}{(m-i)!i!(n-1-i)!}\mathcal{T}^{n-1-i}\mathcal{D}^{m-i-1} \\
        & =\sum\limits_{i=0}^m\dfrac{m!(n-1)!}{(m-i)!i!(n-1-i)!}\mathcal{T}^{n-i}\mathcal{D}^{m-i}+\sum\limits_{i=0}^{m-1}\dfrac{m!(n-1)!(m-i)}{(m-i)!i!(n-1-i)!}\mathcal{T}^{n-1-i}\mathcal{D}^{m-i-1} \\
        & =\sum\limits_{i=0}^m\dfrac{m!(n-1)!}{(m-i)!i!(n-1-i)!}\mathcal{T}^{n-i}\mathcal{D}^{m-i}+\sum\limits_{i=0}^{m-1}\dfrac{m!(n-1)!}{(m-i-1)!i!(n-1-i)!}\mathcal{T}^{n-1-i}\mathcal{D}^{m-i-1} \\
        & =\sum\limits_{i=0}^m\dfrac{m!(n-1)!}{(m-i)!i!(n-1-i)!}\mathcal{T}^{n-i}\mathcal{D}^{m-i}+\sum\limits_{i=0}^{m}\dfrac{m!(n-1)!}{(m-i)!i!(n-i)!}\mathcal{T}^{n-i}\mathcal{D}^{m-i} \\
        & =\sum\limits_{i=0}^m\left(\dfrac{m!(n-1)!}{(m-i)!i!(n-1-i)!}+\dfrac{m!(n-1)!}{(m-i)!i!(n-i)!}\right)\mathcal{T}^{n-i}\mathcal{D}^{m-i} \\
        & =\sum\limits_{i=0}^mi!\dbinom{m}{i}\dbinom{n}{i}\mathcal{T}^{n-i}\mathcal{D}^{m-i}=\sum\limits_{i=0}^{\min\{m,n\}}i!\dbinom{m}{i}\dbinom{n}{i}\mathcal{T}^{n-i}\mathcal{D}^{m-i}.\qedhere
    \end{align*}
\end{proof}
\section{第2章习题}
\stepcounter{exsection}
\begin{exercise}% 1.1
    设 $V_1,V_2,V$ 是 $\mathbb{R}^n$ 中的线性包, 证明: 如果对于每一个向量 $X\in V$, 写法 $X=X_1+X_2$ 都唯一, 其中 $X_i\in V_i$, 则 $V=V_1+V_2$ 是直和.
\end{exercise}
\begin{proof}
    假设 $\exists X_0\neq0,X_0\in V_1\cap V_2$.

    $\because X_0\in V_1,\therefore-X_0\in V_1$. $\because X_0\in V_2,\therefore-X_0\in V_2$.

    $\therefore0$ 有两种表示方式:
    \[0=0+0,\quad0=X_0+(-X_0).\]

    与 $0\in V\Rightarrow$ 写法 $0=X_1+X_2$ 唯一矛盾.
\end{proof}
\begin{exercise}% 1.2
    设 $V_1,V_2,V$ 是 $\mathbb{R}^n$ 中的线性包, 且 $V\subset V_1+V_2$. 等式 $V=V_1\cap V+V_2\cap V$ 一定成立吗? 证明在 $V_1\subset V$ 的情况下等式 $V=V_1\cap V+V_2\cap V$ 一定成立.
\end{exercise}
\begin{proof}
    $\because V_1\cap V\subset V,V_2\cap V\subset V$,

    $\therefore\forall X_1\in V_1\cap V,X_2\in V_2\cap V,X_1\in V,X_2\in V\Rightarrow X_1+X_2\in V$.

    $\therefore V\supset V_1\cap V+V_2\cap V$.

    但是 $V=V_1\cap V+V_2\cap V$ 不一定成立. 考察 $\mathbb{R}^2$ 中的 $V_1=\left<(0,1)\right>,V_2=\left<(1,0)\right>,V=\left<(1,1)\right>$, 有
    \[V_1\cap V=V_2\cap V=V_1\cap V+V_2\cap V=\{0\}\neq V.\]

    前面已经证明了等式 $V\supset V_1\cap V+V_2\cap V$ 一定成立, 下面只需证明在 $V_1\subset V$ 的情况下有 $V\subset V_1\cap V+V_2\cap V$.

    $\because V_1\subset V\Rightarrow V_1\cap V=V_1,\therefore V_1\cap V+V_2\cap V=V_1+V_2\cap V$.

    $\because V\subset V_1+V_2,\therefore\forall X\in V,\exists X_1\in V_1,X_2\in V_2$ 使得
    \begin{equation}\label{eq4.1}
        X=X_1+X_2.
    \end{equation}

    假设 $X_2\notin V_2\cap V\Rightarrow X_2\in V_2\backslash V$.

    $\because V_1\subset V,\therefore X_1\in V,\therefore-X_1\in V$.

    在 (\ref{eq4.1}) 式两边加上 $-X_1$, 有
    \[X+(-X_1)=X_1-X_1+X_2=X_2.\]

    $\because X\in V,-X_1\in V,\therefore X+(-X_1)\in V$, 与 $X_2\in V_2\backslash V$ 矛盾.

    $\therefore\forall X\in V,\exists X_1\in V_1=V_1\cap V,X_2\in V_2\cap V$, 使得 $X=X_1+X_2$.

    $\therefore V\subset V_1\cap V+V_2\cap V.$
\end{proof}
\addtocounter{exercise}{2}
\begin{exercise}% 1.5
    证明: $\mathbb{R}^n$ 中的向量组 $X_1,X_2,\cdots,X_n$ 张成 $\mathbb{R}^n$ 的充要条件是它们线性无关.
\end{exercise}
\begin{proof}
    (充分性) 设 $V=\left<X_1,X_2,\cdots,X_n\right>$. 假设 $V\neq\mathbb{R}^n$.

    $\because\forall X\in V,X=\alpha_1X_1+\alpha_2X_2+\cdots+\alpha_nX_n\subset\mathbb{R}^n$, $\therefore V\subset\mathbb{R}^n$.

    取 $X_{n+1}\in\mathbb{R}^n\backslash V$, 则 $X_1,X_2,\cdots,X_n,X_{n+1}$ 线性无关.

    $\therefore\dim\left<X_1,X_2,\cdots,X_n\right>=n+1>n$, 与书上的定理 2 矛盾.

    (必要性) 假设 $X_1,X_2,\cdots,X_n$ 线性相关.

    由书上的定理 1(iii), $X_1,X_2,\cdots,X_n$ 中至少有一个向量是其余向量的线性组合.
    
    不妨设 $X_n$ 是 $X_1,X_2,\cdots,X_{n-1}$ 的线性组合.

    若 $X_1,X_2,\cdots,X_{n-1}$ 线性相关, 则由书上的定理 1(iii), $X_1,X_2,\cdots,X_{n-1}$ 中至少有一个向量是其余向量的线性组合.

    不妨设 $X_{n-1}$ 是 $X_1,X_2,\cdots,X_{n-2}$ 的线性组合, 则 $X_n,X_{n-1}$ 是 $X_1,X_2,\cdots,X_{n-2}$ 的线性组合.

    重复上述步骤, 可以得到: $\exists r$ 满足 $1\leq r<n$, $X_1,X_2,\cdots,X_r$ 线性无关, $X_{r+1},X_{r+2},\cdots,X_n$ 是 $X_1,X_2,\cdots,X_r$ 的线性组合.

    $\therefore\mathbb{R}^n=\left<X_1,X_2,\cdots,X_r\right>$.

    $\therefore\dim\mathbb{R}^n=r<n$, 与 $\mathbb{R}^n=n$ 矛盾.
\end{proof}
证明第 1.7 题需要证明一个引理.
\begin{lemma}\label{l4.1}
    若 $V_1=<\alpha_1,\alpha_2,\cdots,\alpha_s>,V_2=<\beta_1,\beta_2,\cdots,\beta_t>$,
    
    则 $V_1+V_2=<\alpha_1,\alpha_2,\cdots,\alpha_s,\beta_1,\beta_2,\cdots,\beta_t>$.
\end{lemma}
\begin{proof}
    $\forall\alpha+\beta\in V_1+V_2$(不妨设 $\alpha\in V_1,\beta\in V_2$), 有
    \[\alpha=l_1\alpha_1+l_2\alpha_2+\cdots+l_s\alpha_s,\quad\beta=k_1\beta_1+k_2\beta_2+\cdots+k_t\beta_t,\]
    \[\alpha+\beta=l_1\alpha_1+l_2\alpha_2+\cdots+l_s\alpha_s+k_1\beta_1+k_2\beta_2+\cdots+k_t\beta_t.\]

    $\therefore$
    \[\alpha+\beta\in<\alpha_1,\alpha_2,\cdots,\alpha_s,\beta_1,\beta_2,\cdots,\beta_t>.\]
    
    $\therefore$
    \[(V_1+V_2)\subseteq<\alpha_1,\alpha_2,\cdots,\alpha_s,\beta_1,\beta_2,\cdots,\beta_t>.\]

    $\forall\gamma\in<\alpha_1,\alpha_2,\cdots,\alpha_s,\beta_1,\beta_2,\cdots,\beta_t>$, 有
    \[\gamma=l_1\alpha_1+l_2\alpha_2+\cdots+l_s\alpha_s+l_{s+1}\beta_1+l_{s+2}\beta_2+\cdots+l_{s+t}\beta_t.\]

    令 $\alpha=l_1\alpha_1+l_2\alpha_2+\cdots+l_s\alpha_s,\beta=l_{s+1}\beta_1+l_{s+2}\beta_2+\cdots+l_{s+t}\beta_t$, 有
    \begin{align*}
        \alpha\in V_1\land\beta\in V_2\land\gamma=\alpha+\beta & \Rightarrow\gamma\in V_1+V_2 \\
        & \Rightarrow(V_1+V_2)\supseteq<\alpha_1,\alpha_2,\cdots,\alpha_s,\beta_1,\beta_2,\cdots,\beta_t>.
    \end{align*}

    $\therefore(V_1+V_2)=<\alpha_1,\alpha_2,\cdots,\alpha_s,\beta_1,\beta_2,\cdots,\beta_t>.$
\end{proof}
\stepcounter{exercise}
\begin{exercise}[有改动]\label{ex1.7}
设 $V_1$ 和 $V_2$ 都是 $\mathbb{R}^n$ 中的线性包,证明:

$$\mathrm{dim}(V_1+V_2)=\mathrm{dim}V_1+\mathrm{dim}V_2-\mathrm{dim}(V_1\cap V_2).$$
\end{exercise}
\begin{proof}
    由书上定理 2 的证明过程得 $V_1\cap V_2$ 的一个基
    \[\alpha_1,\alpha_2,\cdots,\alpha_s\]
    可以扩充成 $V_1$ 的一个基
    \[\alpha_1,\alpha_2,\cdots,\alpha_s,\beta_1,\beta_2,\cdots,\beta_{m-s}\]
    和 $V_2$ 的一个基
    \[\alpha_1,\alpha_2,\cdots,\alpha_s,\gamma_1,\gamma_2,\cdots,\gamma_{n-s}.\]

    由引理 \ref{l4.1},
    \begin{align*}
        V_1+V_2 & =<\alpha_1,\alpha_2,\cdots,\alpha_s,\beta_1,\beta_2,\cdots,\beta_{m-s}>+<\alpha_1,\alpha_2,\cdots,\alpha_s,\gamma_1,\gamma_2,\cdots,\gamma_{n-s}> \\
        & =<\alpha_1,\alpha_2,\cdots,\alpha_s,\beta_1,\beta_2,\cdots,\beta_{m-s},\alpha_1,\alpha_2,\cdots,\alpha_s,\gamma_1,\gamma_2,\cdots,\gamma_{n-s}> \\
        & =<\alpha_1,\alpha_2,\cdots,\alpha_s,\beta_1,\beta_2,\cdots,\beta_{m-s},\gamma_1,\gamma_2,\cdots,\gamma_{n-s}>.
    \end{align*}

    设
    \begin{equation}\label{eq4.2}
        \sum\limits_{i=1}^sp_i\alpha_i+\sum\limits_{i=1}^{m-s}q_i\beta_i+\sum\limits_{i=1}^{n-s}r_i\gamma_i=0.
    \end{equation}

    则有
    \[\sum\limits_{i=1}^sp_i\alpha_i+\sum\limits_{i=1}^{m-s}q_i\beta_i=\sum\limits_{i=1}^{n-s}(-r_i)\gamma_i.\]

    设 $\zeta=(-r_1)\gamma_1+(-r_2)\gamma_2+\cdots+(-r_{n-s})\gamma_{n-s}\in V_2$. 由式 (\ref{eq4.2}) 得
    \[\zeta=\sum\limits_{i=1}^sp_i\alpha_i+\sum\limits_{i=1}^{m-s}q_i\beta_i\in V_1.\]

    $\therefore\zeta\in V_1\cap V_2$. 设 $\zeta=l_1\alpha_1+l_2\alpha_2+\cdots+l_s\alpha_s$. 则有
    \[\sum\limits_{i=1}^sl_i\alpha_i=\sum\limits_{i=1}^{n-s}(-r_i)\gamma_i\Rightarrow\sum\limits_{i=1}^sl_i\alpha_i+\sum\limits_{i=1}^{n-s}r_i\gamma_i=0.\]

    $\because\alpha_1,\alpha_2,\cdots,\alpha_s,\gamma_1,\gamma_2,\cdots,\gamma_{n-s}$ 是 $V_2$ 的一个基,

    $\therefore l_1=l_2=\cdots=l_s=r_1=r_2=\cdots=r_{n-s}=0$.

    把 $r_1=r_2=\cdots=r_{n-s}=0$ 代入式 (\ref{eq4.2}) 得
    \[\sum\limits_{i=1}^sp_i\alpha_i+\sum\limits_{i=1}^{m-s}q_i\beta_i=0.\]

    $\because\alpha_1,\alpha_2,\cdots,\alpha_s,\beta_1,\beta_2,\cdots,\beta_{m-s}$ 是 $V_1$ 的一个基,

    $\therefore p_1=p_2=\cdots=p_s=q_1=q_2=\cdots=q_{m-s}=0$.

    $\therefore$ 向量组 $\alpha_1,\alpha_2,\cdots,\alpha_s,\beta_1,\beta_2,\cdots,\beta_{m-s},\gamma_1,\gamma_2,\cdots,\gamma_{n-s}$ 线性无关.

    $\therefore\alpha_1,\alpha_2,\cdots,\alpha_s,\beta_1,\beta_2,\cdots,\beta_{m-s},\gamma_1,\gamma_2,\cdots,\gamma_{n-s}$ 是 $V_1+V_2$ 的一个基.

    $\therefore$
    \begin{align*}
        \mathrm{dim}(V_1+V_2) & =s+(m-s)+(n-s) \\
        & =m+n-s \\
        & =\mathrm{dim}V_1+\mathrm{dim}V_2-\mathrm{dim}(V_1\cap V_2).\qedhere
    \end{align*}
\end{proof}
\stepcounter{exsection}
\setcounter{exercise}{3}
\begin{exercise}% 2.4
    设两个矩阵
    \[A=\begin{pmatrix}
        \alpha_1 & \alpha_2 & \cdots & \alpha_n \\
        \beta_1 &  \beta_2 & \cdots &  \beta_n
    \end{pmatrix},\quad B=\begin{pmatrix}
        \alpha_1 & \alpha_2 & \cdots & \alpha_n \\
        \beta_1 &  \beta_2 & \cdots &  \beta_n \\
        \gamma_1 & \gamma_2 & \cdots & \gamma_n
    \end{pmatrix},\]

    用平面上 $n$ 条直线组成的集合的几何性质给出 $A$ 和 $B$ 有相等秩的条件.
\end{exercise}
\begin{solution}
    平面上 $n$ 条直线组成的集合为
    \[\{\alpha_ix+\beta_iy=\gamma_i|\alpha_i,\beta_i,\gamma_i\in\mathbb{R},i=1,2,\cdots,n\}.\]

    如果这 $n$ 条直线有公共点 $(p,q)$, 则
    \[p\alpha_i+q\beta_i=\gamma_i\quad(i=1,2,\cdots,n).\]

    $\therefore$
    \[pB_{(1)}+qB_{(2)}=B_{(3)}.\]

    $\therefore$
    \begin{align*}
        \left<B_{(1)},B_{(2)},B_{(3)}\right> & =\left<B_{(1)},B_{(2)}\right> \\
        & =\left<A_{(1)},A_{(2)}\right>.
    \end{align*}

    $\therefore\operatorname{rank}B=\operatorname{rank}A$.
\end{solution}
\stepcounter{exsection}
\setcounter{exercise}{3}
\begin{exercise}% 3.4
    由 Markov 矩阵
    \[P=(p_{ij}),\quad p_{ij}\geq0,\quad\sum\limits_{j=1}^np_{ij}=1,\quad i=1,2,\cdots,n\]
    
    确定的线性变换 $\phi_P$ 通常作用于概率列向量
    \[X=[x_1,\cdots,x_n],\quad x_i\geq0,\quad\sum\limits_{i=1}^nx_i=1.\]
    
    证明:
    \begin{itemize}
        \item[(a)] 矩阵 $P\in M_n(\mathbb{R})$ 是 Markov 的, 当且仅当对任一概率向量 $X$, 向量 $PX$ 仍然是概率向量.
        \item[(b)] 如果 $P$ 是正的 Markov 矩阵($\forall i,j,p_{ij}>0$), 那么对任一概率向量 $X,PX$ 是正的概率向量.
        \item[(c)] 如果矩阵 $P,Q$ 是 Markov 的, 那么矩阵 $A=PQ$ 也是 Markov 的.
    \end{itemize}
\end{exercise}
\begin{proof}
    (a) 记
    \[PX=[y_1,y_2,\cdots,y_n],\]

    由书上的式 (11) 得
    \[y_i=P_{(i)}X=\sum\limits_{j=1}^np_{ij}x_{j}\geq0,\]
    \begin{align*}
        \sum\limits_{i=1}^ny_i & =\sum\limits_{i=1}^n\sum\limits_{j=1}^np_{ij}x_{j} \\
        & =\sum\limits_{j=1}^nx_{j}\sum\limits_{i=1}^np_{ij} \\
        & =\sum\limits_{j=1}^nx_{j}=1.
    \end{align*}

    (b) $\because x_i\geq0,\sum\limits_{i=1}^nx_i=1,\therefore\exists k$ 满足 $1\leq k\leq n$ 使得 $x_k>0.\therefore$
    \[y_i=\sum\limits_{j=1}^np_{ij}x_{j}\geq p_{ik}x_k>0.\]

    (c) $\because Q$ 是 Markov 的, $\therefore Q^{(i)}\ (i=1,2,\cdots,n)$ 是概率向量.

    由 (a) 得 $PQ^{(i)}$ 是概率向量. $\therefore PQ$ 也是 Markov 的.
\end{proof}
\stepcounter{exercise}
\begin{exercise}% 3.6
    由 $S_n$ 中的 $n$ 阶循环确定的置换矩阵为
    \[P=\begin{pmatrix}
        0 & 0 & \cdots & 0 & 1 \\
        1 & 0 & \cdots & 0 & 0 \\
        0 & 1 & \cdots & 0 & 0 \\
        \vdots & \vdots & \ddots & \vdots & \vdots \\
        0 & 0 & \cdots & 1 & 0 \\
    \end{pmatrix}.\]

    验证 $P^n=E$.
\end{exercise}
\begin{solution}
    $\because PE^{(1)}=E^{(2)},PE^{(2)}=E^{(3)},\cdots,PE^{(n-1)}=E^{(n)},PE^{(n)}=E^{(1)},\therefore$
    \[P^{n}E^{(1)}=P^{n-1}E^{(2)}=\cdots=PE^{(n)}=E^{(1)},\]
    \[P^{n}E^{(2)}=P^{n-1}E^{(3)}=\cdots=PE^{(1)}=E^{(2)},\]
    \[\cdots,\]
    \[P^{n}E^{(n)}=P^{n-1}E^{(1)}=\cdots=PE^{(n-1)}=E^{(n)},\]

    $\therefore P^n=E$.
\end{solution}
\begin{exercise}% 3.7
    对于任意两个 $m\times n$ 矩阵 $A,B$, 证明
    \[\operatorname{rank}(A+B)\leq\operatorname{rank}A+\operatorname{rank}B.\]
\end{exercise}
\begin{proof}
    在等式
    \begin{equation}
        \sum\limits_{i=1}^n\lambda_iA^{(j)}=0
    \end{equation}

    两边乘一非零常数 $\mu$, 等式仍然成立. $\therefore$ 由式 $(3)$ 得
    \[\sum\limits_{i=1}^n\mu\lambda_iA^{(j)}=0\Rightarrow\sum\limits_{i=1}^n\lambda_i(\mu A^{(j)})=0.\]

    $\therefore$ 对向量组的每个向量乘一非零常数不改变向量组的线性相关性. $\therefore$
    \[\operatorname{rank}(-B)=\operatorname{rank}B.\]

    假设 $\operatorname{rank}(A+B)>\operatorname{rank}A+\operatorname{rank}B$, 则
    \begin{align*}
        \operatorname{rank}A & =\operatorname{rank}(A+B-B) \\
        & >\operatorname{rank}(A+B)+\operatorname{rank}(-B) \\
        & >\operatorname{rank}A+\operatorname{rank}B+\operatorname{rank}(-B) \\
        & =\operatorname{rank}A+2\operatorname{rank}B.
    \end{align*}

    $\therefore 0>2\operatorname{rank}B$, 与 $\operatorname{rank}B\geq0$ 矛盾.
\end{proof}
证明第 \ref{ex3.8} 题需要证明两个引理.
\begin{lemma}\label{l4.2}
    是 $A$ 是 $m\times s$ 矩阵, $B$ 是 $s\times n$ 矩阵, $A$ 确定的线性映射为 $\phi_A$, $B$ 的列空间为 $V_c(B)$, 则
    \[\dim\phi_A(V_c(B))=\operatorname{rank}AB.\]
\end{lemma}
\begin{proof}
    设 $AB$ 的列空间为 $V_c(AB)$, 有
    \[\dim V_c(AB)=\operatorname{rank}AB.\]

    $\therefore$ 只需证明 $V_c(AB)=\phi_A(V_c(B))$.

    $\because\phi_A(B^{(i)})=AB^{(i)}\ (i=1,2,\cdots,s),$

    $\therefore V_c(AB)=\left<AB^{(1)},AB^{(2)},\cdots,AB^{(s)}\right>=\phi_A(V_c(B))$.
\end{proof}
\begin{lemma}\label{l4.3}
    设 $\phi$ 是线性映射, $V_1,V_2$ 是线性包, 则
    \[\phi(V_1+V_2)=\phi(V_1)+\phi(V_2).\]
\end{lemma}
\begin{proof}
    $\forall Y'\in\phi(V_1+V_2),\exists Y\in V_1+V_2,\phi(Y)=Y'$.

    $\because Y\in V_1+V_2,\therefore\exists X_1\in V_1,X_2\in V_2$ 使得 $Y=X_1+X_2$ (注意这样的 $X_1,X_2$ 不一定是唯一的, 不过这不影响证明, 下面的 $\exists X_1',X_2'$ 也一样). $\therefore$
    \[Y'=\phi(Y)=\phi(X_1+X_2)=\phi(X_1)+\phi(X_2).\]

    $\therefore\phi(V_1+V_2)\subset\phi(V_1)+\phi(V_2)$.

    $\forall Y'\in\phi(V_1)+\phi(V_2),\exists X_1'\in\phi(V_1),X_2'\in\phi(V_2),Y'=X_1'+X_2'$.

    $\because\exists X_1\in V_1,X_2\in V_2,\phi(X_1)=X_1',\phi(X_2)=X_2'$.

    $\therefore\exists Y=X_1+X_2,Y\in V_1+V_2,$
    \[\phi(Y)=\phi(X_1+X_2)=\phi(X_1)+\phi(X_2)=X_1'+X_2'=Y.\]

    $\therefore\phi(V_1+V_2)\supset\phi(V_1)+\phi(V_2)$.

    $\therefore\phi(V_1+V_2)=\phi(V_1)+\phi(V_2)$.
\end{proof}
\begin{exercise}\label{ex3.8}
    对于任意 $m\times s$ 矩阵 $A$ 和 $s\times n$ 矩阵 $B$, 证明
    \[\operatorname{rank}A+\operatorname{rank}B-s\leq\operatorname{rank} AB.\]
\end{exercise}
下面是比较容易想到的证明方法(至少是我自己想到的方法).
\begin{thought}
    首先我们需要处理不等式中看似与矩阵的秩无关的 $s$ (事实上, 对 $s$ 的不同处理方法会导出不同的证明, 比如说将 $s$ 看成是 $M_s(\mathbb{R})$ 中的单位矩阵的秩就可以得到一般的教科书上的利用分块矩阵的证明). 这里我们利用等式 $s=\dim\mathbb{R}^s$ 将不等式变为
    \[\operatorname{rank}A+\operatorname{rank}B-\dim\mathbb{R}^s\leq\operatorname{rank} AB.\]

    然后我们把矩阵的秩都转化为线性包的维数. 注意到:
    \begin{enumerate}
        \item $V_c(B)\subset\dim\mathbb{R}^s$,
        \item 设 $A$ 确定的线性映射为 $\phi_A$, 则 $\operatorname{rank}A=\dim\phi_A(\mathbb{R}^s),V_c(AB)=\phi_A(V_c(B))$,
    \end{enumerate}

    将不等式变为
    \begin{align*}
        \dim\phi_A(\mathbb{R}^s)-\dim\phi_A(V_c(B)) & \leq\dim\mathbb{R}^s-\dim V_c(B) \\
        & =\dim\mathbb{R}^s\backslash V_c(B).
    \end{align*}

    如果能证明
    \[\dim\phi_A(\mathbb{R}^s)-\dim\phi_A(V_c(B))\leq\dim\phi_A(\mathbb{R}^s\backslash V_c(B)),\]

    那么只需证明
    \[\dim\phi_A(\mathbb{R}^s\backslash V_c(B))\leq\dim\mathbb{R}^s\backslash V_c(B).\]
\end{thought}
\begin{proof}
    设 $A$ 确定的线性映射为 $\phi_A$, $V_c(B)\subset\dim\mathbb{R}^s$ 为 $B$ 的列空间.

    令 $C$ 是满足 $V_c(C)+V_c(B)=\mathbb{R}^s$ 的矩阵(注意这样的 $C$ 不一定是唯一的, 不过这不影响证明). 由书上的定理 3 得
    \[\operatorname{rank}AC\leq\operatorname{rank}C.\]

    由引理 \ref{l4.2} 得
    \[\operatorname{rank}AC=\dim\phi_A(V_c(C)),\]

    $\therefore$
    \begin{equation}\label{eq4.4}
        \dim\phi_A(V_c(C))\leq\dim V_c(C).
    \end{equation}

    $\because V_c(C)+V_c(B)=\mathbb{R}^s$, 由引理 \ref{l4.3} 得
    \[\phi_A(V_c(C))+\phi_A(V_c(B))=\phi_A(V_c(C)+V_c(B))=\phi_A(\mathbb{R}^s).\]

    由习题 \ref{ex1.7} 得
    \begin{align*}
        \dim\phi_A(\mathbb{R}^s) & =\dim(\phi_A(V_c(C))+\phi_A(V_c(B))) \\
        & =\dim\phi_A(V_c(C))+\dim\phi_A(V_c(B))-\dim(\phi_A(V_c(C))\cap\phi_A(V_c(B))) \\
        & \leq\dim\phi_A(V_c(C))+\dim\phi_A(V_c(B)).
    \end{align*}

    由 (\ref{eq4.4}) 式得
    \begin{equation}\label{eq4.5}
        \dim\phi_A(\mathbb{R}^s)\leq\dim V_c(C)+\dim\phi_A(V_c(B)).
    \end{equation}

    由引理 \ref{l4.2} 得
    \[\dim\phi_A(\mathbb{R}^s)=\operatorname{rank}AE=\operatorname{rank}A,\]
    \[\dim\phi_A(V_c(B))=\operatorname{rank}AB,\]

    代入 (\ref{eq4.5}) 式得
    \begin{equation}
        \operatorname{rank}A\leq\dim V_c(C)+\operatorname{rank}AB.
    \end{equation}

    由习题 \ref{ex1.7} 得
    \begin{align*}
        & V_c(C)+V_c(B)=\mathbb{R}^s \\
        \Rightarrow & \dim (V_c(C)+V_c(B))=\dim\mathbb{R}^s \\
        \Rightarrow & \dim V_c(C)+\dim V_c(B)\leq\dim\mathbb{R}^s \\
        \Rightarrow & \dim V_c(C)\leq\dim\mathbb{R}^s-\dim V_c(B)=s-\operatorname{rank}B.
    \end{align*}

    代入 (\ref{eq4.5}) 式得
    \begin{align*}
        & \operatorname{rank}A\leq s-\operatorname{rank}B+\operatorname{rank}AB \\
        \Rightarrow & \operatorname{rank}A+\operatorname{rank}B-s\leq\operatorname{rank}AB.\qedhere
    \end{align*}
\end{proof}
\setcounter{exercise}{13}
\begin{exercise}% 3.14
    设
    \[A=\begin{pmatrix}
        a & b \\
        c & d
    \end{pmatrix}.\]
    
    利用关系 $A^2=(a+d)A-(ad-bc)E$ 求解 $A^{-1}$.
\end{exercise}
\begin{solution}
    \begin{align*}
        & A^2=(a+d)A-(ad-bc)E \\
        \Rightarrow & A^2=(a+d)AE-(ad-bc)E \\
        \Rightarrow & A^2+(-a-d)AE=-(ad-bc)E.
    \end{align*}

    由线性映射的性质(ii),
    \[(-a-d)AE=A(-a-d)E=A\operatorname{diag}(-a-d,-a-d).\]

    $\therefore$ 原式
    \[\Rightarrow A^2+A\operatorname{diag}(-a-d,-a-d)=-(ad-bc)E.\]

    由线性映射的性质(i), 原式
    \[\Rightarrow A(A-(a+d)E)=-(ad-bc)E.\]

    由线性映射的性质(ii), 原式
    \[\Rightarrow A\left(-\dfrac{1}{ad-bc}A+\dfrac{a+d}{ad-bc}E\right)=E.\]

    $\therefore$
    \begin{align*}
        A^{-1} & =\left(-\dfrac{1}{ad-bc}A+\dfrac{a+d}{ad-bc}E\right) \\
        & =\begin{pmatrix}
            -\dfrac{a}{ad-bc}+\dfrac{a+d}{ad-bc} & -\dfrac{b}{ad-bc} \\[8pt]
            -\dfrac{c}{ad-bc} & -\dfrac{d}{ad-bc}+\dfrac{a+d}{ad-bc}
        \end{pmatrix} \\
        & =\begin{pmatrix}
            \dfrac{d}{ad-bc} & -\dfrac{b}{ad-bc} \\[8pt]
            -\dfrac{c}{ad-bc} & \dfrac{a}{ad-bc}
        \end{pmatrix}.
    \end{align*}
\end{solution}
\stepcounter{exercise}
\begin{exercise}% 3.16
    证明: 若
    \[\begin{pmatrix}
        a & b \\
        c & d
    \end{pmatrix}^m=0,\]

    则
    \[\begin{pmatrix}
        a & b \\
        c & d
    \end{pmatrix}^2=0.\]
\end{exercise}
\begin{proof}
    设
    \[A=\begin{pmatrix}
        a & b \\
        c & d
    \end{pmatrix}.\]

    由第 \ref{ex3.8} 题和书上第 3 节的定理 3 得
    \[\operatorname{rank}A+\operatorname{rank}A-2\leq\operatorname{rank}A^2\leq\operatorname{rank}A,\]

    \[\operatorname{rank}A^2+\operatorname{rank}A-2\leq\operatorname{rank}A^3\leq\operatorname{rank}A^2,\]

    \[\cdots,\]

    \[\operatorname{rank}A^{n-1}+\operatorname{rank}A-2\leq\operatorname{rank}A^3\leq\operatorname{rank}A^{n-1},\]

    假设 $\operatorname{rank}A=2$, 则 $\operatorname{rank}A^m=2$, 与 $A^m=0\Rightarrow\operatorname{rank}A^m=0$ 矛盾. $\therefore\operatorname{rank}A\leq 1$.

    若 $A$ 有一列全为 $0$, 不妨设 $a=c=0$, 则
    \[A^2=\begin{pmatrix}
        0 & bd \\
        0 & d^2
    \end{pmatrix},\quad A^3=\begin{pmatrix}
        0 & bd^2 \\
        0 & d^3
    \end{pmatrix},\quad\cdots,\quad A^m=\begin{pmatrix}
        0 & bd^{m-1} \\
        0 & d^m
    \end{pmatrix}.\]

    若 $A^m=0$, 则 $d=0,A^2=0$.

    假设 $A$ 的两列都不全为 $0$. $\because\operatorname{rank}A\leq1,\therefore A^{(1)},A^{(2)}$ 线性相关.

    设 $A^{(1)}=[a,c],A^{(2)}=\lambda A^{(1)}\ (\lambda\neq0)$, 则
    \begin{align*}
        A^2 & =(AA^{(1)},AA^{(2)}) \\
        & =(AA^{(1)},\lambda AA^{(1)}) \\
        & =((A^{(1)},A^{(2)})[a,c],\lambda(A^{(1)},A^{(2)})[a,c]) \\
        & =(aA^{(1)}+cA^{(2)},\lambda(aA^{(1)}+cA^{(2)})) \\
        & =((a+\lambda c)A^{(1)},\lambda(a+\lambda c)A^{(1)}) \\
        & =(\mu A^{(1)},\lambda\mu A^{(1)}) \\
        & =\mu A\quad\mu \in\mathbb{R},
    \end{align*}
    \begin{align*}
        A^3 & =(A(A^2)^{(1)},A(A^2)^{(2)}) \\
        & =(\mu AA^{(1)},\lambda\mu AA^{(1)}) \\
        & =\mu(AA^{(1)},\lambda AA^{(1)}) \\
        & =\mu(\mu A^{(1)},\lambda\mu A^{(1)}) \\
        & =\mu^2 A,
    \end{align*}
    \[\cdots,\]
    \[A^m=\mu^{m-1} A.\]

    $\because\lambda\neq0,A^{(1)}\neq0,\therefore A^m=0\Rightarrow \mu^{m-1}=0\Rightarrow \mu=0\Rightarrow\cdots\Rightarrow A^2=0$.
\end{proof}
\begin{exercise}% 3.17
    设 $m\times s$ 矩阵
    \[X=\begin{pmatrix}
        X_{11} & X_{12} & \cdots & X_{1k} \\
        X_{21} & X_{22} & \cdots & X_{2k} \\
        \vdots & \vdots & \ddots & \vdots \\
        X_{l1} & X_{l2} & \cdots & X_{lk}
    \end{pmatrix}\]

    的块 $X_{ij}$ 是 $m_i\times s_j$ 矩阵 $(m_1+m_2+\cdots+m_l=m,s_1+s_2+\cdots+s_k=s)$, $s\times n$ 矩阵
    \[Y=\begin{pmatrix}
        Y_{11} & Y_{12} & \cdots & Y_{1r} \\
        Y_{21} & Y_{22} & \cdots & Y_{2r} \\
        \vdots & \vdots & \ddots & \vdots \\
        Y_{k1} & Y_{k2} & \cdots & Y_{kr} \\
    \end{pmatrix}\]

    的块 $Y_{ij}$ 是 $s_i\times n_j$ 矩阵 $(n_1+n_2+\cdots+n_r=n)$, 则 $Z=XY$ 可以分块计算, 它的块

    \[Z_{ij}=\sum\limits_{t=1}^kX_{it}Y_{tj}.\]
\end{exercise}
\begin{proof}
    将 $X,Y,Z$ 分成下列矩阵的和:
    \[X=\sum\limits_{i=1}^{l}\sum\limits_{j=1}^{k}{}^{(ij)}X,\]
    \[Y=\sum\limits_{i=1}^{k}\sum\limits_{j=1}^{r}{}^{(ij)}Y,\]
    \[Z=\sum\limits_{i=1}^{l}\sum\limits_{j=1}^{r}{}^{(ij)}Z,\]

    其中 ${}^{(ij)}X$ 的块 ${}^{(ij)}X_{ij}=X_{ij}$, 其余块都是零矩阵, ${}^{(ij)}X_{ut}=X_{ut}$, ${}^{(ij)}Y$ 的块 ${}^{(ij)}Y_{ij}=Y_{ij}$, 其余块都是零矩阵, ${}^{(ij)}Z$ 的块 ${}^{(ij)}Z_{ij}=Z_{ij}$, 其余块都是零矩阵.

    由矩阵乘法的分配律得
    \[Z=XY=\sum\limits_{a=1}^{l}\sum\limits_{b=1}^{k}\sum\limits_{c=1}^{k}\sum\limits_{d=1}^{r}{}^{(ab)}X{}^{(cd)}Y.\]

    考察 $b\neq c$ 的情况.

    定义 $s_{(0)}=0,s_{(i)}=s_1+s_2+\cdots+s_i$ (对 $m,n$ 也有类似的定义), 则 ${}^{(ab)}X{}^{(cd)}Y$ 的每一列都可以看成是 ${}^{(ab)}X$ 的第 $s_{(c-1)}+1,s_{(c-1)}+2,\cdots,s_{(c)}$ 列的线性组合.

    $\because b\neq c,\therefore{}^{(ab)}X$ 的第 $s_{(c-1)}+1,s_{(c-1)}+2,\cdots,s_{(c)}$ 列都是 $0$.

    $\therefore b\neq c\Rightarrow{}^{(ab)}X{}^{(cd)}Y=0$,
    \begin{align*}
        Z & =\sum\limits_{a=1}^{l}\sum\limits_{b=1}^{k}\sum\limits_{d=1}^{r}{}^{(ab)}X{}^{(bd)}Y \\
        & =\sum\limits_{a=1}^{l}\sum\limits_{d=1}^{r}\left(\sum\limits_{b=1}^{k}{}^{(ab)}X{}^{(bd)}Y\right),
    \end{align*}
    \[Z=\sum\limits_{a=1}^{l}\sum\limits_{d=1}^{r}{}^{(ad)}Z,\]

    $\therefore$
    \begin{equation}\label{eq4.7}
        \sum\limits_{a=1}^{l}\sum\limits_{d=1}^{r}\left({}^{(ad)}Z-\sum\limits_{b=1}^{k}{}^{(ab)}X{}^{(bd)}Y\right)=0.
    \end{equation}

    考察 $^{(ad)}Z'=\sum\limits_{b=1}^{k}{}^{(ab)}X{}^{(bd)}Y$ 的块 $^{(ad)}Z'_{ij}$. 设 $i\neq a$, 则 $\sum\limits_{b=1}^{k}{}^{(ab)}X{}^{(bd)}Y$ 中的每一个 $^{(ab)}X$ 的第 $m_{(i-1)}+1,m_{(i-1)}+2,\cdots,m_{(i)}$ 行都为 $0$,

    $\therefore{}^{(ad)}Z'$ 的第 $m_{(i-1)}+1,m_{(i-1)}+2,\cdots,m_{(i)}$ 行都为 $0$, $\therefore Z'_{ij}=0$.

    同理, 若 $j\neq d$, 则 $Z'_{ij}=0$.

    $\therefore{}^{(ad)}Z'$ 的块 $^{(ad)}Z'_{ij}$ 中除 $^{(ad)}Z'_{ad}$ 外都是零矩阵.

    $\therefore{}^{(ad)}Z-{}^{(ad)}Z'$ 的块 $^{(ad)}Z_{ij}-^{(ad)}Z'_{ij}$ 中除 $^{(ad)}Z_{ad}-^{(ad)}Z'_{ad}$ 外都是零矩阵.

    假设 $\exists$ 非空集合 $U$ 使得 $\forall (a,d)\in U,{}^{(ad)}Z'\neq{}^{(ad)}Z,\forall (a,d)\in[1,l]\times[1,r]\backslash U,{}^{(ad)}Z'={}^{(ad)}Z$.

    $\because\forall (a,d)\in U,$
    \[{}^{(ad)}Z'-{}^{(ad)}Z=\sum\limits_{i=s(a-1)}^{s(a)}\sum\limits_{j=s(d-1)}^{s(d)}\lambda_{ij}E_{ij},\]

    其中 $E_{ij}$ 为矩阵单位, 定义见书上的定理 4, $\therefore$
    \begin{align*}
        {}^{(ad)}Z'-{}^{(ad)}Z\in & \left<E_{s(a-1)s(d-1)},E_{s(a-1),s(d-1)+1},\cdots,E_{s(a-1)s(d)}\right. \\
        & E_{s(a-1)+1,s(d-1)},E_{s(a-1)+1,s(d-1)+1},\cdots,E_{s(a-1)+1,s(d)}, \\
        & \cdots, \\
        & \left.E_{s(a)s(d-1)},E_{s(a),s(d-1)+1},\cdots,E_{s(a),s(d)}\right>
    \end{align*}
    
    $\therefore$ 集合 $\{{}^{(ad)}Z-{}^{(ad)}Z'|(a,d)\in U\}$ 线性无关.

    由 (\ref{eq4.7}) 式,
    \[\sum\limits_{a=1}^{l}\sum\limits_{d=1}^{r}({}^{(ad)}Z'-{}^{(ad)}Z)=\sum\limits_{(a,d)\in U}({}^{(ad)}Z'-{}^{(ad)}Z)=0,\]

    即关于 $x$ 的方程
    \[\sum\limits_{(a,d)\in U}x_{ad}({}^{(ad)}Z'-{}^{(ad)}Z)=0\]

    有非零解 $\{x_{ad}=1|(a,d)\in U\}$, 与集合 $\{{}^{(ad)}Z-{}^{(ad)}Z'|(a,d)\in U\}$ 线性无关矛盾.

    $\therefore{}^{(ad)}Z'={}^{(ad)}Z,$
    \[^{(ij)}Z_{ij}=\sum\limits_{t=1}^k{}^{(it)}X_{it}{}^{(tj)}Y_{tj}.\]

    $\therefore$
    \[Z_{ij}=\sum\limits_{t=1}^kX_{it}Y_{tj}.\qedhere\]
\end{proof}
\begin{note}
    把 $^{(ad)}Z$ 之类的矩阵换成矩阵单位, 可以用类似的方法推导矩阵的乘法公式.
\end{note}
\begin{exercisec}[第 4 章的习题 3.9]
    设 $A\in M_n(\mathbb{R}),B\in M_m(\mathbb{R})$ 可逆, $C$ 是任意的 $n\times m$ 矩阵, 证明
    \[\begin{pmatrix}
        A & C \\
        0 & B
    \end{pmatrix}\]

    可逆, 并求其逆.
\end{exercisec}
\begin{proof}
    $\because$
    \[\begin{pmatrix}
        A^{-1} & 0 \\
        0 & B^{-1}
    \end{pmatrix}\begin{pmatrix}
        A & C \\
        0 & B
    \end{pmatrix}=\begin{pmatrix}
        E_n & A^{-1}C \\
        0 & E_m
    \end{pmatrix},\]

    而矩阵
    \[\begin{pmatrix}
        E_n & A^{-1}C \\
        0 & E_m
    \end{pmatrix}\]

    经过初等行变换可以得到 $E_{m+n}$, $\therefore$ 原矩阵可逆.

    考察矩阵
    \[\begin{pmatrix}
        E_n & A^{-1}C \\
        0 & E_m
    \end{pmatrix}.\]

    将其第 $2$ 行的 $-A^{-1}C$ 倍加到第 $1$ 行得到 $E_{m+n}$. 可以把初等变换写成左乘分块初等矩阵的形式:
    \[\begin{pmatrix}
        E_n & -A^{-1}C \\
        0 & E_m
    \end{pmatrix}\begin{pmatrix}
        E_n & A^{-1}C \\
        0 & E_m
    \end{pmatrix}=E_{m+n}.\]

    $\therefore$
    \[\begin{pmatrix}
        E_n & -A^{-1}C \\
        0 & E_m
    \end{pmatrix}\begin{pmatrix}
        A^{-1} & 0 \\
        0 & B^{-1}
    \end{pmatrix}\begin{pmatrix}
        A & C \\
        0 & B
    \end{pmatrix}=E_{m+n},\]
    \begin{align*}
        \begin{pmatrix}
            A & C \\
            0 & B
        \end{pmatrix}^{-1}
        & =\begin{pmatrix}
                E_n & -A^{-1}C \\
                0 & E_m
            \end{pmatrix}\begin{pmatrix}
                A^{-1} & 0 \\
                0 & B^{-1}
            \end{pmatrix} \\
        & =\begin{pmatrix}
            A^{-1} & -A^{-1}CB^{-1} \\
            0 & B^{-1}
        \end{pmatrix}.\qedhere
    \end{align*}
\end{proof}
\end{document}