\documentclass{ctexart}
\usepackage[bgcolor]{lecturenote}

\title{第2章笔记和习题}

\begin{document}
\maketitle
\section{第2章笔记}
补充证明一个定理.
\begin{theorem}
    若 $A$ 非退化, 则 ${}^t(A^{-1})=({}^tA)^{-1}$.
\end{theorem}
\begin{proof}
    在 $AA^{-1}=E$ 两边转置得
    \[{}^tA{}^t(A^{-1})=E\Rightarrow{}^t(A^{-1})=({}^tA)^{-1}.\]
\end{proof}

下面讨论一下线性映射的核与像.

设矩阵 $A=(A^{(1)},A^{(2)},\cdots,A^{(n)})$ 对应的线性映射为 $\phi_A$.

$\because\forall X=[x_1,x_2,\cdots,x_n]\in\mathbb{R}^n,\phi_A(X)=x_1A^{(1)}+x_2A^{(2)}+\cdots+x_nA^{(n)}\in\left<A^{(1)},A^{(2)},\cdots,A^{(n)}\right>$,

$\therefore\operatorname{Im}\phi_A\subset\left<A^{(1)},A^{(2)},\cdots,A^{(n)}\right>$.

$\because\forall Y\in\left<A^{(1)},A^{(2)},\cdots,A^{(n)}\right>,\exists y_1,y_2,\cdots,y_n$ 使得 $Y=y_1A^{(1)}+y_2A^{(2)}+\cdots+y_nA^{(n)}$,

$\therefore\exists Y_0=(y_1,y_2,\cdots,y_n)$ 使得 $\phi_A(Y_0)=Y$.

$\therefore\operatorname{Im}\phi_A\supset\left<A^{(1)},A^{(2)},\cdots,A^{(n)}\right>$.

$\therefore\operatorname{Im}\phi_A=\left<A^{(1)},A^{(2)},\cdots,A^{(n)}\right>,\dim\operatorname{Im}\phi_A=\operatorname{rank}A$.

上述讨论对于实际的计算也有帮助. 比如求 $\left<X_1,\cdots,X_n\right>$ 的维数, 只需要求矩阵 $(X_1,\cdots,X_n)$ 的秩, 而后者可以通过行或列变换得到.

按照定义,
\begin{align*}
    \ker\phi_A & =\{X\in\mathbb{R}^n|\phi_A(X)=0\} \\
    & =\{X\in\mathbb{R}^n|AX=0\}.
\end{align*}

$\therefore\ker\phi_A$ 是方程 $AX=0$ 的解空间.

由第 3 节第 8 小节的定理 7 得
\[\dim\ker\phi_A=n-\operatorname{rank}A.\]

$\therefore$
\[\dim\ker\phi_A+\dim\operatorname{Im}\phi_A=n.\]
\section{第2章习题}
\stepcounter{exsection}
\begin{exercise}% 1.1
    设 $V_1,V_2,V$ 是 $\mathbb{R}^n$ 中的线性包, 证明: 如果对于每一个向量 $X\in V$, 写法 $X=X_1+X_2$ 都唯一, 其中 $X_i\in V_i$, 则 $V=V_1+V_2$ 是直和.
\end{exercise}
\begin{proof}
    假设 $\exists X_0\neq0,X_0\in V_1\cap V_2$.

    $\because X_0\in V_1,\therefore-X_0\in V_1$. $\because X_0\in V_2,\therefore-X_0\in V_2$.

    $\therefore0$ 有两种表示方式:
    \[0=0+0,\quad0=X_0+(-X_0).\]

    与 $0\in V\Rightarrow$ 写法 $0=X_1+X_2$ 唯一矛盾.
\end{proof}
\begin{exercise}% 1.2
    设 $V_1,V_2,V$ 是 $\mathbb{R}^n$ 中的线性包, 且 $V\subset V_1+V_2$. 等式 $V=V_1\cap V+V_2\cap V$ 一定成立吗? 证明在 $V_1\subset V$ 的情况下等式 $V=V_1\cap V+V_2\cap V$ 一定成立.
\end{exercise}
\begin{proof}
    $\because V_1\cap V\subset V,V_2\cap V\subset V$,

    $\therefore\forall X_1\in V_1\cap V,X_2\in V_2\cap V,X_1\in V,X_2\in V\Rightarrow X_1+X_2\in V$.

    $\therefore V\supset V_1\cap V+V_2\cap V$.

    但是 $V=V_1\cap V+V_2\cap V$ 不一定成立. 考察 $\mathbb{R}^2$ 中的 $V_1=\left<(0,1)\right>,V_2=\left<(1,0)\right>,V=\left<(1,1)\right>$, 有
    \[V_1\cap V=V_2\cap V=V_1\cap V+V_2\cap V=\{0\}\neq V.\]

    前面已经证明了等式 $V\supset V_1\cap V+V_2\cap V$ 一定成立, 下面只需证明在 $V_1\subset V$ 的情况下有 $V\subset V_1\cap V+V_2\cap V$.

    $\because V_1\subset V\Rightarrow V_1\cap V=V_1,\therefore V_1\cap V+V_2\cap V=V_1+V_2\cap V$.

    $\because V\subset V_1+V_2,\therefore\forall X\in V,\exists X_1\in V_1,X_2\in V_2$ 使得
    \begin{equation}\label{eq2.1}
        X=X_1+X_2.
    \end{equation}

    假设 $X_2\notin V_2\cap V\Rightarrow X_2\in V_2\backslash V$.

    $\because V_1\subset V,\therefore X_1\in V,\therefore-X_1\in V$.

    在 (\ref{eq2.1}) 式两边加上 $-X_1$, 有
    \[X+(-X_1)=X_1-X_1+X_2=X_2.\]

    $\because X\in V,-X_1\in V,\therefore X+(-X_1)\in V$, 与 $X_2\in V_2\backslash V$ 矛盾.

    $\therefore\forall X\in V,\exists X_1\in V_1=V_1\cap V,X_2\in V_2\cap V$, 使得 $X=X_1+X_2$.

    $\therefore V\subset V_1\cap V+V_2\cap V.$
\end{proof}
\addtocounter{exercise}{2}
\begin{exercise}% 1.5
    证明: $\mathbb{R}^n$ 中的向量组 $X_1,X_2,\cdots,X_n$ 张成 $\mathbb{R}^n$ 的充要条件是它们线性无关.
\end{exercise}
\begin{proof}
    (充分性) 设 $V=\left<X_1,X_2,\cdots,X_n\right>$. 假设 $V\neq\mathbb{R}^n$.

    $\because\forall X\in V,X=\alpha_1X_1+\alpha_2X_2+\cdots+\alpha_nX_n\subset\mathbb{R}^n$, $\therefore V\subset\mathbb{R}^n$.

    取 $X_{n+1}\in\mathbb{R}^n\backslash V$, 则 $X_1,X_2,\cdots,X_n,X_{n+1}$ 线性无关.

    $\therefore\dim\left<X_1,X_2,\cdots,X_n\right>=n+1>n$, 与书上的定理 2 矛盾.

    (必要性) 假设 $X_1,X_2,\cdots,X_n$ 线性相关.

    由书上的定理 1(iii), $X_1,X_2,\cdots,X_n$ 中至少有一个向量是其余向量的线性组合.
    
    不妨设 $X_n$ 是 $X_1,X_2,\cdots,X_{n-1}$ 的线性组合.

    若 $X_1,X_2,\cdots,X_{n-1}$ 线性相关, 则由书上的定理 1(iii), $X_1,X_2,\cdots,X_{n-1}$ 中至少有一个向量是其余向量的线性组合.

    不妨设 $X_{n-1}$ 是 $X_1,X_2,\cdots,X_{n-2}$ 的线性组合, 则 $X_n,X_{n-1}$ 是 $X_1,X_2,\cdots,X_{n-2}$ 的线性组合.

    重复上述步骤, 可以得到: $\exists r$ 满足 $1\leq r<n$, $X_1,X_2,\cdots,X_r$ 线性无关, $X_{r+1},X_{r+2},\cdots,X_n$ 是 $X_1,X_2,\cdots,X_r$ 的线性组合.

    $\therefore\mathbb{R}^n=\left<X_1,X_2,\cdots,X_r\right>$.

    $\therefore\dim\mathbb{R}^n=r<n$, 与 $\mathbb{R}^n=n$ 矛盾.
\end{proof}
证明第 1.7 题需要证明一个引理.
\begin{lemma}\label{l2.1}
    若 $V_1=<\alpha_1,\alpha_2,\cdots,\alpha_s>,V_2=<\beta_1,\beta_2,\cdots,\beta_t>$,
    
    则 $V_1+V_2=<\alpha_1,\alpha_2,\cdots,\alpha_s,\beta_1,\beta_2,\cdots,\beta_t>$.
\end{lemma}
\begin{proof}
    $\forall\alpha+\beta\in V_1+V_2$(不妨设 $\alpha\in V_1,\beta\in V_2$), 有
    \[\alpha=l_1\alpha_1+l_2\alpha_2+\cdots+l_s\alpha_s,\quad\beta=k_1\beta_1+k_2\beta_2+\cdots+k_t\beta_t,\]
    \[\alpha+\beta=l_1\alpha_1+l_2\alpha_2+\cdots+l_s\alpha_s+k_1\beta_1+k_2\beta_2+\cdots+k_t\beta_t.\]

    $\therefore$
    \[\alpha+\beta\in<\alpha_1,\alpha_2,\cdots,\alpha_s,\beta_1,\beta_2,\cdots,\beta_t>.\]
    
    $\therefore$
    \[(V_1+V_2)\subseteq<\alpha_1,\alpha_2,\cdots,\alpha_s,\beta_1,\beta_2,\cdots,\beta_t>.\]

    $\forall\gamma\in<\alpha_1,\alpha_2,\cdots,\alpha_s,\beta_1,\beta_2,\cdots,\beta_t>$, 有
    \[\gamma=l_1\alpha_1+l_2\alpha_2+\cdots+l_s\alpha_s+l_{s+1}\beta_1+l_{s+2}\beta_2+\cdots+l_{s+t}\beta_t.\]

    令 $\alpha=l_1\alpha_1+l_2\alpha_2+\cdots+l_s\alpha_s,\beta=l_{s+1}\beta_1+l_{s+2}\beta_2+\cdots+l_{s+t}\beta_t$, 有
    \begin{align*}
        \alpha\in V_1\land\beta\in V_2\land\gamma=\alpha+\beta & \Rightarrow\gamma\in V_1+V_2 \\
        & \Rightarrow(V_1+V_2)\supseteq<\alpha_1,\alpha_2,\cdots,\alpha_s,\beta_1,\beta_2,\cdots,\beta_t>.
    \end{align*}

    $\therefore(V_1+V_2)=<\alpha_1,\alpha_2,\cdots,\alpha_s,\beta_1,\beta_2,\cdots,\beta_t>.$
\end{proof}
\stepcounter{exercise}
\begin{exercise}[有改动]\label{ex1.7}
设 $V_1$ 和 $V_2$ 都是 $\mathbb{R}^n$ 中的线性包,证明:

$$\mathrm{dim}(V_1+V_2)=\mathrm{dim}V_1+\mathrm{dim}V_2-\mathrm{dim}(V_1\cap V_2).$$
\end{exercise}
\begin{proof}
    由书上定理 2 的证明过程得 $V_1\cap V_2$ 的一个基
    \[\alpha_1,\alpha_2,\cdots,\alpha_s\]
    可以扩充成 $V_1$ 的一个基
    \[\alpha_1,\alpha_2,\cdots,\alpha_s,\beta_1,\beta_2,\cdots,\beta_{m-s}\]
    和 $V_2$ 的一个基
    \[\alpha_1,\alpha_2,\cdots,\alpha_s,\gamma_1,\gamma_2,\cdots,\gamma_{n-s}.\]

    由引理 \ref{l2.1},
    \begin{align*}
        V_1+V_2 & =<\alpha_1,\alpha_2,\cdots,\alpha_s,\beta_1,\beta_2,\cdots,\beta_{m-s}>+<\alpha_1,\alpha_2,\cdots,\alpha_s,\gamma_1,\gamma_2,\cdots,\gamma_{n-s}> \\
        & =<\alpha_1,\alpha_2,\cdots,\alpha_s,\beta_1,\beta_2,\cdots,\beta_{m-s},\alpha_1,\alpha_2,\cdots,\alpha_s,\gamma_1,\gamma_2,\cdots,\gamma_{n-s}> \\
        & =<\alpha_1,\alpha_2,\cdots,\alpha_s,\beta_1,\beta_2,\cdots,\beta_{m-s},\gamma_1,\gamma_2,\cdots,\gamma_{n-s}>.
    \end{align*}

    设
    \begin{equation}\label{eq2.2}
        \sum\limits_{i=1}^sp_i\alpha_i+\sum\limits_{i=1}^{m-s}q_i\beta_i+\sum\limits_{i=1}^{n-s}r_i\gamma_i=0.
    \end{equation}

    则有
    \[\sum\limits_{i=1}^sp_i\alpha_i+\sum\limits_{i=1}^{m-s}q_i\beta_i=\sum\limits_{i=1}^{n-s}(-r_i)\gamma_i.\]

    设 $\zeta=(-r_1)\gamma_1+(-r_2)\gamma_2+\cdots+(-r_{n-s})\gamma_{n-s}\in V_2$. 由式 (\ref{eq2.2}) 得
    \[\zeta=\sum\limits_{i=1}^sp_i\alpha_i+\sum\limits_{i=1}^{m-s}q_i\beta_i\in V_1.\]

    $\therefore\zeta\in V_1\cap V_2$. 设 $\zeta=l_1\alpha_1+l_2\alpha_2+\cdots+l_s\alpha_s$. 则有
    \[\sum\limits_{i=1}^sl_i\alpha_i=\sum\limits_{i=1}^{n-s}(-r_i)\gamma_i\Rightarrow\sum\limits_{i=1}^sl_i\alpha_i+\sum\limits_{i=1}^{n-s}r_i\gamma_i=0.\]

    $\because\alpha_1,\alpha_2,\cdots,\alpha_s,\gamma_1,\gamma_2,\cdots,\gamma_{n-s}$ 是 $V_2$ 的一个基,

    $\therefore l_1=l_2=\cdots=l_s=r_1=r_2=\cdots=r_{n-s}=0$.

    把 $r_1=r_2=\cdots=r_{n-s}=0$ 代入式 (\ref{eq2.2}) 得
    \[\sum\limits_{i=1}^sp_i\alpha_i+\sum\limits_{i=1}^{m-s}q_i\beta_i=0.\]

    $\because\alpha_1,\alpha_2,\cdots,\alpha_s,\beta_1,\beta_2,\cdots,\beta_{m-s}$ 是 $V_1$ 的一个基,

    $\therefore p_1=p_2=\cdots=p_s=q_1=q_2=\cdots=q_{m-s}=0$.

    $\therefore$ 向量组 $\alpha_1,\alpha_2,\cdots,\alpha_s,\beta_1,\beta_2,\cdots,\beta_{m-s},\gamma_1,\gamma_2,\cdots,\gamma_{n-s}$ 线性无关.

    $\therefore\alpha_1,\alpha_2,\cdots,\alpha_s,\beta_1,\beta_2,\cdots,\beta_{m-s},\gamma_1,\gamma_2,\cdots,\gamma_{n-s}$ 是 $V_1+V_2$ 的一个基.

    $\therefore$
    \begin{align*}
        \mathrm{dim}(V_1+V_2) & =s+(m-s)+(n-s) \\
        & =m+n-s \\
        & =\mathrm{dim}V_1+\mathrm{dim}V_2-\mathrm{dim}(V_1\cap V_2).\qedhere
    \end{align*}
\end{proof}
\stepcounter{exsection}
\setcounter{exercise}{3}
\begin{exercise}% 2.4
    设两个矩阵
    \[A=\begin{pmatrix}
        \alpha_1 & \alpha_2 & \cdots & \alpha_n \\
        \beta_1 &  \beta_2 & \cdots &  \beta_n
    \end{pmatrix},\quad B=\begin{pmatrix}
        \alpha_1 & \alpha_2 & \cdots & \alpha_n \\
        \beta_1 &  \beta_2 & \cdots &  \beta_n \\
        \gamma_1 & \gamma_2 & \cdots & \gamma_n
    \end{pmatrix},\]

    用平面上 $n$ 条直线组成的集合的几何性质给出 $A$ 和 $B$ 有相等秩的条件.
\end{exercise}
\begin{solution}
    平面上 $n$ 条直线组成的集合为
    \[\{\alpha_ix+\beta_iy=\gamma_i|\alpha_i,\beta_i,\gamma_i\in\mathbb{R},i=1,2,\cdots,n\}.\]

    如果这 $n$ 条直线有公共点 $(p,q)$, 则
    \[p\alpha_i+q\beta_i=\gamma_i\quad(i=1,2,\cdots,n).\]

    $\therefore$
    \[pB_{(1)}+qB_{(2)}=B_{(3)}.\]

    $\therefore$
    \begin{align*}
        \left<B_{(1)},B_{(2)},B_{(3)}\right> & =\left<B_{(1)},B_{(2)}\right> \\
        & =\left<A_{(1)},A_{(2)}\right>.
    \end{align*}

    $\therefore\operatorname{rank}B=\operatorname{rank}A$.
\end{solution}
\stepcounter{exsection}
\setcounter{exercise}{3}
\begin{exercise}% 3.4
    由 Markov 矩阵
    \[P=(p_{ij}),\quad p_{ij}\geq0,\quad\sum\limits_{j=1}^np_{ij}=1,\quad i=1,2,\cdots,n\]
    
    确定的线性变换 $\phi_P$ 通常作用于概率列向量
    \[X=[x_1,\cdots,x_n],\quad x_i\geq0,\quad\sum\limits_{i=1}^nx_i=1.\]
    
    证明:
    \begin{itemize}
        \item[(a)] 矩阵 $P\in M_n(\mathbb{R})$ 是 Markov 的, 当且仅当对任一概率向量 $X$, 向量 $PX$ 仍然是概率向量.
        \item[(b)] 如果 $P$ 是正的 Markov 矩阵($\forall i,j,p_{ij}>0$), 那么对任一概率向量 $X,PX$ 是正的概率向量.
        \item[(c)] 如果矩阵 $P,Q$ 是 Markov 的, 那么矩阵 $A=PQ$ 也是 Markov 的.
    \end{itemize}
\end{exercise}
\begin{proof}
    (a) 记
    \[PX=[y_1,y_2,\cdots,y_n],\]

    由书上的式 (11) 得
    \[y_i=P_{(i)}X=\sum\limits_{j=1}^np_{ij}x_{j}\geq0,\]
    \begin{align*}
        \sum\limits_{i=1}^ny_i & =\sum\limits_{i=1}^n\sum\limits_{j=1}^np_{ij}x_{j} \\
        & =\sum\limits_{j=1}^nx_{j}\sum\limits_{i=1}^np_{ij} \\
        & =\sum\limits_{j=1}^nx_{j}=1.
    \end{align*}

    (b) $\because x_i\geq0,\sum\limits_{i=1}^nx_i=1,\therefore\exists k$ 满足 $1\leq k\leq n$ 使得 $x_k>0.\therefore$
    \[y_i=\sum\limits_{j=1}^np_{ij}x_{j}\geq p_{ik}x_k>0.\]

    (c) $\because Q$ 是 Markov 的, $\therefore Q^{(i)}\ (i=1,2,\cdots,n)$ 是概率向量.

    由 (a) 得 $PQ^{(i)}$ 是概率向量. $\therefore PQ$ 也是 Markov 的.
\end{proof}
\stepcounter{exercise}
\begin{exercise}% 3.6
    由 $S_n$ 中的 $n$ 阶循环确定的置换矩阵为
    \[P=\begin{pmatrix}
        0 & 0 & \cdots & 0 & 1 \\
        1 & 0 & \cdots & 0 & 0 \\
        0 & 1 & \cdots & 0 & 0 \\
        \vdots & \vdots & \ddots & \vdots & \vdots \\
        0 & 0 & \cdots & 1 & 0 \\
    \end{pmatrix}.\]

    验证 $P^n=E$.
\end{exercise}
\begin{solution}
    $\because PE^{(1)}=E^{(2)},PE^{(2)}=E^{(3)},\cdots,PE^{(n-1)}=E^{(n)},PE^{(n)}=E^{(1)},\therefore$
    \[P^{n}E^{(1)}=P^{n-1}E^{(2)}=\cdots=PE^{(n)}=E^{(1)},\]
    \[P^{n}E^{(2)}=P^{n-1}E^{(3)}=\cdots=PE^{(1)}=E^{(2)},\]
    \[\cdots,\]
    \[P^{n}E^{(n)}=P^{n-1}E^{(1)}=\cdots=PE^{(n-1)}=E^{(n)},\]

    $\therefore P^n=E$.
\end{solution}
\begin{exercise}% 3.7
    对于任意两个 $m\times n$ 矩阵 $A,B$, 证明
    \[\operatorname{rank}(A+B)\leq\operatorname{rank}A+\operatorname{rank}B.\]
\end{exercise}
\begin{proof}
    在等式
    \begin{equation}
        \sum\limits_{i=1}^n\lambda_iA^{(j)}=0
    \end{equation}

    两边乘一非零常数 $\mu$, 等式仍然成立. $\therefore$ 由式 $(3)$ 得
    \[\sum\limits_{i=1}^n\mu\lambda_iA^{(j)}=0\Rightarrow\sum\limits_{i=1}^n\lambda_i(\mu A^{(j)})=0.\]

    $\therefore$ 对向量组的每个向量乘一非零常数不改变向量组的线性相关性. $\therefore$
    \[\operatorname{rank}(-B)=\operatorname{rank}B.\]

    假设 $\operatorname{rank}(A+B)>\operatorname{rank}A+\operatorname{rank}B$, 则
    \begin{align*}
        \operatorname{rank}A & =\operatorname{rank}(A+B-B) \\
        & >\operatorname{rank}(A+B)+\operatorname{rank}(-B) \\
        & >\operatorname{rank}A+\operatorname{rank}B+\operatorname{rank}(-B) \\
        & =\operatorname{rank}A+2\operatorname{rank}B.
    \end{align*}

    $\therefore 0>2\operatorname{rank}B$, 与 $\operatorname{rank}B\geq0$ 矛盾.
\end{proof}
证明第 \ref{ex3.8} 题需要证明两个引理.
\begin{lemma}\label{l2.2}
    是 $A$ 是 $m\times s$ 矩阵, $B$ 是 $s\times n$ 矩阵, $A$ 确定的线性映射为 $\phi_A$, $B$ 的列空间为 $V_c(B)$, 则
    \[\dim\phi_A(V_c(B))=\operatorname{rank}AB.\]
\end{lemma}
\begin{proof}
    设 $AB$ 的列空间为 $V_c(AB)$, 有
    \[\dim V_c(AB)=\operatorname{rank}AB.\]

    $\therefore$ 只需证明 $V_c(AB)=\phi_A(V_c(B))$.

    $\because\phi_A(B^{(i)})=AB^{(i)}\ (i=1,2,\cdots,s),$

    $\therefore V_c(AB)=\left<AB^{(1)},AB^{(2)},\cdots,AB^{(s)}\right>=\phi_A(V_c(B))$.
\end{proof}
\begin{lemma}\label{l2.3}
    设 $\phi$ 是线性映射, $V_1,V_2$ 是线性包, 则
    \[\phi(V_1+V_2)=\phi(V_1)+\phi(V_2).\]
\end{lemma}
\begin{proof}
    $\forall Y'\in\phi(V_1+V_2),\exists Y\in V_1+V_2,\phi(Y)=Y'$.

    $\because Y\in V_1+V_2,\therefore\exists X_1\in V_1,X_2\in V_2$ 使得 $Y=X_1+X_2$ (注意这样的 $X_1,X_2$ 不一定是唯一的, 不过这不影响证明, 下面的 $\exists X_1',X_2'$ 也一样). $\therefore$
    \[Y'=\phi(Y)=\phi(X_1+X_2)=\phi(X_1)+\phi(X_2).\]

    $\therefore\phi(V_1+V_2)\subset\phi(V_1)+\phi(V_2)$.

    $\forall Y'\in\phi(V_1)+\phi(V_2),\exists X_1'\in\phi(V_1),X_2'\in\phi(V_2),Y'=X_1'+X_2'$.

    $\because\exists X_1\in V_1,X_2\in V_2,\phi(X_1)=X_1',\phi(X_2)=X_2'$.

    $\therefore\exists Y=X_1+X_2,Y\in V_1+V_2,$
    \[\phi(Y)=\phi(X_1+X_2)=\phi(X_1)+\phi(X_2)=X_1'+X_2'=Y.\]

    $\therefore\phi(V_1+V_2)\supset\phi(V_1)+\phi(V_2)$.

    $\therefore\phi(V_1+V_2)=\phi(V_1)+\phi(V_2)$.
\end{proof}
\begin{exercise}\label{ex3.8}
    对于任意 $m\times s$ 矩阵 $A$ 和 $s\times n$ 矩阵 $B$, 证明
    \[\operatorname{rank}A+\operatorname{rank}B-s\leq\operatorname{rank} AB.\]
\end{exercise}
下面是比较容易想到的证明方法(至少是我自己想到的方法).
\begin{thought}
    首先我们需要处理不等式中看似与矩阵的秩无关的 $s$ (事实上, 对 $s$ 的不同处理方法会导出不同的证明, 比如说将 $s$ 看成是 $M_s(\mathbb{R})$ 中的单位矩阵的秩就可以得到一般的教科书上的利用分块矩阵的证明). 这里我们利用等式 $s=\dim\mathbb{R}^s$ 将不等式变为
    \[\operatorname{rank}A+\operatorname{rank}B-\dim\mathbb{R}^s\leq\operatorname{rank} AB.\]

    然后我们把矩阵的秩都转化为线性包的维数. 注意到:
    \begin{enumerate}
        \item $V_c(B)\subset\dim\mathbb{R}^s$,
        \item 设 $A$ 确定的线性映射为 $\phi_A$, 则 $\operatorname{rank}A=\dim\phi_A(\mathbb{R}^s),V_c(AB)=\phi_A(V_c(B))$,
    \end{enumerate}

    将不等式变为
    \begin{align*}
        \dim\phi_A(\mathbb{R}^s)-\dim\phi_A(V_c(B)) & \leq\dim\mathbb{R}^s-\dim V_c(B) \\
        & =\dim\mathbb{R}^s\backslash V_c(B).
    \end{align*}

    如果能证明
    \[\dim\phi_A(\mathbb{R}^s)-\dim\phi_A(V_c(B))\leq\dim\phi_A(\mathbb{R}^s\backslash V_c(B)),\]

    那么只需证明
    \[\dim\phi_A(\mathbb{R}^s\backslash V_c(B))\leq\dim\mathbb{R}^s\backslash V_c(B).\]
\end{thought}
\begin{proof}
    设 $A$ 确定的线性映射为 $\phi_A$, $V_c(B)\subset\dim\mathbb{R}^s$ 为 $B$ 的列空间.

    令 $C$ 是满足 $V_c(C)+V_c(B)=\mathbb{R}^s$ 的矩阵(注意这样的 $C$ 不一定是唯一的, 不过这不影响证明). 由书上的定理 3 得
    \[\operatorname{rank}AC\leq\operatorname{rank}C.\]

    由引理 \ref{l2.2} 得
    \[\operatorname{rank}AC=\dim\phi_A(V_c(C)),\]

    $\therefore$
    \begin{equation}\label{eq2.4}
        \dim\phi_A(V_c(C))\leq\dim V_c(C).
    \end{equation}

    $\because V_c(C)+V_c(B)=\mathbb{R}^s$, 由引理 \ref{l2.3} 得
    \[\phi_A(V_c(C))+\phi_A(V_c(B))=\phi_A(V_c(C)+V_c(B))=\phi_A(\mathbb{R}^s).\]

    由习题 \ref{ex1.7} 得
    \begin{align*}
        \dim\phi_A(\mathbb{R}^s) & =\dim(\phi_A(V_c(C))+\phi_A(V_c(B))) \\
        & =\dim\phi_A(V_c(C))+\dim\phi_A(V_c(B))-\dim(\phi_A(V_c(C))\cap\phi_A(V_c(B))) \\
        & \leq\dim\phi_A(V_c(C))+\dim\phi_A(V_c(B)).
    \end{align*}

    由 (\ref{eq2.4}) 式得
    \begin{equation}\label{eq2.5}
        \dim\phi_A(\mathbb{R}^s)\leq\dim V_c(C)+\dim\phi_A(V_c(B)).
    \end{equation}

    由引理 \ref{l2.2} 得
    \[\dim\phi_A(\mathbb{R}^s)=\operatorname{rank}AE=\operatorname{rank}A,\]
    \[\dim\phi_A(V_c(B))=\operatorname{rank}AB,\]

    代入 (\ref{eq2.5}) 式得
    \begin{equation}
        \operatorname{rank}A\leq\dim V_c(C)+\operatorname{rank}AB.
    \end{equation}

    由习题 \ref{ex1.7} 得
    \begin{align*}
        & V_c(C)+V_c(B)=\mathbb{R}^s \\
        \Rightarrow & \dim (V_c(C)+V_c(B))=\dim\mathbb{R}^s \\
        \Rightarrow & \dim V_c(C)+\dim V_c(B)\leq\dim\mathbb{R}^s \\
        \Rightarrow & \dim V_c(C)\leq\dim\mathbb{R}^s-\dim V_c(B)=s-\operatorname{rank}B.
    \end{align*}

    代入 (\ref{eq2.5}) 式得
    \begin{align*}
        & \operatorname{rank}A\leq s-\operatorname{rank}B+\operatorname{rank}AB \\
        \Rightarrow & \operatorname{rank}A+\operatorname{rank}B-s\leq\operatorname{rank}AB.\qedhere
    \end{align*}
\end{proof}
\setcounter{exercise}{13}
\begin{exercise}% 3.14
    设
    \[A=\begin{pmatrix}
        a & b \\
        c & d
    \end{pmatrix}.\]
    
    利用关系 $A^2=(a+d)A-(ad-bc)E$ 求解 $A^{-1}$.
\end{exercise}
\begin{solution}
    \begin{align*}
        & A^2=(a+d)A-(ad-bc)E \\
        \Rightarrow & A^2=(a+d)AE-(ad-bc)E \\
        \Rightarrow & A^2+(-a-d)AE=-(ad-bc)E.
    \end{align*}

    由线性映射的性质(ii),
    \[(-a-d)AE=A(-a-d)E=A\operatorname{diag}(-a-d,-a-d).\]

    $\therefore$ 原式
    \[\Rightarrow A^2+A\operatorname{diag}(-a-d,-a-d)=-(ad-bc)E.\]

    由线性映射的性质(i), 原式
    \[\Rightarrow A(A-(a+d)E)=-(ad-bc)E.\]

    由线性映射的性质(ii), 原式
    \[\Rightarrow A\left(-\dfrac{1}{ad-bc}A+\dfrac{a+d}{ad-bc}E\right)=E.\]

    $\therefore$
    \begin{align*}
        A^{-1} & =\left(-\dfrac{1}{ad-bc}A+\dfrac{a+d}{ad-bc}E\right) \\
        & =\begin{pmatrix}
            -\dfrac{a}{ad-bc}+\dfrac{a+d}{ad-bc} & -\dfrac{b}{ad-bc} \\[8pt]
            -\dfrac{c}{ad-bc} & -\dfrac{d}{ad-bc}+\dfrac{a+d}{ad-bc}
        \end{pmatrix} \\
        & =\begin{pmatrix}
            \dfrac{d}{ad-bc} & -\dfrac{b}{ad-bc} \\[8pt]
            -\dfrac{c}{ad-bc} & \dfrac{a}{ad-bc}
        \end{pmatrix}.
    \end{align*}
\end{solution}
\stepcounter{exercise}
\begin{exercise}% 3.16
    证明: 若
    \[\begin{pmatrix}
        a & b \\
        c & d
    \end{pmatrix}^m=0,\]

    则
    \[\begin{pmatrix}
        a & b \\
        c & d
    \end{pmatrix}^2=0.\]
\end{exercise}
\begin{proof}
    设
    \[A=\begin{pmatrix}
        a & b \\
        c & d
    \end{pmatrix}.\]

    由第 \ref{ex3.8} 题和书上第 3 节的定理 3 得
    \[\operatorname{rank}A+\operatorname{rank}A-2\leq\operatorname{rank}A^2\leq\operatorname{rank}A,\]

    \[\operatorname{rank}A^2+\operatorname{rank}A-2\leq\operatorname{rank}A^3\leq\operatorname{rank}A^2,\]

    \[\cdots,\]

    \[\operatorname{rank}A^{n-1}+\operatorname{rank}A-2\leq\operatorname{rank}A^3\leq\operatorname{rank}A^{n-1},\]

    假设 $\operatorname{rank}A=2$, 则 $\operatorname{rank}A^m=2$, 与 $A^m=0\Rightarrow\operatorname{rank}A^m=0$ 矛盾. $\therefore\operatorname{rank}A\leq 1$.

    若 $A$ 有一列全为 $0$, 不妨设 $a=c=0$, 则
    \[A^2=\begin{pmatrix}
        0 & bd \\
        0 & d^2
    \end{pmatrix},\quad A^3=\begin{pmatrix}
        0 & bd^2 \\
        0 & d^3
    \end{pmatrix},\quad\cdots,\quad A^m=\begin{pmatrix}
        0 & bd^{m-1} \\
        0 & d^m
    \end{pmatrix}.\]

    若 $A^m=0$, 则 $d=0,A^2=0$.

    假设 $A$ 的两列都不全为 $0$. $\because\operatorname{rank}A\leq1,\therefore A^{(1)},A^{(2)}$ 线性相关.

    设 $A^{(1)}=[a,c],A^{(2)}=\lambda A^{(1)}\ (\lambda\neq0)$, 则
    \begin{align*}
        A^2 & =(AA^{(1)},AA^{(2)}) \\
        & =(AA^{(1)},\lambda AA^{(1)}) \\
        & =((A^{(1)},A^{(2)})[a,c],\lambda(A^{(1)},A^{(2)})[a,c]) \\
        & =(aA^{(1)}+cA^{(2)},\lambda(aA^{(1)}+cA^{(2)})) \\
        & =((a+\lambda c)A^{(1)},\lambda(a+\lambda c)A^{(1)}) \\
        & =(\mu A^{(1)},\lambda\mu A^{(1)}) \\
        & =\mu A\quad\mu \in\mathbb{R},
    \end{align*}
    \begin{align*}
        A^3 & =(A(A^2)^{(1)},A(A^2)^{(2)}) \\
        & =(\mu AA^{(1)},\lambda\mu AA^{(1)}) \\
        & =\mu(AA^{(1)},\lambda AA^{(1)}) \\
        & =\mu(\mu A^{(1)},\lambda\mu A^{(1)}) \\
        & =\mu^2 A,
    \end{align*}
    \[\cdots,\]
    \[A^m=\mu^{m-1} A.\]

    $\because\lambda\neq0,A^{(1)}\neq0,\therefore A^m=0\Rightarrow \mu^{m-1}=0\Rightarrow \mu=0\Rightarrow\cdots\Rightarrow A^2=0$.
\end{proof}
\begin{exercise}% 3.17
    设 $m\times s$ 矩阵
    \[X=\begin{pmatrix}
        X_{11} & X_{12} & \cdots & X_{1k} \\
        X_{21} & X_{22} & \cdots & X_{2k} \\
        \vdots & \vdots & \ddots & \vdots \\
        X_{l1} & X_{l2} & \cdots & X_{lk}
    \end{pmatrix}\]

    的块 $X_{ij}$ 是 $m_i\times s_j$ 矩阵 $(m_1+m_2+\cdots+m_l=m,s_1+s_2+\cdots+s_k=s)$, $s\times n$ 矩阵
    \[Y=\begin{pmatrix}
        Y_{11} & Y_{12} & \cdots & Y_{1r} \\
        Y_{21} & Y_{22} & \cdots & Y_{2r} \\
        \vdots & \vdots & \ddots & \vdots \\
        Y_{k1} & Y_{k2} & \cdots & Y_{kr} \\
    \end{pmatrix}\]

    的块 $Y_{ij}$ 是 $s_i\times n_j$ 矩阵 $(n_1+n_2+\cdots+n_r=n)$, 则 $Z=XY$ 可以分块计算, 它的块

    \[Z_{ij}=\sum\limits_{t=1}^kX_{it}Y_{tj}.\]
\end{exercise}
\begin{proof}
    将 $X,Y,Z$ 分成下列矩阵的和:
    \[X=\sum\limits_{i=1}^{l}\sum\limits_{j=1}^{k}{}^{(ij)}X,\]
    \[Y=\sum\limits_{i=1}^{k}\sum\limits_{j=1}^{r}{}^{(ij)}Y,\]
    \[Z=\sum\limits_{i=1}^{l}\sum\limits_{j=1}^{r}{}^{(ij)}Z,\]

    其中 ${}^{(ij)}X$ 的块 ${}^{(ij)}X_{ij}=X_{ij}$, 其余块都是零矩阵, ${}^{(ij)}X_{ut}=X_{ut}$, ${}^{(ij)}Y$ 的块 ${}^{(ij)}Y_{ij}=Y_{ij}$, 其余块都是零矩阵, ${}^{(ij)}Z$ 的块 ${}^{(ij)}Z_{ij}=Z_{ij}$, 其余块都是零矩阵.

    由矩阵乘法的分配律得
    \[Z=XY=\sum\limits_{a=1}^{l}\sum\limits_{b=1}^{k}\sum\limits_{c=1}^{k}\sum\limits_{d=1}^{r}{}^{(ab)}X{}^{(cd)}Y.\]

    考察 $b\neq c$ 的情况.

    定义 $s_{(0)}=0,s_{(i)}=s_1+s_2+\cdots+s_i$ (对 $m,n$ 也有类似的定义), 则 ${}^{(ab)}X{}^{(cd)}Y$ 的每一列都可以看成是 ${}^{(ab)}X$ 的第 $s_{(c-1)}+1,s_{(c-1)}+2,\cdots,s_{(c)}$ 列的线性组合.

    $\because b\neq c,\therefore{}^{(ab)}X$ 的第 $s_{(c-1)}+1,s_{(c-1)}+2,\cdots,s_{(c)}$ 列都是 $0$.

    $\therefore b\neq c\Rightarrow{}^{(ab)}X{}^{(cd)}Y=0$,
    \begin{align*}
        Z & =\sum\limits_{a=1}^{l}\sum\limits_{b=1}^{k}\sum\limits_{d=1}^{r}{}^{(ab)}X{}^{(bd)}Y \\
        & =\sum\limits_{a=1}^{l}\sum\limits_{d=1}^{r}\left(\sum\limits_{b=1}^{k}{}^{(ab)}X{}^{(bd)}Y\right),
    \end{align*}
    \[Z=\sum\limits_{a=1}^{l}\sum\limits_{d=1}^{r}{}^{(ad)}Z,\]

    $\therefore$
    \begin{equation}\label{eq2.7}
        \sum\limits_{a=1}^{l}\sum\limits_{d=1}^{r}\left({}^{(ad)}Z-\sum\limits_{b=1}^{k}{}^{(ab)}X{}^{(bd)}Y\right)=0.
    \end{equation}

    考察 $^{(ad)}Z'=\sum\limits_{b=1}^{k}{}^{(ab)}X{}^{(bd)}Y$ 的块 $^{(ad)}Z'_{ij}$. 设 $i\neq a$, 则 $\sum\limits_{b=1}^{k}{}^{(ab)}X{}^{(bd)}Y$ 中的每一个 $^{(ab)}X$ 的第 $m_{(i-1)}+1,m_{(i-1)}+2,\cdots,m_{(i)}$ 行都为 $0$,

    $\therefore{}^{(ad)}Z'$ 的第 $m_{(i-1)}+1,m_{(i-1)}+2,\cdots,m_{(i)}$ 行都为 $0$, $\therefore Z'_{ij}=0$.

    同理, 若 $j\neq d$, 则 $Z'_{ij}=0$.

    $\therefore{}^{(ad)}Z'$ 的块 $^{(ad)}Z'_{ij}$ 中除 $^{(ad)}Z'_{ad}$ 外都是零矩阵.

    $\therefore{}^{(ad)}Z-{}^{(ad)}Z'$ 的块 $^{(ad)}Z_{ij}-^{(ad)}Z'_{ij}$ 中除 $^{(ad)}Z_{ad}-^{(ad)}Z'_{ad}$ 外都是零矩阵.

    假设 $\exists$ 非空集合 $U$ 使得 $\forall (a,d)\in U,{}^{(ad)}Z'\neq{}^{(ad)}Z,\forall (a,d)\in[1,l]\times[1,r]\backslash U,{}^{(ad)}Z'={}^{(ad)}Z$.

    $\because\forall (a,d)\in U,$
    \[{}^{(ad)}Z'-{}^{(ad)}Z=\sum\limits_{i=s(a-1)}^{s(a)}\sum\limits_{j=s(d-1)}^{s(d)}\lambda_{ij}E_{ij},\]

    其中 $E_{ij}$ 为矩阵单位, 定义见书上的定理 4, $\therefore$
    \begin{align*}
        {}^{(ad)}Z'-{}^{(ad)}Z\in & \left<E_{s(a-1)s(d-1)},E_{s(a-1),s(d-1)+1},\cdots,E_{s(a-1)s(d)}\right. \\
        & E_{s(a-1)+1,s(d-1)},E_{s(a-1)+1,s(d-1)+1},\cdots,E_{s(a-1)+1,s(d)}, \\
        & \cdots, \\
        & \left.E_{s(a)s(d-1)},E_{s(a),s(d-1)+1},\cdots,E_{s(a),s(d)}\right>
    \end{align*}
    
    $\therefore$ 集合 $\{{}^{(ad)}Z-{}^{(ad)}Z'|(a,d)\in U\}$ 线性无关.

    由 (\ref{eq2.7}) 式,
    \[\sum\limits_{a=1}^{l}\sum\limits_{d=1}^{r}({}^{(ad)}Z'-{}^{(ad)}Z)=\sum\limits_{(a,d)\in U}({}^{(ad)}Z'-{}^{(ad)}Z)=0,\]

    即关于 $x$ 的方程
    \[\sum\limits_{(a,d)\in U}x_{ad}({}^{(ad)}Z'-{}^{(ad)}Z)=0\]

    有非零解 $\{x_{ad}=1|(a,d)\in U\}$, 与集合 $\{{}^{(ad)}Z-{}^{(ad)}Z'|(a,d)\in U\}$ 线性无关矛盾.

    $\therefore{}^{(ad)}Z'={}^{(ad)}Z,$
    \[^{(ij)}Z_{ij}=\sum\limits_{t=1}^k{}^{(it)}X_{it}{}^{(tj)}Y_{tj}.\]

    $\therefore$
    \[Z_{ij}=\sum\limits_{t=1}^kX_{it}Y_{tj}.\qedhere\]
\end{proof}
\begin{note}
    把 $^{(ad)}Z$ 之类的矩阵换成矩阵单位, 可以用类似的方法推导矩阵的乘法公式.
\end{note}
\begin{exercisec}[第 4 章的习题 3.9]
    设 $A\in M_n(\mathbb{R}),B\in M_m(\mathbb{R})$ 可逆, $C$ 是任意的 $n\times m$ 矩阵, 证明
    \[\begin{pmatrix}
        A & C \\
        0 & B
    \end{pmatrix}\]

    可逆, 并求其逆.
\end{exercisec}
\begin{proof}
    $\because$
    \[\begin{pmatrix}
        A^{-1} & 0 \\
        0 & B^{-1}
    \end{pmatrix}\begin{pmatrix}
        A & C \\
        0 & B
    \end{pmatrix}=\begin{pmatrix}
        E_n & A^{-1}C \\
        0 & E_m
    \end{pmatrix},\]

    而矩阵
    \[\begin{pmatrix}
        E_n & A^{-1}C \\
        0 & E_m
    \end{pmatrix}\]

    经过初等行变换可以得到 $E_{m+n}$, $\therefore$ 原矩阵可逆.

    考察矩阵
    \[\begin{pmatrix}
        E_n & A^{-1}C \\
        0 & E_m
    \end{pmatrix}.\]

    将其第 $2$ 行的 $-A^{-1}C$ 倍加到第 $1$ 行得到 $E_{m+n}$. 可以把初等变换写成左乘分块初等矩阵的形式:
    \[\begin{pmatrix}
        E_n & -A^{-1}C \\
        0 & E_m
    \end{pmatrix}\begin{pmatrix}
        E_n & A^{-1}C \\
        0 & E_m
    \end{pmatrix}=E_{m+n}.\]

    $\therefore$
    \[\begin{pmatrix}
        E_n & -A^{-1}C \\
        0 & E_m
    \end{pmatrix}\begin{pmatrix}
        A^{-1} & 0 \\
        0 & B^{-1}
    \end{pmatrix}\begin{pmatrix}
        A & C \\
        0 & B
    \end{pmatrix}=E_{m+n},\]
    \begin{align*}
        \begin{pmatrix}
            A & C \\
            0 & B
        \end{pmatrix}^{-1}
        & =\begin{pmatrix}
                E_n & -A^{-1}C \\
                0 & E_m
            \end{pmatrix}\begin{pmatrix}
                A^{-1} & 0 \\
                0 & B^{-1}
            \end{pmatrix} \\
        & =\begin{pmatrix}
            A^{-1} & -A^{-1}CB^{-1} \\
            0 & B^{-1}
        \end{pmatrix}.\qedhere
    \end{align*}
\end{proof}
\end{document}