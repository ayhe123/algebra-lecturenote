\documentclass{ctexart}
\usepackage[bgcolor]{lecturenote}

\title{第2章笔记和习题}

\begin{document}
\maketitle
\section{第2章笔记}
这里对线性子空间的定义与《基础代数》的一致, 称由集合 $S\subset V$ 张成的 $V$ 的子空间为 $S$ 的线性包, 记作 $\left<S\right>$.

不加说明的话, 认为 $E_n$ 是 $n$ 阶单位矩阵.

补充证明一个定理.
\begin{theorem}
    若 $A$ 非退化, 则 ${}^t(A^{-1})=({}^tA)^{-1}$.
\end{theorem}
\begin{proof}
    在 $AA^{-1}=E$ 两边转置得
    \[{}^tA{}^t(A^{-1})=E\Rightarrow{}^t(A^{-1})=({}^tA)^{-1}.\qedhere\]
\end{proof}

从分块矩阵的乘法 (习题 \ref{ex3.17}) 可以得到\textbf{分块初等矩阵}的概念. 设 $n\times n$ 矩阵
\[X=\begin{pmatrix}
    X_{11} & X_{12} & \cdots & X_{1k} \\
    X_{21} & X_{22} & \cdots & X_{2k} \\
    \vdots & \vdots & \ddots & \vdots \\
    X_{k1} & X_{k2} & \cdots & X_{kk}
\end{pmatrix}\]
的块 $X_{ij}$ 是 $n_i\times n_j$ 矩阵 $(n_1+n_2+\cdots+n_k=n)$.
\begin{definition}
    沿用习题 \ref{ex3.17} 的记号 $n_{(i)}$, 设 ${}^{(ab)}A$ 是第 $n_{(a-1)}+1$ 到第 $n_{(a)}$ 行, 第 $n_{(b-1)}+1$ 到第 $n_{(b)}$ 列为一个 $n_a\times n_b$ 的矩阵 $A$ 的 $n\times n$ 矩阵. 称形如 $F_{(a,b)}=E_n-{}^{(aa)}E_{n_a}-{}^{(bb)}E_{n_a}+{}^{(ab)}E_{n_a}+{}^{(ba)}E_{n_a}$ (这里要求 $n_a=n_b$) 的矩阵为第 I 型分块初等矩阵, 形如 $F_{(a,b)}(A)=E_n+{}^{(ab)}A$ (这里对 $n_a,n_b$ 没有要求)的矩阵为第 II 型分块初等矩阵, 形如 $F_{(a)}(A)=E_n+{}^{(aa)}(A-E_{n_a})$ 的矩阵为第 III 型分块初等矩阵.
\end{definition}

容易验证:
\begin{theorem}
    \begin{enumerate}
        \def\labelenumi{(\arabic{enumi})}
        \item 分块初等矩阵能被化为 $E_n$, $\therefore$ 分块初等矩阵是非退化的;
        \item $F_{(a,b)}$ 是若干 $F_{s,t}$ 的乘积, $F_{(a,b)}(A)$ 是若干 $F_{s,t}(\lambda)$ 的乘积.
        \item 矩阵 $X$ 左乘 $F_{(a,b)}$ 相当于将 $X$ 的第 $n_{(a-1)}+i\ (i=1,2,\cdots,n_{(a)})$ 行与第 $n_{(b-1)}+i$ 行互换, $X$ 右乘 $F_{(a,b)}$ 相当于将 $X$ 的第 $n_{(a-1)}+i\ (i=1,2,\cdots,n_{(a)})$ 列与第 $n_{(b-1)}+i$ 列互换;
        \item $X$ 左乘 $F_{(a,b)}(A)$ 相当于将 $A(X_{b1},X_{b2},\cdots,X_{bk})$ 加到 $X$ 的第 $n_{(a-1)}+1,n_{(a-1)}+2,\cdots,n_{(a)}$ 行, $X$ 右乘 $F_{(a,b)}(A)$ 相当于将 $[X_{1a},X_{2a},\cdots,X_{ka}]A$ 加到 $X$ 的第 $n_{(b-1)}+1,n_{(b-1)}+2,\cdots,n_{(b)}$ 列;
        \item $X$ 左乘 $F_{(a)}(A)$ 相当于将 $(X_{a1},\cdots,X_{ak})$ 左乘 $A$, $X$ 右乘 $F_{(a)}(A)$ 相当于将 $[X_{1a},\cdots,X_{ka}]$ 右乘 $A$;
        \item 任意可逆矩阵 $X$ 都是分块初等矩阵的乘积.
    \end{enumerate}
\end{theorem}

下面讨论一下线性映射的核与像.

设矩阵 $A=(A^{(1)},A^{(2)},\cdots,A^{(n)})$ 对应的线性映射为 $\phi_A$.

$\because\forall X=[x_1,x_2,\cdots,x_n]\in\mathbb{R}^n,\phi_A(X)=x_1A^{(1)}+x_2A^{(2)}+\cdots+x_nA^{(n)}\in\left<A^{(1)},A^{(2)},\cdots,A^{(n)}\right>$,

$\therefore\im \phi_A\subset\left<A^{(1)},A^{(2)},\cdots,A^{(n)}\right>$.

$\because\forall Y\in\left<A^{(1)},A^{(2)},\cdots,A^{(n)}\right>,\exists y_1,y_2,\cdots,y_n$ 使得 $Y=y_1A^{(1)}+y_2A^{(2)}+\cdots+y_nA^{(n)}$,

$\therefore\exists Y_0=(y_1,y_2,\cdots,y_n)$ 使得 $\phi_A(Y_0)=Y$.

$\therefore\im \phi_A\supset\left<A^{(1)},A^{(2)},\cdots,A^{(n)}\right>$.

$\therefore\im \phi_A=\left<A^{(1)},A^{(2)},\cdots,A^{(n)}\right>,\dim\im \phi_A=\rank A$.

上述讨论对于实际的计算也有帮助. 比如求 $\left<X_1,\cdots,X_n\right>$ 的维数, 只需要求矩阵 $(X_1,\cdots,X_n)$ 的秩, 而后者可以通过行或列变换得到.

按照定义,
\begin{align*}
    \ker\phi_A & =\{X\in\mathbb{R}^n|\phi_A(X)=0\} \\
    & =\{X\in\mathbb{R}^n|AX=0\}.
\end{align*}

$\therefore\ker\phi_A$ 是方程 $AX=0$ 的解空间.

由书上第 3 节的定理 7 得
\[\dim\ker\phi_A=n-\rank A.\]

$\therefore$
\[\dim\ker\phi_A+\dim\im \phi_A=n.\]
\section{第2章习题}
\subsection{习题2.1}
\stepcounter{exsection}
\begin{exercise}[有改动]\label{ex1.1}
    设 $V_1,V_2$ 是 $\mathbb{R}^n$ 的线性子空间, 证明:
    \begin{enumerate}
        \def\labelenumi{(\arabic{enumi})}
        \item $\left<V_1\cup V_2\right>=\{v_1+v_2|v_i\in V_i\}$, 将 $\left<V_1\cup V_2\right>$ 称为 $V_1$ 和 $V_2$ 的\textbf{和}, 记作 $V_1+V_2$.
        \item 设 $V=V_1+V_2$, 则 $\forall X\in V$, 写法 $X=X_1+X_2\ (X_i\in V_i)$ 都唯一 $\Leftrightarrow V_1\cap V_2=\{0\}$. 这时称 $V$ 是 $V_1$ 与 $V_2$ 的\textbf{直和}, 记作 $V=V_1\oplus V_2$.
    \end{enumerate}
\end{exercise}
\begin{proof}
    (1) $\forall X\in\left<V_1\cup V_2\right>$, $\exists Y_1,Y_2,\cdots,Y_n\in V_1,Z_1,Z_2,\cdots,Z_m\in V_2$ 使得
    \[X=\alpha_1Y_1+\alpha_2Y_2+\cdots+\alpha_nY_n+\beta_1Z_1+\beta_2Z_2+\cdots+\beta_mZ_m.\]

    令 $Y=\alpha_1Y_1+\alpha_2Y_2+\cdots+\alpha_nY_n\in V_1,Z=\beta_1Z_1+\beta_2Z_2+\cdots+\beta_mZ_m\in V_2$, 则 $X=Y+Z$. $\therefore X\in\{v_1+v_2|v_i\in V_i\}$.

    $\forall X\in\{v_1+v_2|v_i\in V_i\},\exists Y\in V_1,Z\in V_2$ 使得 $X=Y+Z$.

    $\because Y\in V_1,Z\in V_1$, $\therefore Y,Z\in V_1\cup V_2$. $\therefore X\in\left<V_1\cup V_2\right>$.
    
    (2) ($\Leftarrow$) 设 $X=X_1+X_2=X_1'+X_2'\ (X_i,X_i'\in V_i)$, 则 $X_1-X_1'=X_2'-X_2$.

    $\because X_1,X_1'\in V_1$, $\therefore X_1-X_1'\in V_1$. 同理 $X_2'-X_2\in V_2$. $\because X_1-X_1'=X_2'-X_2$, $\therefore X_1-X_1'\in V_2$. $\therefore X_1-X_1'\in V_1\cap V_2$.

    $\because V_1\cap V_2=\{0\}$, $\therefore X_1=X_1'$. 同理 $X_2=X_2'$.

    ($\Rightarrow$) 假设 $\exists X_0\neq0,X_0\in V_1\cap V_2$.

    $\because X_0\in V_1,\therefore-X_0\in V_1$. $\because X_0\in V_2,\therefore-X_0\in V_2$.

    $\therefore0$ 有两种表示方式:
    \[0=0+0,\quad0=X_0+(-X_0).\]

    与 $0\in V\Rightarrow$ 写法 $0=X_1+X_2$ 唯一矛盾.
\end{proof}
\begin{exercise}% 1.2
    设 $V_1,V_2,V$ 是 $\mathbb{R}^n$ 的线性子空间, 且 $V\subset V_1+V_2$. 等式 $V=V_1\cap V+V_2\cap V$ 一定成立吗? 证明在 $V_1\subset V$ 的情况下等式 $V=V_1\cap V+V_2\cap V$ 一定成立.
\end{exercise}
\begin{proof}
    $\because V_1\cap V\subset V,V_2\cap V\subset V$,

    $\therefore\forall X_1\in V_1\cap V,X_2\in V_2\cap V,X_1\in V,X_2\in V\Rightarrow X_1+X_2\in V$.

    $\therefore V\supset V_1\cap V+V_2\cap V$.

    但是 $V=V_1\cap V+V_2\cap V$ 不一定成立. 考察 $\mathbb{R}^2$ 中的 $V_1=\left<(0,1)\right>,V_2=\left<(1,0)\right>,V=\left<(1,1)\right>$, 有
    \[V_1\cap V=V_2\cap V=V_1\cap V+V_2\cap V=\{0\}\neq V.\]

    前面已经证明了等式 $V\supset V_1\cap V+V_2\cap V$ 一定成立, 下面只需证明在 $V_1\subset V$ 的情况下有 $V\subset V_1\cap V+V_2\cap V$.

    $\because V_1\subset V\Rightarrow V_1\cap V=V_1,\therefore V_1\cap V+V_2\cap V=V_1+V_2\cap V$.

    $\because V\subset V_1+V_2,\therefore\forall X\in V,\exists X_1\in V_1,X_2\in V_2$ 使得 $X=X_1+X_2$.

    $\because X_1\in V_1\subset V,X\in V$, $\therefore X_2=X-X_1\in V$. $\therefore X_2\in V\cap V_2$. $\therefore X\in V_1+V\cap V_2$.
\end{proof}
\addtocounter{exercise}{2}
\begin{exercise}% 1.5
    证明: $\mathbb{R}^n$ 中的向量组 $X_1,X_2,\cdots,X_n$ 张成 $\mathbb{R}^n$ 的充要条件是它们线性无关.
\end{exercise}
\begin{proof}
    (充分性) 设 $V=\left<X_1,X_2,\cdots,X_n\right>$. 假设 $V\neq\mathbb{R}^n$.

    $\because\forall X\in V,X=\alpha_1X_1+\alpha_2X_2+\cdots+\alpha_nX_n\subset\mathbb{R}^n$, $\therefore V\subset\mathbb{R}^n$.

    取 $X_{n+1}\in\mathbb{R}^n\backslash V$, 则 $X_1,X_2,\cdots,X_n,X_{n+1}$ 线性无关.

    $\therefore\dim\left<X_1,X_2,\cdots,X_{n+1}\right>=n+1>n$, 与书上的定理 2 矛盾.

    (必要性) 假设 $X_1,X_2,\cdots,X_n$ 线性相关.

    由书上的定理 1(iii), $X_1,X_2,\cdots,X_n$ 中至少有一个向量是其余向量的线性组合.
    
    不妨设 $X_n$ 是 $X_1,X_2,\cdots,X_{n-1}$ 的线性组合.

    若 $X_1,X_2,\cdots,X_{n-1}$ 线性相关, 则由书上的定理 1(iii), $X_1,X_2,\cdots,X_{n-1}$ 中至少有一个向量是其余向量的线性组合.

    不妨设 $X_{n-1}$ 是 $X_1,X_2,\cdots,X_{n-2}$ 的线性组合, 则 $X_n,X_{n-1}$ 是 $X_1,X_2,\cdots,X_{n-2}$ 的线性组合.

    重复上述步骤, 可以得到: $\exists r$ 满足 $1\leq r<n$, $X_1,X_2,\cdots,X_r$ 线性无关, $X_{r+1},X_{r+2},\cdots,X_n$ 是 $X_1,X_2,\cdots,X_r$ 的线性组合.

    $\therefore\mathbb{R}^n=\left<X_1,X_2,\cdots,X_n\right>=\left<X_1,X_2,\cdots,X_r\right>$.

    $\therefore\dim\mathbb{R}^n=r<n$, 与 $\mathbb{R}^n=n$ 矛盾.
\end{proof}
证明第 \ref{ex1.7} 题需要证明两个引理.
\begin{lemma}\label{l2.1}
    设集合 $S\subset\mathbb{R}^n$, 线性子空间 $V\subset\mathbb{R}^n$. 如果 $S\subset V$, 那么 $\left<S\right>\subset V$.
\end{lemma}
\begin{proof}
    $\because S\subset V$, $\therefore\forall X\in S,X\in V$.

    $\forall X\in\left<S\right>$, $\exists$ 有限个向量 $X_1,X_2,\cdots,X_m\in S\subset V$ 使得 $X=a_1X_1+a_2X_2+\cdots+a_mX_m$.

    $\because X_1,X_2,\cdots,X_m\in V$, $\therefore X=a_1X_1+a_2X_2+\cdots+a_mX_m\in V$.
\end{proof}
\begin{lemma}\label{l2.2}
    若 $V_1=\left<X_1,X_2,\cdots,X_s\right>,V_2=\left<Y_1,Y_2,\cdots,Y_t\right>$,
    
    则 $V_1+V_2=\left<X_1,X_2,\cdots,X_s,Y_1,Y_2,\cdots,Y_t\right>$.
\end{lemma}
\begin{proof}
    由第 \ref{ex1.1} 题得 $V_1+V_2=\left<V_1\cup V_2\right>$.

    $\because X_1,X_2,\cdots,X_s\in V_1,Y_1,Y_2,\cdots,Y_t\in V_2$, $\therefore\{X_1,X_2,\cdots,X_s,Y_1,Y_2,\cdots,Y_t\}\subset V_1\cup V_2$, $\therefore\{X_1,X_2,\cdots,X_s,Y_1,Y_2,\cdots,Y_t\}\subset\left<V_1\cup V_2\right>$.
    
    由引理 \ref{l2.1} 得 $\left<X_1,X_2,\cdots,X_s,Y_1,Y_2,\cdots,Y_t\right>\subset\left<V_1\cup V_2\right>$.

    $\forall X\in V_1\cup V_2$, $X\in V_1=\left<X_1,\cdots,X_s\right>$ 或 $X\in V_2=\left<Y_1,\cdots,Y_t\right>$. $\therefore X\in\left<X_1,\cdots,X_s,Y_2,\cdots,Y_t\right>$. $\therefore V_1\cup V_2\subset\left<X_1,X_2,\cdots,X_s,Y_1,Y_2,\cdots,Y_t\right>$.
    
    由引理 \ref{l2.1} 得 $\left<V_1\cup V_2\right>\subset\left<X_1,X_2,\cdots,X_s,Y_1,Y_2,\cdots,Y_t\right>$.
\end{proof}
\stepcounter{exercise}
\begin{exercise}[有改动]\label{ex1.7}
设 $V_1$ 和 $V_2$ 都是 $\mathbb{R}^n$ 的线性子空间,证明:
$$\mathrm{dim}(V_1+V_2)=\mathrm{dim}V_1+\mathrm{dim}V_2-\mathrm{dim}(V_1\cap V_2).$$
\end{exercise}
\begin{proof}
    由书上定理 2 的证明过程得 $V_1\cap V_2$ 的一个基
    \[X_1,X_2,\cdots,X_s\]
    可以扩充成 $V_1$ 的一个基
    \[X_1,X_2,\cdots,X_s,Y_1,Y_2,\cdots,Y_{m-s}\]
    和 $V_2$ 的一个基
    \[X_1,X_2,\cdots,X_s,Z_1,Z_2,\cdots,Z_{n-s}.\]

    由引理 \ref{l2.2},
    \begin{align*}
        V_1+V_2 & =\left<X_1,X_2,\cdots,X_s,Y_1,Y_2,\cdots,Y_{m-s}\right>+\left<X_1,X_2,\cdots,X_s,Z_1,Z_2,\cdots,Z_{n-s}\right> \\
        & =\left<X_1,X_2,\cdots,X_s,Y_1,Y_2,\cdots,Y_{m-s},X_1,X_2,\cdots,X_s,Z_1,Z_2,\cdots,Z_{n-s}\right> \\
        & =\left<X_1,X_2,\cdots,X_s,Y_1,Y_2,\cdots,Y_{m-s},Z_1,Z_2,\cdots,Z_{n-s}\right>.
    \end{align*}

    设
    \begin{equation}\label{eq2.1}
        \sum\limits_{i=1}^sp_iX_i+\sum\limits_{i=1}^{m-s}q_iY_i+\sum\limits_{i=1}^{n-s}r_iZ_i=0.
    \end{equation}

    则有
    \[\sum\limits_{i=1}^sp_iX_i+\sum\limits_{i=1}^{m-s}q_iY_i=\sum\limits_{i=1}^{n-s}(-r_i)Z_i.\]

    设 $W=(-r_1)Z_1+(-r_2)Z_2+\cdots+(-r_{n-s})Z_{n-s}\in V_2$. 由式 (\ref{eq2.1}) 得
    \[W=\sum\limits_{i=1}^sp_iX_i+\sum\limits_{i=1}^{m-s}q_iY_i\in V_1.\]

    $\therefore W\in V_1\cap V_2$. 设 $W=l_1X_1+l_2X_2+\cdots+l_sX_s$. 则有
    \[\sum\limits_{i=1}^sl_iX_i=\sum\limits_{i=1}^{n-s}(-r_i)Z_i\Rightarrow\sum\limits_{i=1}^sl_iX_i+\sum\limits_{i=1}^{n-s}r_iZ_i=0.\]

    $\because X_1,X_2,\cdots,X_s,Z_1,Z_2,\cdots,Z_{n-s}$ 是 $V_2$ 的一个基,

    $\therefore l_1=l_2=\cdots=l_s=r_1=r_2=\cdots=r_{n-s}=0$.

    把 $r_1=r_2=\cdots=r_{n-s}=0$ 代入式 (\ref{eq2.1}) 得
    \[\sum\limits_{i=1}^sp_iX_i+\sum\limits_{i=1}^{m-s}q_iY_i=0.\]

    $\because X_1,X_2,\cdots,X_s,Y_1,Y_2,\cdots,Y_{m-s}$ 是 $V_1$ 的一个基,

    $\therefore p_1=p_2=\cdots=p_s=q_1=q_2=\cdots=q_{m-s}=0$.

    $\therefore$ 向量组 $X_1,X_2,\cdots,X_s,Y_1,Y_2,\cdots,Y_{m-s},Z_1,Z_2,\cdots,Z_{n-s}$ 线性无关.

    $\therefore X_1,X_2,\cdots,X_s,Y_1,Y_2,\cdots,Y_{m-s},Z_1,Z_2,\cdots,Y_{n-s}$ 是 $V_1+V_2$ 的一个基.

    $\therefore$
    \begin{align*}
        \mathrm{dim}(V_1+V_2) & =s+(m-s)+(n-s) \\
        & =m+n-s \\
        & =\mathrm{dim}V_1+\mathrm{dim}V_2-\mathrm{dim}(V_1\cap V_2).\qedhere
    \end{align*}
\end{proof}
\subsection{习题2.2}
\stepcounter{exsection}
\begin{exercise}% 2.1
    不把 $m\times n$ 矩阵 $A=(a_{ij})$ 化成阶梯型, 证明 $r_c(A)=r_r(A)$.
\end{exercise}
\begin{proof}
    将证明分成两个部分.

    (1) 设 $r_r(A)=r,r_c(A)=s$.
    
    取 $A$ 的 $r$ 个线性无关的行向量 $A_{(i_1)},A_{(i_2)},\cdots,A_{(i_r)}$ 组成矩阵 $\widetilde{A}=[A_{(i_1)},A_{(i_2)},\cdots,A_{(i_r)}]$. $\because\widetilde{A}$ 的列空间是 $\mathbb{R}^r$ 的子空间, $\therefore\widetilde{A}$ 的列秩 $t\leq r$.

    设 $\widetilde{A}$ 的列 $\widetilde{A}^{(j_1)},\widetilde{A}^{(j_2)},\cdots,\widetilde{A}^{(j_u)}$ 线性相关, 即方程组
    \[\begin{cases}
        a_{i_1j_1}x_1+a_{i_1j_2}x_2+\cdots+a_{i_1j_u}x_u=0 \\
        a_{i_2j_1}x_1+a_{i_2j_2}x_2+\cdots+a_{i_2j_u}x_u=0 \\
        \cdots \\
        a_{i_rj_1}x_1+a_{i_rj_2}x_2+\cdots+a_{i_rj_u}x_u=0 \\
    \end{cases}\]
    有非零解. 设 $\tilde{x}_1,\tilde{x}_2,\cdots,\tilde{x}_u$ 是上述方程组的非零解.

    $\because r_r(A)=r$, $\therefore\left<A_{(i_1)},A_{(i_2)},\cdots,A_{(i_r)}\right>=\left<A_{(1)},A_{(2)},\cdots,A_{(m)}\right>$. 设 $A_{(v)}=b_1A_{(i_1)}+b_2A_{(i_2)}+\cdots+b_rA_{(i_r)}\ (v=1,2,\cdots,m)$, 则 $a_{vj_k}=b_1a_{i_1j_k}+b_2a_{i_2j_k}+\cdots+b_ra_{i_rj_k}\ (k=1,2,\cdots,u)$,
    \[\sum\limits_{k=1}^ua_{vj_k}\tilde{x}_k=\sum\limits_{k=1}^u\sum\limits_{l=1}^rb_la_{i_lj_k}\tilde{x}_k=\sum\limits_{l=1}^rb_l\sum\limits_{k=1}^ua_{i_lj_k}\tilde{x}_k=\sum\limits_{l=1}^rb_l\cdot0=0.\]

    上式对任意的 $v=1,2,\cdots,m$ 都成立, $\therefore\tilde{x}_1,\tilde{x}_2,\cdots,\tilde{x}_u$ 是方程组
    \[\begin{cases}
        a_{1j_1}x_1+a_{1j_2}x_2+\cdots+a_{1j_u}x_u=0 \\
        a_{2j_1}x_1+a_{2j_2}x_2+\cdots+a_{2j_u}x_u=0 \\
        \cdots \\
        a_{mj_1}x_1+a_{mj_2}x_2+\cdots+a_{mj_u}x_u=0 \\
    \end{cases}\]
    的非零解.
    
    $\therefore A^{(j_1)},A^{(j_2)},\cdots,A^{(j_u)}$ 线性相关. $\therefore$ 如果 $A^{(j_1)},A^{(j_2)},\cdots,A^{(j_s)}$ 是 $A^{(1)},A^{(2)},\cdots,A^{(n)}$ 的极大线性无关组, 那么 $\widetilde{A}^{(j_1)},\widetilde{A}^{(j_2)},\cdots,\widetilde{A}^{(j_s)}$ 线性无关.
    
    $\therefore s\leq t$. $\therefore s\leq r$.

    (2) 考察矩阵 ${}^tA=({}^tA_{(1)},{}^tA_{(2)},\cdots,{}^tA_{(m)})=[{}^tA^{(1)},{}^tA^{(2)},\cdots,{}^tA^{(n)}]$, $\because\left<{}^tA_{(1)},{}^tA_{(2)},\cdots,{}^tA_{(m)}\right>=\left<A_{(1)},A_{(2)},\cdots,A_{(m)}\right>$, $\therefore r_c({}^tA)=r_r(A)=r$. 同理 $r_r({}^tA)=r_c(A)=s$. 与 (1) 类似, 有
    \[r=r_c({}^tA)\leq r_r({}^tA)=s.\]

    另一方面, $r\geq s$, $\therefore r=s$.
\end{proof}
\setcounter{exercise}{2}
\begin{exercise}% 2.4
    设两个矩阵
    \[A=\begin{pmatrix}
        \alpha_1 & \alpha_2 & \cdots & \alpha_n \\
        \beta_1 &  \beta_2 & \cdots &  \beta_n
    \end{pmatrix},\quad B=\begin{pmatrix}
        \alpha_1 & \alpha_2 & \cdots & \alpha_n \\
        \beta_1 &  \beta_2 & \cdots &  \beta_n \\
        \gamma_1 & \gamma_2 & \cdots & \gamma_n
    \end{pmatrix},\]

    用平面上 $n$ 条直线组成的集合的几何性质给出 $A$ 和 $B$ 有相等秩的条件.
\end{exercise}
\begin{solution}
    平面上 $n$ 条直线组成的集合为
    \[\{\alpha_ix+\beta_iy=\gamma_i|\alpha_i,\beta_i,\gamma_i\in\mathbb{R},i=1,2,\cdots,n\}.\]

    如果这 $n$ 条直线有公共点 $(p,q)$, 则
    \[p\alpha_i+q\beta_i=\gamma_i\quad(i=1,2,\cdots,n).\]

    $\therefore$
    \[pB_{(1)}+qB_{(2)}=B_{(3)}.\]

    $\therefore$
    \[\left<B_{(1)},B_{(2)},B_{(3)}\right>=\left<B_{(1)},B_{(2)}\right>=\left<A_{(1)},A_{(2)}\right>.\]

    $\therefore\rank B=\rank A$.
\end{solution}
\subsection{习题2.3}
\stepcounter{exsection}
\stepcounter{exercise}
\begin{exercise}% 3.2
    (1) 证明: $\forall m\in\mathbb{N}_+$,
    \[\begin{pmatrix}
        1 & a & c \\
        0 & 1 & b \\
        0 & 0 & 1 \\
    \end{pmatrix}^m=\begin{pmatrix}
        1 & ma & \frac{m(m-1)}{2}ab+mc \\
        0 & 1 & mb \\
        0 & 0 & 1 \\
    \end{pmatrix}.\]

    (2) 求
    \[\begin{pmatrix}
        1 & a & c \\
        0 & 1 & b \\
        0 & 0 & 1 \\
    \end{pmatrix}^{-1}.\]
\end{exercise}
\begin{solution}
    (1) 用数学归纳法. 容易验证当 $m=1$ 时结论成立. 假设有
    \[\begin{pmatrix}
        1 & a & c \\
        0 & 1 & b \\
        0 & 0 & 1 \\
    \end{pmatrix}^{m-1}=\begin{pmatrix}
        1 & (m-1)a & \frac{(m-1)(m-2)}{2}ab+(m-1)c \\
        0 & 1 & (m-1)b \\
        0 & 0 & 1 \\
    \end{pmatrix},\]

    则
    \begin{align*}
        \begin{pmatrix}
            1 & a & c \\
            0 & 1 & b \\
            0 & 0 & 1 \\
        \end{pmatrix}^m & =\begin{pmatrix}
            1 & a & c \\
            0 & 1 & b \\
            0 & 0 & 1 \\
        \end{pmatrix}^{m-1}\begin{pmatrix}
            1 & a & c \\
            0 & 1 & b \\
            0 & 0 & 1 \\
        \end{pmatrix} \\
        & =\begin{pmatrix}
            1 & (m-1)a & \frac{(m-1)(m-2)}{2}ab+(m-1)c \\
            0 & 1 & (m-1)b \\
            0 & 0 & 1 \\
        \end{pmatrix}\begin{pmatrix}
            1 & a & c \\
            0 & 1 & b \\
            0 & 0 & 1 \\
        \end{pmatrix} \\
        & =\begin{pmatrix}
            1 & ma & \frac{m(m-1)}{2}ab+mc \\
            0 & 1 & mb \\
            0 & 0 & 1 \\
        \end{pmatrix}.
    \end{align*}

    (2) 由 (1) 得 $\forall m\in\mathbb{N}_+$,
    \[\begin{pmatrix}
        1 & (m-1)a & \frac{(m-1)(m-2)}{2}ab+(m-1)c \\
        0 & 1 & (m-1)b \\
        0 & 0 & 1 \\
    \end{pmatrix}\begin{pmatrix}
        1 & a & c \\
        0 & 1 & b \\
        0 & 0 & 1 \\
    \end{pmatrix}=\begin{pmatrix}
        1 & ma & \frac{m(m-1)}{2}ab+mc \\
        0 & 1 & mb \\
        0 & 0 & 1 \\
    \end{pmatrix}.\]

    当 $m=0$ 时上式也成立. 有
    \[\begin{pmatrix}
        1 & -a & ab-c \\
        0 & 1 & -b \\
        0 & 0 & 1 \\
    \end{pmatrix}\begin{pmatrix}
        1 & a & c \\
        0 & 1 & b \\
        0 & 0 & 1 \\
    \end{pmatrix}=\begin{pmatrix}
        1 & 0 & 0 \\
        0 & 1 & 0 \\
        0 & 0 & 1 \\
    \end{pmatrix}.\]

    $\therefore$
    \[\begin{pmatrix}
        1 & a & c \\
        0 & 1 & b \\
        0 & 0 & 1 \\
    \end{pmatrix}^{-1}=\begin{pmatrix}
        1 & -a & ab-c \\
        0 & 1 & -b \\
        0 & 0 & 1 \\
    \end{pmatrix}.\]
\end{solution}
\stepcounter{exercise}
\begin{exercise}% 3.4
    由 Markov 矩阵
    \[P=(p_{ij}),\quad p_{ij}\geq0,\quad\sum\limits_{i=1}^np_{ij}=1,\quad i=1,2,\cdots,n\]
    
    确定的线性变换 $\phi_P$ 通常作用于概率列向量
    \[X=[x_1,\cdots,x_n],\quad x_i\geq0,\quad\sum\limits_{i=1}^nx_i=1.\]
    
    证明:
    \begin{itemize}
        \item[(a)] 矩阵 $P\in M_n(\mathbb{R})$ 是 Markov 的, 当且仅当对任一概率向量 $X$, 向量 $PX$ 仍然是概率向量.
        \item[(b)] 如果 $P$ 是正的 Markov 矩阵($\forall i,j,p_{ij}>0$), 那么对任一概率向量 $X,PX$ 是正的概率向量.
        \item[(c)] 如果矩阵 $P,Q$ 是 Markov 的, 那么矩阵 $A=PQ$ 也是 Markov 的.
    \end{itemize}
\end{exercise}
\begin{proof}
    (a) 记
    \[PX=[y_1,y_2,\cdots,y_n],\]

    由书上的式 (11) 得
    \[y_i=P_{(i)}X=\sum\limits_{j=1}^np_{ij}x_{j}\geq0,\]

    $\therefore$
    \[\sum\limits_{i=1}^ny_i=\sum\limits_{i=1}^n\sum\limits_{j=1}^np_{ij}x_{j}=\sum\limits_{j=1}^nx_{j}\sum\limits_{i=1}^np_{ij}=\sum\limits_{j=1}^nx_{j}=1.\]

    (b) $\because x_i\geq0,\sum\limits_{i=1}^nx_i=1,\therefore\exists k$ 满足 $1\leq k\leq n$ 使得 $x_k>0.\therefore$
    \[y_i=\sum\limits_{j=1}^np_{ij}x_{j}\geq p_{ik}x_k>0.\]

    (c) $\because Q$ 是 Markov 的, $\therefore Q^{(i)}\ (i=1,2,\cdots,n)$ 是概率向量.

    由 (a) 得 $PQ^{(i)}$ 是概率向量. $\therefore PQ$ 也是 Markov 的.
\end{proof}
\stepcounter{exercise}
\begin{exercise}% 3.6
    由 $S_n$ 中的 $n$ 阶循环确定的置换矩阵为
    \[P=\begin{pmatrix}
        0 & 0 & \cdots & 0 & 1 \\
        1 & 0 & \cdots & 0 & 0 \\
        0 & 1 & \cdots & 0 & 0 \\
        \vdots & \vdots & \ddots & \vdots & \vdots \\
        0 & 0 & \cdots & 1 & 0 \\
    \end{pmatrix}.\]

    验证 $P^n=E$.
\end{exercise}
\begin{solution}
    $\because PE^{(1)}=E^{(2)},PE^{(2)}=E^{(3)},\cdots,PE^{(n-1)}=E^{(n)},PE^{(n)}=E^{(1)},\therefore$
    \[P^{n}E^{(1)}=P^{n-1}E^{(2)}=\cdots=PE^{(n)}=E^{(1)},\]
    \[P^{n}E^{(2)}=P^{n-1}E^{(3)}=\cdots=PE^{(1)}=E^{(2)},\]
    \[\cdots,\]
    \[P^{n}E^{(n)}=P^{n-1}E^{(1)}=\cdots=PE^{(n-1)}=E^{(n)},\]

    $\therefore P^n=E$.
\end{solution}
\begin{exercise}% 3.7
    对于任意两个 $m\times n$ 矩阵 $A,B$, 证明
    \[\rank (A+B)\leq\rank A+\rank B.\]
\end{exercise}
\begin{proof}
    在等式
    \begin{equation}\label{eq2.2}
        \sum\limits_{i=1}^n\lambda_iA^{(j)}=0
    \end{equation}

    两边乘一非零常数 $\mu$, 等式仍然成立. $\therefore$ 由式 (\ref{eq2.2}) 得
    \[\sum\limits_{i=1}^n\mu\lambda_iA^{(j)}=0\Rightarrow\sum\limits_{i=1}^n\lambda_i(\mu A^{(j)})=0.\]

    $\therefore$ 对向量组的每个向量乘一非零常数不改变向量组的线性相关性. $\therefore$
    \[\rank (-B)=\rank B.\]

    假设 $\rank(A+B)>\rank A+\rank B$, 则
    \begin{align*}
        \rank A & =\rank (A+B-B) \\
        & >\rank (A+B)+\rank (-B) \\
        & >\rank A+\rank B+\rank (-B) \\
        & =\rank A+2\rank B.
    \end{align*}

    $\therefore 0>2\rank B$, 与 $\rank B\geq0$ 矛盾.
\end{proof}
\begin{proof}[另一种证明]
    $A+B$ 的列空间为
    \[\left<A^{(1)}+B^{(1)},A^{(2)}+B^{(2)},\cdots,A^{(n)}+B^{(n)}\right>=\left<A^{(1)},A^{(2)},\cdots,A^{(n)}\right>+\left<B^{(1)},B^{(2)},\cdots,B^{(n)}\right>.\]

    由习题 \ref{ex1.7} 得
    \begin{align*}
        \rank(A+B) & =\dim\left(\left<A^{(1)},A^{(2)},\cdots,A^{(n)}\right>+\left<B^{(1)},B^{(2)},\cdots,B^{(n)}\right>\right) \\
        & \leq\dim\left<A^{(1)},A^{(2)},\cdots,A^{(n)}\right>+\dim\left<B^{(1)},B^{(2)},\cdots,B^{(n)}\right> \\
        & =\rank A+\rank B. \qedhere
    \end{align*}
\end{proof}

证明第 \ref{ex3.8} 题需要证明两个引理.
\begin{lemma}\label{l2.3}
    是 $A$ 是 $m\times s$ 矩阵, $B$ 是 $s\times n$ 矩阵, $A$ 确定的线性映射为 $\phi_A$, $B$ 的列空间为 $V_c(B)$, 则
    \[\dim\phi_A(V_c(B))=\rank AB.\]
\end{lemma}
\begin{proof}
    设 $AB$ 的列空间为 $V_c(AB)$, 有
    \[\dim V_c(AB)=\rank AB.\]

    $\therefore$ 只需证明 $V_c(AB)=\phi_A(V_c(B))$.

    $\because\phi_A(B^{(i)})=AB^{(i)}\ (i=1,2,\cdots,n)$, $\therefore$
    \[V_c(AB)=\left<AB^{(1)},AB^{(2)},\cdots,AB^{(n)}\right>=\left<\phi_A(B^{(1)}),\phi_A(B^{(2)}),\cdots,\phi_A(B^{(n)})\right>.\]

    $\because\left\{\phi_A(B^{(1)}),\cdots,\phi_A(B^{(n)})\right\}\subset\phi_A(V_c(B))$, 由引理 \ref{l2.1} 得 $\left<\phi_A(B^{(1)}),\cdots,\phi_A(B^{(n)})\right>\subset\phi_A(V_c(B))$.

    $\forall X\in\phi_A(V_c(B)),\exists Y\in V_c(B)$ 使得 $X=\phi_A(Y)$.
    
    $\because Y\in V_c(B)$, $\therefore\exists c_1,c_2,\cdots,c_n$ 使得 $Y=\sum\limits_{i=1}^nc_iB^{(i)}$. 有
    \[X=\phi_A\left(\sum\limits_{i=1}^nc_iB^{(i)}\right)=\sum\limits_{i=1}^nc_i\phi_A(B^{(i)})\in\left<\phi_A(B^{(1)}),\phi_A(B^{(2)}),\cdots,\phi_A(B^{(n)})\right>.\]

    $\therefore\phi_A(V_c(B))\subset\left<\phi_A(B^{(1)}),\phi_A(B^{(2)}),\cdots,\phi_A(B^{(n)})\right>$.
\end{proof}
\begin{lemma}\label{l2.4}
    设 $\phi$ 是线性映射, $V_1,V_2$ 是 $\mathbb{R}$ 的线性子空间, 则
    \[\phi(V_1+V_2)=\phi(V_1)+\phi(V_2).\]
\end{lemma}
\begin{proof}
    $\forall Y'\in\phi(V_1+V_2),\exists Y\in V_1+V_2,\phi(Y)=Y'$.

    $\because Y\in V_1+V_2,\therefore\exists X_1\in V_1,X_2\in V_2$ 使得 $Y=X_1+X_2$.\footnote{这样的 $X_1,X_2$ 不一定是唯一的, 不过这不影响证明, 下面的 $\exists X_1',X_2'$ 也一样} $\therefore$
    \[Y'=\phi(Y)=\phi(X_1+X_2)=\phi(X_1)+\phi(X_2).\]

    $\therefore\phi(V_1+V_2)\subset\phi(V_1)+\phi(V_2)$.

    $\forall Y'\in\phi(V_1)+\phi(V_2),\exists X_1'\in\phi(V_1),X_2'\in\phi(V_2),Y'=X_1'+X_2'$.

    $\because\exists X_1\in V_1,X_2\in V_2,\phi(X_1)=X_1',\phi(X_2)=X_2'$,

    $\therefore\exists Y=X_1+X_2\in V_1+V_2$,
    \[\phi(Y)=\phi(X_1+X_2)=\phi(X_1)+\phi(X_2)=X_1'+X_2'=Y.\]

    $\therefore\phi(V_1+V_2)\supset\phi(V_1)+\phi(V_2)$.
\end{proof}
\begin{exercise}\label{ex3.8}
    对于任意 $m\times s$ 矩阵 $A$ 和 $s\times n$ 矩阵 $B$, 证明
    \[\rank A+\rank B-s\leq\rank  AB.\]
\end{exercise}
下面是比较容易想到的证明方法(至少是我自己想到的方法).
\begin{thought}
    首先我们需要处理不等式中看似与矩阵的秩无关的 $s$ (事实上, 对 $s$ 的不同处理方法会导出不同的证明, 比如说将 $s$ 看成是 $M_s(\mathbb{R})$ 中的单位矩阵的秩就可以得到一般的教科书上的利用分块矩阵的证明). 这里我们利用等式 $s=\dim\mathbb{R}^s$ 将不等式变为
    \[\rank A+\rank B-\dim\mathbb{R}^s\leq\rank  AB.\]

    然后我们把矩阵的秩都转化为线性子空间的维数. 注意到:
    \begin{enumerate}
        \item $V_c(B)\subset\dim\mathbb{R}^s$,
        \item 设 $A$ 确定的线性映射为 $\phi_A$, 则 $\rank A=\dim\phi_A(\mathbb{R}^s),V_c(AB)=\phi_A(V_c(B))$,
    \end{enumerate}

    将不等式变为
    \begin{align*}
        \dim\phi_A(\mathbb{R}^s)-\dim\phi_A(V_c(B)) & \leq\dim\mathbb{R}^s-\dim V_c(B) \\
        & =\dim\mathbb{R}^s\backslash V_c(B).
    \end{align*}

    如果能证明
    \[\dim\phi_A(\mathbb{R}^s)-\dim\phi_A(V_c(B))\leq\dim\phi_A(\mathbb{R}^s\backslash V_c(B)),\]

    那么只需证明
    \[\dim\phi_A(\mathbb{R}^s\backslash V_c(B))\leq\dim\mathbb{R}^s\backslash V_c(B).\]
\end{thought}
\begin{proof}
    设 $A$ 确定的线性映射为 $\phi_A$, $V_c(B)\subset\dim\mathbb{R}^s$ 为 $B$ 的列空间.

    令 $C$ 是满足 $V_c(C)+V_c(B)=\mathbb{R}^s$ 的矩阵.\footnote{这样的 $C$ 不一定是唯一的, 不过这不影响证明} 由书上的定理 3 得
    \[\rank AC\leq\rank C.\]

    由引理 \ref{l2.3} 得
    \[\rank AC=\dim\phi_A(V_c(C)),\]

    $\therefore$
    \begin{equation}\label{eq2.3}
        \dim\phi_A(V_c(C))\leq\dim V_c(C).
    \end{equation}

    $\because V_c(C)+V_c(B)=\mathbb{R}^s$, 由引理 \ref{l2.4} 得
    \[\phi_A(V_c(C))+\phi_A(V_c(B))=\phi_A(V_c(C)+V_c(B))=\phi_A(\mathbb{R}^s).\]

    由习题 \ref{ex1.7} 得
    \begin{align*}
        \dim\phi_A(\mathbb{R}^s) & =\dim(\phi_A(V_c(C))+\phi_A(V_c(B))) \\
        & \leq\dim\phi_A(V_c(C))+\dim\phi_A(V_c(B)).
    \end{align*}

    由 (\ref{eq2.3}) 式得
    \begin{equation}\label{eq2.4}
        \dim\phi_A(\mathbb{R}^s)\leq\dim V_c(C)+\dim\phi_A(V_c(B)).
    \end{equation}

    由引理 \ref{l2.3} 得
    \[\dim\phi_A(\mathbb{R}^s)=\rank AE=\rank A,\]
    \[\dim\phi_A(V_c(B))=\rank AB,\]

    代入 (\ref{eq2.4}) 式得
    \[\rank A\leq\dim V_c(C)+\rank AB.\]

    由习题 \ref{ex1.7} 得
    \begin{align*}
        & V_c(C)+V_c(B)=\mathbb{R}^s \\
        & \Rightarrow\dim (V_c(C)+V_c(B))=\dim\mathbb{R}^s \\
        & \Rightarrow\dim V_c(C)+\dim V_c(B)\leq\dim\mathbb{R}^s \\
        & \Rightarrow\dim V_c(C)\leq\dim\mathbb{R}^s-\dim V_c(B)=s-\rank B.
    \end{align*}

    代入 (\ref{eq2.4}) 式得
    \[\rank A\leq s-\rank B+\rank AB\Rightarrow\rank A+\rank B-s\leq\rank AB.\qedhere\]
\end{proof}
\setcounter{exercise}{10}
\begin{exercise}[有修改]
    若 $A=(a_{ij})$ 是非退化(斜)对称矩阵, 证明 $A^{-1}$ 是(斜)对称矩阵.
\end{exercise}
\begin{solution}
    在 $AA^{-1}=E$ 两边转置得 ${}^t(A^{-1}){}^tA=E$.

    如果 $A$ 是对称矩阵, 则 ${}^tA=A$, $\therefore{}^t(A^{-1})A=E\Rightarrow{}^t(A^{-1})=A^{-1}$, 即 $A^{-1}$ 是对称矩阵.

    如果 $A$ 是斜对称矩阵, 则 ${}^tA=-A$, $\therefore-{}^t(A^{-1})A=E\Rightarrow{}^t(A^{-1})=-A^{-1}$, 即 $A^{-1}$ 是斜对称矩阵.
\end{solution}
\setcounter{exercise}{13}
\begin{exercise}% 3.14
    设
    \[A=\begin{pmatrix}
        a & b \\
        c & d
    \end{pmatrix}.\]
    
    利用关系 $A^2=(a+d)A-(ad-bc)E$ 求解 $A^{-1}$.
\end{exercise}
\begin{solution}
    (1) 设 $ad-bc\neq0$.
    \begin{align*}
        & A^2=(a+d)A-(ad-bc)E \\
        & \Rightarrow A^2=(a+d)AE-(ad-bc)E \\
        & \Rightarrow A^2+(-a-d)AE=-(ad-bc)E.
    \end{align*}

    由线性映射的性质(ii),
    \[(-a-d)AE=A(-a-d)E=A\diag(-a-d,-a-d).\]

    $\therefore$ 原式
    \[\Rightarrow A^2+A\diag(-a-d,-a-d)=-(ad-bc)E.\]

    由线性映射的性质(i), 原式
    \[\Rightarrow A(A-(a+d)E)=-(ad-bc)E.\]

    由线性映射的性质(ii), 原式
    \[\Rightarrow A\left(-\dfrac{1}{ad-bc}A+\dfrac{a+d}{ad-bc}E\right)=E.\]

    $\therefore$
    \begin{align*}
        A^{-1} & =\left(-\dfrac{1}{ad-bc}A+\dfrac{a+d}{ad-bc}E\right) \\
        & =\begin{pmatrix}
            -\dfrac{a}{ad-bc}+\dfrac{a+d}{ad-bc} & -\dfrac{b}{ad-bc} \\[8pt]
            -\dfrac{c}{ad-bc} & -\dfrac{d}{ad-bc}+\dfrac{a+d}{ad-bc}
        \end{pmatrix} \\
        & =\begin{pmatrix}
            \dfrac{d}{ad-bc} & -\dfrac{b}{ad-bc} \\[8pt]
            -\dfrac{c}{ad-bc} & \dfrac{a}{ad-bc}
        \end{pmatrix}.
    \end{align*}

    (2) 假设 $ad-bc=0$ 且 $A$ 可逆, 有 $A^2=(a+d)A$. $\therefore$
    \[A=A^2A^{-1}=(a+d)AA^{-1}=\diag(a+d,a+d).\]

    $\therefore a+d=a=d,a=d=0,A=0$, 与 $A$ 可逆矛盾. $\therefore$ 当 $ad-bc=0$ 时 $A$ 不可逆.
\end{solution}
\stepcounter{exercise}
\begin{exercise}% 3.16
    证明: 若
    \[\begin{pmatrix}
        a & b \\
        c & d
    \end{pmatrix}^m=0,\]

    则
    \[\begin{pmatrix}
        a & b \\
        c & d
    \end{pmatrix}^2=0.\]
\end{exercise}
\begin{proof}
    设
    \[A=\begin{pmatrix}
        a & b \\
        c & d
    \end{pmatrix}.\]

    由第 \ref{ex3.8} 题和书上第 3 节的定理 3 得
    \[\rank A+\rank A-2\leq\rank A^2\leq\rank A,\]
    \[\rank A^2+\rank A-2\leq\rank A^3\leq\rank A^2,\]
    \[\cdots,\]
    \[\rank A^{n-1}+\rank A-2\leq\rank A^m\leq\rank A^{n-1},\]

    假设 $\rank A=2$, 则 $\rank A^m=\rank A^{m-1}=\cdots=\rank A=2$, 与 $A^m=0\Rightarrow\rank A^m=0$ 矛盾. $\therefore\rank A\leq 1$.

    如果 $\rank A=0$, 那么 $A=A^2=0$. 如果 $\rank A=1$, 由补充题 \ref{ex3.3.6} 得存在 $2\times1$ 矩阵 $B$ 和 $1\times 2$ 矩阵 $C$ 使得 $A=BC$. $CB$ 是一个数, 设 $CB=k$, 有
    \[A^2=BCBC=B(kE)C=kBC=kA,\]
    \[A^m=\underbrace{BC\cdots BC}_{m\text{个}BC}=BCB\underbrace{CB\cdots CB}_{m-2\text{个}CB}C=k^{m-2}BCBC=k^{m-2}A^2=k^{m-1}A.\]

    $\because A\neq0$, $\therefore A^m=0\Rightarrow k^{m-1}=0\Rightarrow k=0$. $\therefore A^2=kA=0$.
\end{proof}
\begin{exercise}\label{ex3.17}
    设 $m\times s$ 矩阵
    \[X=\begin{pmatrix}
        X_{11} & X_{12} & \cdots & X_{1k} \\
        X_{21} & X_{22} & \cdots & X_{2k} \\
        \vdots & \vdots & \ddots & \vdots \\
        X_{l1} & X_{l2} & \cdots & X_{lk}
    \end{pmatrix}\]
    的块 $X_{ij}$ 是 $m_i\times s_j$ 矩阵 $(m_1+m_2+\cdots+m_l=m,s_1+s_2+\cdots+s_k=s)$, $s\times n$ 矩阵
    \[Y=\begin{pmatrix}
        Y_{11} & Y_{12} & \cdots & Y_{1r} \\
        Y_{21} & Y_{22} & \cdots & Y_{2r} \\
        \vdots & \vdots & \ddots & \vdots \\
        Y_{k1} & Y_{k2} & \cdots & Y_{kr} \\
    \end{pmatrix}\]
    的块 $Y_{ij}$ 是 $s_i\times n_j$ 矩阵 $(n_1+n_2+\cdots+n_r=n)$, 则 $Z=XY$ 可以分块计算, 它的块
    \[Z_{ij}=\sum\limits_{t=1}^kX_{it}Y_{tj}.\]
\end{exercise}
\begin{proof}
    将 $X,Y,Z$ 分成下列矩阵的和:
    \[X=\sum\limits_{i=1}^{l}\sum\limits_{j=1}^{k}{}^{(ij)}X,\]
    \[Y=\sum\limits_{i=1}^{k}\sum\limits_{j=1}^{r}{}^{(ij)}Y,\]
    \[Z=\sum\limits_{i=1}^{l}\sum\limits_{j=1}^{r}{}^{(ij)}Z,\]

    其中 ${}^{(ij)}X$ 的块 ${}^{(ij)}X_{ij}=X_{ij}$, 其余块都是零矩阵, ${}^{(ij)}Y$ 的块 ${}^{(ij)}Y_{ij}=Y_{ij}$, 其余块都是零矩阵, ${}^{(ij)}Z$ 的块 ${}^{(ij)}Z_{ij}=Z_{ij}$, 其余块都是零矩阵.

    由矩阵乘法的分配律得
    \[Z=XY=\sum\limits_{a=1}^{l}\sum\limits_{b=1}^{k}\sum\limits_{c=1}^{k}\sum\limits_{d=1}^{r}{}^{(ab)}X{}^{(cd)}Y.\]

    定义 $s_{(0)}=0,s_{(i)}=s_1+s_2+\cdots+s_i$ (对 $m,n$ 也有类似的定义), 则 ${}^{(ab)}X{}^{(cd)}Y$ 的每一列都可以看成是 ${}^{(ab)}X$ 的第 $s_{(c-1)}+1,s_{(c-1)}+2,\cdots,s_{(c)}$ 列的线性组合.

    如果 $b\neq c$, 那么 ${}^{(ab)}X$ 的第 $s_{(c-1)}+1,s_{(c-1)}+2,\cdots,s_{(c)}$ 列都是 $0$.

    $\therefore b\neq c\Rightarrow{}^{(ab)}X{}^{(cd)}Y=0$,
    \[Z=\sum\limits_{a=1}^{l}\sum\limits_{b=1}^{k}\sum\limits_{d=1}^{r}{}^{(ab)}X{}^{(bd)}Y=\sum\limits_{a=1}^{l}\sum\limits_{d=1}^{r}\left(\sum\limits_{b=1}^{k}{}^{(ab)}X{}^{(bd)}Y\right).\]

    $\because$
    \[Z=\sum\limits_{a=1}^{l}\sum\limits_{d=1}^{r}{}^{(ad)}Z,\]

    $\therefore$
    \begin{equation}\label{eq2.5}
        \sum\limits_{a=1}^{l}\sum\limits_{d=1}^{r}\left({}^{(ad)}Z-\sum\limits_{b=1}^{k}{}^{(ab)}X{}^{(bd)}Y\right)=0.
    \end{equation}

    考察 $^{(ad)}Z'=\sum\limits_{b=1}^{k}{}^{(ab)}X{}^{(bd)}Y$ 的块 $^{(ad)}Z'_{ij}$. 设 $i\neq a$, 则 $\sum\limits_{b=1}^{k}{}^{(ab)}X{}^{(bd)}Y$ 中的每一个 $^{(ab)}X$ 的第 $m_{(i-1)}+1,m_{(i-1)}+2,\cdots,m_{(i)}$ 行都为 $0$,

    $\therefore{}^{(ad)}Z'$ 的第 $m_{(i-1)}+1,m_{(i-1)}+2,\cdots,m_{(i)}$ 行都为 $0$, $\therefore Z'_{ij}=0$.

    同理, 若 $j\neq d$, 则 $^{(bd)}Y$ 的第 $n_{(j-1)}+1,n_{(j-1)}+2,\cdots,n_{(j)}$ 列都为 $0$, $Z'_{ij}=0$.

    $\therefore{}^{(ad)}Z'$ 的块 $^{(ad)}Z'_{ij}$ 中除 $^{(ad)}Z'_{ad}$ 外都是零矩阵.

    $\therefore{}^{(ad)}Z-{}^{(ad)}Z'$ 的块 $^{(ad)}Z_{ij}-^{(ad)}Z'_{ij}$ 中除 $^{(ad)}Z_{ad}-^{(ad)}Z'_{ad}$ 外都是零矩阵.

    $\because\forall a,d$,
    \[{}^{(ad)}Z'-{}^{(ad)}Z=\sum\limits_{i=m(a-1)}^{m(a)}\sum\limits_{j=n(d-1)}^{n(d)}\lambda_{ij}E_{ij},\]

    其中 $E_{ij}$ 为矩阵单位, 定义见书上的定理 4, $\therefore$ (\ref{eq2.5}) 式 $\Rightarrow$
    \begin{equation}\label{eq2.6}
        \sum\limits_{a=1}^{l}\sum\limits_{d=1}^{r}\sum\limits_{i=m(a-1)}^{m(a)}\sum\limits_{j=n(d-1)}^{n(d)}\lambda_{ij}E_{ij}=0.
    \end{equation}

    $\because\{E_{ij}|i=1,2,\cdots,m,j=1,2,\cdots,n\}$ 线性无关, $\therefore$ 关于 $\lambda_{ij}$ 的方程 (\ref{eq2.6}) 只有零解, $\therefore\forall a,d,\forall i=m(a-1),\cdots,m(a),j=n(d-1),\cdots,n(d),\lambda_{ij}=0$. $\therefore\forall a,b,{}^{(ad)}Z'-{}^{(ad)}Z=0$, $\therefore$
    \[^{(ab)}Z_{ab}=\sum\limits_{t=1}^k{}^{(at)}X_{at}{}^{(tb)}Y_{tb}.\]

    $\therefore$
    \[Z_{ij}=\sum\limits_{t=1}^kX_{it}Y_{tj}.\qedhere\]
\end{proof}
\begin{note}
    把 $^{(ad)}Z$ 之类的矩阵换成矩阵单位, 可以用类似的方法推导矩阵的乘法公式.
\end{note}
\subsection{补充题}
\begin{exercisec}[3.1.3(4)]
    判断向量组
    \[X_1=(4,-5,2,6),X_2=(2,-2,1,3),X_3=(5,-3,3,9),X_4=(4,-1,5,6)\]
    是否线性无关, 并计算向量组的秩.
\end{exercisec}
\begin{solution}
    考察关于 $y_1,y_2,y_3,y_4$ 的方程组
    \[y_1X_1+y_2X_2+y_3X_3+y_4X_4=0,\]

    即
    \[\begin{cases}
        4y_1+2y_2+5y_3+4y_4=0, \\
        -5y_1-2y_2-3y_3-y_4=0, \\
        2y_1+y_2+3y_3+5y_4=0, \\
        6y_1+3y_2+9y_3+6y_4=0. \\
    \end{cases}\]

    这个方程组只有零解, $\therefore$ 向量组线性无关, 秩为 $4$.
\end{solution}
\begin{exercisec}[3.1.4(1)]
    假设向量组 $X_1,X_2,\cdots,X_k$ 线性无关. 判断向量组
    \[\begin{cases}
        Y_1=3X_1+2X_2+X_3+X_4, \\
        Y_2=2X_1+5X_2+3X_3+2X_4, \\
        Y_3=3X_1+4X_2-X_3+2X_4 \\
    \end{cases}\]
    是否线性无关, 并计算向量组的秩.
\end{exercisec}
\begin{solution}
    考察关于 $y_1,y_2,y_3$ 的方程组
    \[y_1Y_1+y_2Y_2+y_3Y_3=0,\]
    
    即
    \begin{align*}
        & y_1(3X_1+2X_2+X_3+X_4)+y_2(2X_1+5X_2+3X_3+2X_4)+y_3(3X_1+4X_2-X_3+2X_4)=0 \\
        & \Rightarrow(3y_1+2y_2+3y_3)X_1+(2y_1+5y_2+4y_3)X_2+(y_1+3y_2-y_3)X_3+(y_1+2y_2+2y_3)X_4=0.
    \end{align*}

    $\because X_1,X_2,\cdots,X_k$ 线性无关, $\therefore$
    \[\begin{cases}
        3y_1+2y_2+3y_3=0, \\
        2y_1+5y_2+4y_3=0, \\
        y_1+3y_2-y_3=0, \\
        y_1+2y_2+2y_3=0. \\
    \end{cases}\]

    这个方程组只有零解, $\therefore$ 向量组线性无关, 秩为 $3$.
\end{solution}
\begin{exercisec}[3.2.1(1)]\label{ex3.2.1}
    证明: 初等列变换不改变矩阵的秩.
\end{exercisec}
\begin{proof}
    由书上的引理 3.11 (1) 对称地可以证明. 这里从略.
\end{proof}
\begin{exercisec}[3.2.2(5)]
    求矩阵的秩:
    \[A=\begin{pmatrix}
        x & 1 & 2 & 3 & \cdots & n-1 & 1 \\
        1 & x & 2 & 3 & \cdots & n-1 & 1 \\
        1 & 2 & x & 3 & \cdots & n-1 & 1 \\
        \vdots & \vdots & \vdots & \vdots && \vdots & \vdots \\
        1 & 2 & 3 & 4 & \cdots & x & 1 \\
        1 & 2 & 3 & 4 & \cdots & n & 1 \\
    \end{pmatrix}\ (x \text{是变量})\]
\end{exercisec}
\begin{solution}
    $A$ 的第 $i\ (i\neq n)$ 行减去 $i$ 倍的第 $n$ 行, 得
    \[A'=\begin{pmatrix}
        x-1 & -1 & -1 & -1 & \cdots & -1 & 1 \\
        0 & x-2 & -1 & -1 & \cdots & -1 & 1 \\
        0 & 0 & x-3 & -1 & \cdots & -1 & 1 \\
        \vdots & \vdots & \vdots & \vdots && \vdots & \vdots \\
        0 & 0 & 0 & 0 & \cdots & x-n & 1 \\
        0 & 0 & 0 & 0 & \cdots & 0 & 1 \\
    \end{pmatrix}.\]

    书上定理 3.12 的证明过程得
    \[\rank A'=\begin{cases}
        n-1, & x\in\{1,2,\cdots,n\}, \\
        n, & x\notin\{1,2,\cdots,n\}. \\
    \end{cases}\]

    由补充题 \ref{ex3.2.1}, $\rank A=\rank A'$.
\end{solution}
\begin{exercisec}[3.2.3]
    证明: 若 $a_0\neq0$, 则方阵
    \[A=\begin{pmatrix}
        a_0 & 0 & 0 & \cdots & 0 & 0 & 0 \\
        a_1 & 0 & 0 & \cdots & 0 & 0 & 1 \\
        a_2 & 0 & 0 & \cdots & 0 & 1 & 0 \\
        \vdots & \vdots & \vdots && \vdots & \vdots & \vdots \\
        a_{n-1} & 0 & 1 & \cdots & 0 & 0 & 0 \\
        a_n & 1 & 0 & \cdots & 0 & 0 & 0 \\
    \end{pmatrix}\]
    的秩为 $n$.
\end{exercisec}
\begin{proof}
    考察方阵
    \[A'=\begin{pmatrix}
        0 & 0 & 0 & \cdots & 0 & 0 & a_0 \\
        1 & 0 & 0 & \cdots & 0 & 0 & a_1 \\
        0 & 1 & 0 & \cdots & 0 & 0 & a_2 \\
        \vdots & \vdots & \vdots & \cdots & \vdots & \vdots & \vdots \\
        0 & 0 & 0 & \cdots & 1 & 0 & a_{n-1} \\
        0 & 0 & 0 & \cdots & 0 & 1 & a_n \\
    \end{pmatrix},\]

    由补充题 \ref{ex3.2.1} 得 $\rank A'=\rank A$.

    交换 $A'$ 的第 $i,i+1$ ($i$ 按顺序从 $1$ 到 $n-1$), 得
    \[A''=\begin{pmatrix}
        1 & 0 & 0 & \cdots & 0 & 0 & a_1 \\
        0 & 1 & 0 & \cdots & 0 & 0 & a_2 \\
        \vdots & \vdots & \vdots & \cdots & \vdots & \vdots & \vdots \\
        0 & 0 & 0 & \cdots & 1 & 0 & a_{n-1} \\
        0 & 0 & 0 & \cdots & 0 & 1 & a_n \\
        0 & 0 & 0 & \cdots & 0 & 0 & a_0 \\
    \end{pmatrix}.\]

    $\because a_0\neq0$, 由书上定理 3.12 的证明过程得 $\rank A''=n$. 由书上的引理 3.11 得 $\rank A''=\rank A'$. $\therefore\rank A=n$.
\end{proof}
\begin{exercisec}[3.3.6]\label{ex3.3.6}
    证明: 如果 $m\times n$ 矩阵 $A$ 的秩为 $1$, 则存在 $m\times1$ 矩阵 $B$ 和 $1\times n$ 矩阵 $C$ 使得 $A=BC$.
\end{exercisec}
\begin{proof}
    $\because\rank A=1$, $\therefore A$ 至少有一列不全为零. 不妨设 $A^{(1)}\neq0$.
    
    $\because\rank A=1$, $\therefore\left<A^{(1)},A^{(2)},\cdots,A^{(n)}\right>=\left<A^{(1)}\right>$. $\therefore\exists\lambda_2,\lambda_3,\cdots,\lambda_n$ 使得 $A^{(i)}=\lambda_iA^{(1)}(i=2,3,\cdots,n)$. $\therefore$
    \[A=(A^{(1)},A^{(2)},\cdots,A^{(n)})=A^{(1)}(1,\lambda_2,\lambda_3,\cdots,\lambda_n).\qedhere\]
\end{proof}
\begin{exercisec}[第 3 章的习题 3.9]
    设 $A\in M_n(\mathbb{R}),B\in M_m(\mathbb{R})$ 可逆, $C$ 是任意的 $n\times m$ 矩阵, 证明
    \[\begin{pmatrix}
        A & C \\
        0 & B
    \end{pmatrix}\]
    可逆, 并求其逆.
\end{exercisec}
\begin{proof}
    对原矩阵和 $E_{m+n}$ 的拼接作分块矩阵的初等行变换, 如果左边能化为 $E_{m+n}$, 那么右边为原矩阵的逆, 否则原矩阵不可逆.
    \begin{align*}
        \left(\begin{array}{cc|cc}
            A & C & E_n & 0 \\
            0 & B & 0 & E_m
        \end{array}\right) & \xrightarrow{F_{(1,2)}(-CB^{-1})}\left(\begin{array}{cc|cc}
            A & 0 & E_n & -CB^{-1} \\
            0 & B & 0 & E_m
        \end{array}\right) \\
        & \xrightarrow{F_{(1)}(A^{-1}),F_{(2)}(B^{-1})}\left(\begin{array}{cc|cc}
            E_n & 0 & A^{-1} & -A^{-1}CB^{-1} \\
            0 & E_m & 0 & B^{-1}
        \end{array}\right),
    \end{align*}

    $\therefore$ 原矩阵可逆,
    \[\begin{pmatrix}
        A & C \\
        0 & B
    \end{pmatrix}^{-1}=\begin{pmatrix}
        A^{-1} & -A^{-1}CB^{-1} \\
        0 & B^{-1}
    \end{pmatrix}.\qedhere\]
\end{proof}
\begin{exercisec}[3.4.7(2)]
    求方阵
    \[X=\begin{pmatrix}
        0 & 0 & A \\
        0 & D & B \\
        F & G & C \\
    \end{pmatrix}\]
    的逆矩阵, 其中 $F,D,A$ 分别是 $t\times t,u\times u,v\times v$ 的可逆矩阵.
\end{exercisec}
\begin{solution}
    对任意的 $n$, 矩阵 $Y\in M_{n}(\mathbb{R})$, 定义 $Y'=F_{1,n}F_{2,n-2}\cdots F_{\lfloor n/2\rfloor,n-\lfloor n/2\rfloor}Y$ (直观上, $Y'$ 由 $Y$ 沿水平线对称得到), 有
    \begin{align*}
        & \left(\begin{array}{ccc|c}
            0 & 0 & A & \\
            0 & D & B & E_{t+u+v} \\
            F & G & C & \\
        \end{array}\right) \\
        & \xrightarrow{F_{1,t+u+v},F_{2,t+u+v-1},\cdots,F_{\lfloor(t+u+v)/2\rfloor,t+u+v-\lfloor(t+u+v)/2\rfloor+1}}\left(\begin{array}{ccc|ccc}
            F' & G' & C'& 0   & 0   & E_v' \\
            0  & D' & B'& 0   & E_u' & 0   \\
            0  & 0  & A'& E_t' & 0   & 0   \\
        \end{array}\right) \\
        & \xrightarrow{F_{u+v+1,u+v+t},F_{u+v+2,u+v+t-1},\cdots,F_{u+v+\lfloor t/2\rfloor,u+v+t-\lfloor t/2\rfloor+1}}\left(\begin{array}{ccc|ccc}
            F' & G' & C' & 0   & 0   & E_v' \\
            0  & D' & B' & 0   & E_u' & 0   \\
            0  & 0  & A  & E_t & 0   & 0   \\
        \end{array}\right) \\
        & \xrightarrow{F_{u+1,u+v},F_{u+2,u+v-1},\cdots,F_{u+\lfloor v/2\rfloor,u+v-\lfloor v/2\rfloor+1}}\left(\begin{array}{ccc|ccc}
            F & G & C & 0   & 0   & E_v' \\
            0 & D & B & 0   & E_u & 0   \\
            0 & 0 & A & E_t & 0   & 0   \\
        \end{array}\right) \\
        & \xrightarrow{F_{1,u},F_{2,u-1},\cdots,F_{\lfloor u/2\rfloor,u-\lfloor u/2\rfloor+1}}\left(\begin{array}{ccc|ccc}
            F & G & C & 0   & 0   & E_v \\
            0 & D & B & 0   & E_u & 0   \\
            0 & 0 & A & E_t & 0   & 0   \\
        \end{array}\right) \\
        & \xrightarrow{F_{(1,3)}(-CA^{-1}),F_{(2,3)}(-BA^{-1})}\left(\begin{array}{ccc|ccc}
            F & G & 0 & -CA^{-1} & 0   & E_v \\
            0 & D & 0 & -BA^{-1} & E_u & 0   \\
            0 & 0 & A & E_t      & 0   & 0   \\
        \end{array}\right) \\
        & \xrightarrow{F_{(1,2)}(-GD^{-1})}\left(\begin{array}{ccc|ccc}
            F & 0 & 0 & -CA^{-1}+GD^{-1}BA^{-1} & -GD^{-1} & E_v \\
            0 & D & 0 & -BA^{-1}                & E_u      & 0   \\
            0 & 0 & A & E_t                     & 0        & 0   \\
        \end{array}\right) \\
        & \xrightarrow{F_{(1)}(F^{-1}),F_{(2)}(D^{-1}),F_{(3)}(A^{-1})}\left(\begin{array}{ccc|ccc}
            E_v & 0 & 0 & -F^{-1}(CA^{-1}+GD^{-1}BA^{-1}) & -F^{-1}GD^{-1} & F^{-1} \\
            0 & E_u & 0 & -D^{-1}BA^{-1}                  & D^{-1}         & 0      \\
            0 & 0 & E_t & A^{-1}                          & 0              & 0      \\
        \end{array}\right).
    \end{align*}

    $\therefore$
    \[X^{-1}=\begin{pmatrix}
        -F^{-1}(CA^{-1}+GD^{-1}BA^{-1}) & -F^{-1}GD^{-1} & F^{-1} \\
        -D^{-1}BA^{-1}                  & D^{-1}         & 0      \\
        A^{-1}                          & 0              & 0      \\
    \end{pmatrix}.\]
\end{solution}
\begin{exercisec}[3.4.8]
    设 $A$ 和 $B$ 是方阵, 证明如果 $E+AB$ 可逆, 那么 $E+BA$ 可逆.
\end{exercisec}
\begin{proof}
    设 $X=(E+AB)^{-1}$, 则
    \[(E+AB)X=E\Rightarrow BXA+BABXA=B(E+AB)XA=BA.\]

    $\therefore$
    \[(E+BA)(E-BXA)=E+BA-BXA-BABXA=E+BA-BA=E.\]

    $\therefore$
    \[(E+BA)^{-1}=E-BXA.\qedhere\]
\end{proof}
\begin{exercisec}[3.5.1]
    求方程组的一个基础解系:
    \[\begin{cases}
        x_1+x_2+x_3+x_4+x_5=0, \\
        3x_1+2x_2+x_3+x_4-3x_5=0, \\
        x_2+2x_3+2x_4+6x_5=0, \\
        5x_1+4x_2+3x_3+3x_4-x_5=0. \\
    \end{cases}\]
\end{exercisec}
\begin{solution}
    将方程组化为阶梯型:
    \begin{align*}
        \begin{pmatrix}
            1 & 1 & 1 & 1 & 1 \\
            3 & 2 & 1 & 1 & -3 \\
            0 & 1 & 2 & 2 & 6 \\
            5 & 4 & 3 & 3 & -1 \\
        \end{pmatrix} & \xrightarrow{F_{2,1}(-3),F_{4,1}(-5)}\begin{pmatrix}
            1 & 1 & 1 & 1 & 1 \\
            0 & -1 & -2 & -2 & -6 \\
            0 & 1 & 2 & 2 & 6 \\
            0 & -1 & -2 & -2 & -6 \\
        \end{pmatrix} \\
        & \xrightarrow{F_{2,3}(1),F_{4,3}(1),F_{1,3}(-1)}\begin{pmatrix}
            1 & 0 & -1 & -1 & -5 \\
            0 & 0 & 0 & 0 & 0 \\
            0 & 1 & 2 & 2 & 6 \\
            0 & 0 & 0 & 0 & 0 \\
        \end{pmatrix} \\
        & \xrightarrow{F_{2,3}}\begin{pmatrix}
            1 & 0 & -1 & -1 & -5 \\
            0 & 1 & 2 & 2 & 6 \\
            0 & 0 & 0 & 0 & 0 \\
            0 & 0 & 0 & 0 & 0 \\
        \end{pmatrix}.
    \end{align*}

    分别取 $(x_3,x_4,x_5)=(1,0,0),(0,1,0),(0,0,1)$ 得原方程的一个基础解系为
    \[\begin{pmatrix}
        1 \\ -2 \\ 1 \\ 0 \\ 0
    \end{pmatrix},\quad\begin{pmatrix}
        1 \\ -2 \\ 0 \\ 1 \\ 0
    \end{pmatrix},\quad\begin{pmatrix}
        5 \\ -6 \\ 0 \\ 0 \\ 1
    \end{pmatrix}.\]
\end{solution}
\end{document}